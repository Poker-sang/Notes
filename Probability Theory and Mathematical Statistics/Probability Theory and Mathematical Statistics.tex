\documentclass{article}

\usepackage{ctex}
\usepackage{amsfonts}
\usepackage{amsmath}
\usepackage{amsthm}
\usepackage{graphicx}
\usepackage{float}
\usepackage{hyperref}
\usepackage{mathabx}
\usepackage{datetime}
\usepackage{tabularray}
\usepackage{mathrsfs}
\usepackage{geometry}

\title{概率论与数理统计}
\author{}
\date{\today}

\geometry{a4paper,scale=0.8}

\begin{document}

\hypersetup{
    hidelinks,
    %colorlinks = true,
    allcolors = black,
    %pdfstartview = Fit,
    breaklinks = true
}

\newtheorem{definition}{Definition}[subsubsection]
\newtheorem{theorem}{Theorem}[subsubsection]
\newtheorem{corollary}{Corollary}[theorem]
\renewcommand{\proofname}{\indent\bf Proof}
\numberwithin{equation}{subsection}

\def\e{\mathrm e}
\def\i{\mathrm i}
\def\j{\mathrm j}
\def\d{\mathrm d}
\def\C{\mathrm C}
\def\sr{\mathbb R}
\def\sn{\mathbb N}
\def\snp{\mathbb N^+}
\def\sc{\mathbb C}
\def\sz{\mathbb Z}
\def\impint{\int\limits_{-\infty}^{+\infty}}

\newcommand{\abs}[1]{\left|#1\right|}
\newcommand{\p}[1]{\left(#1\right)}
\newcommand{\br}[1]{\left\{#1\right\}}
\newcommand{\conditionset}[2]{\br{#1|#2}}

\def\pa{P\p{A}}

\begin{titlepage}
    \maketitle
\end{titlepage}

\tableofcontents
\newpage

\section{随机事件及其概率}

\subsection{符号}

\begin{center}
    \begin{tblr}{c|c|c}
        \hline
        名词         & 符号            & 注释                                              \\
        \hline
        随机实验     & $E$             &                                                   \\
        样本点       & $\omega$        &                                                   \\
        样本空间     & $\Omega$        &                                                   \\
        交(积)事件 & $A\cap B$或$AB$ &                                                   \\
        并事件       & $A\cup B$       &                                                   \\
        差事件       & $A-B$           & $\conditionset\omega{\omega\in A;\omega\notin B}$ \\
        互斥事件     &                 & $A\cap B=\emptyset$                               \\
        对立事件     & $\overline A$   & $\Omega-A$                                        \\
        概率         & $\pa$           &                                                   \\
        \hline
    \end{tblr}
\end{center}

\subsection{条件概率}

已知$A$事件发生,发生$B$事件的概率($\pa>0$)

\[P\p{B|A}=\frac{P\p{AB}}{\pa}\]

\subsection{乘法公式}

$\pa>0$

\[P\p{AB}=\pa P\p{B|A}\]

\subsection{古典概型}

\[\pa=\frac{A\text{所含样本点个数}}{\Omega\text{样本点个数}}\]

\subsection{几何概型}

\[\pa=\frac{A\text{的几何测度}}{\Omega\text{的几何测度}}\]

\subsection{完备事件组}

\[\bigcup A_i=\Omega;A_i\cap A_j=\emptyset\]

\subsection{全概率公式}

$A_i$:完备事件组

\[P\p{B}=\sum P\p{A_i}P\p{B|A_i}\]

\subsubsection{全概率条件公式}

\[P\p{C|B}=\sum P\p{A_i|B}P\p{C|A_iB}\]

\subsection{贝叶斯公式}

$j\in\snp$

\[P\p{A_j|B}=\frac{P\p{A_jB}}{P\p{B}}=\frac{P\p{A_j}P\p{B|A_j}}{\sum P\p{A_i}P\p{B|A_i}}\]

\subsection{独立事件}

\[\begin{aligned}
        A,B\text{相互独立} & \Longleftrightarrow P\p{AB}=\pa P\p{B}              \\
                           & \Longleftrightarrow P\p{A|B}=\pa                       \\
                           & \Longleftrightarrow\overline A,B\text{相互独立}           \\
                           & \Longleftrightarrow A,\overline B\text{相互独立}          \\
                           & \Longleftrightarrow\overline A,\overline B\text{相互独立} \\
                           & \Longleftrightarrow P\p{A|B}=P\p{A|\overline B}     \\
    \end{aligned}\]

\[\br{A_i}\text{相互独立}\Longleftrightarrow P\p{\bigcap A_i}=\prod P\p{A_i}\]

\subsection{伯努利概型}

定义:
\begin{enumerate}
    \item 每次试验对应样本空间相同
    \item 各次试验结果相对独立
    \item 只考虑两种结果
\end{enumerate}

$n$重伯努利试验中,$A$事件恰好发生$k$次的概率为$C_n^kp^k\p{1-p}^{n-k}$

\section{离散型随机变量及其分布律}

\subsection{两点分布}

$X\sim B\p{1,p}$

\[P\br{X=k}=p^k{\p{1-p}}^{1-k}\p{k\in\br{0,1}}\]

\subsection{二项(伯努利)分布}

$X\sim B\p{n,p}$

\[P\br{X=k}=C_n^kp^k{\p{1-p}}^{n-k}\p{n\geqslant k\in\sn}\]

\subsection{泊松分布}

$X\sim P\p{\lambda}$

\[P\br{X=k}=\frac{\lambda^k}{k!}\e^{-\lambda}\p{k\in\sn}\]

\subsection{几何分布}

$X\sim G\p{p}$

$k$:前$k-1$次都失败,第$k$次成功

\[P\br{X=k}={\p{1-p}}^{k-1}p\p{p\in\snp}\]

\subsection{超几何分布}

$X\sim H\p{M,N,n}$

$N$:总样本数$N>1$

$n$:抽取样本数$n\leqslant N$

$M$:指定样本数$M\leqslant N$

$k$:抽到指定样本数$k\in\sn\cap\left[\max\br{0,M+n-N},\min\br{M,n}\right]$

\[P\br{X=k}=\frac{C_M^kC_{N-M}^{n-k}}{C_N^n}\]


\end{document}
