\documentclass{article}

\usepackage{ctex}
\usepackage{amsfonts}
\usepackage{amsmath}
\usepackage{amsthm}
\usepackage{graphicx}
\usepackage{float}
\usepackage{hyperref}
\usepackage{mathabx}
\usepackage{datetime}
\usepackage{tabularray}
\usepackage{mathrsfs}
\usepackage{geometry}

\title{概率论与数理统计}
\author{}
\date{\today}

\geometry{a4paper,scale=0.8}

\begin{document}

\hypersetup{
    hidelinks,
    %colorlinks = true,
    allcolors = black,
    %pdfstartview = Fit,
    breaklinks = true
}

\newtheorem{definition}{Definition}[subsubsection]
\newtheorem{theorem}{Theorem}[subsubsection]
\newtheorem{corollary}{Corollary}[theorem]
\renewcommand{\proofname}{\indent\bf Proof}
\numberwithin{equation}{subsection}

\def\e{\mathrm e}
\def\i{\mathrm i}
\def\j{\mathrm j}
\def\d{\mathrm d}
\def\C{\mathrm C}
\def\sr{\mathbb R}
\def\sn{\mathbb N}
\def\snp{\mathbb N^+}
\def\sc{\mathbb C}
\def\sz{\mathbb Z}
\def\impint{\int\limits_{-\infty}^{+\infty}}

\newcommand{\abs}[1]{\left|#1\right|}
\newcommand{\p}[1]{\left(#1\right)}
\newcommand{\br}[1]{\left\{#1\right\}}
\newcommand{\conditionset}[2]{\br{#1|#2}}
\newcommand{\cov}[2]{\mathrm{Cov}\p{#1,#2}}

\def\pa{P\p{A}}
\def\sumin{\sum_{i=1}^n}

\begin{titlepage}
    \maketitle
\end{titlepage}

\tableofcontents
\newpage

\section{随机事件及其概率}

\subsection{符号}

\begin{center}
    \begin{tblr}{c|c|c}
        \hline
        名词         & 符号            & 注释                                                   \\
        \hline
        随机实验     & $E$             &                                                        \\
        样本点       & $\omega$        &                                                        \\
        样本空间     & $\Omega$        &                                                        \\
        交(积)事件 & $A\cap B$或$AB$ & $\conditionset\omega{\omega\in A\wedge\omega\in B}$    \\
        并事件       & $A\cup B$       & $\conditionset\omega{\omega\in A\vee\omega\in B}$      \\
        差事件       & $A-B$           & $\conditionset\omega{\omega\in A\wedge\omega\notin B}$ \\
        互斥事件     &                 & $A\cap B=\emptyset$                                    \\
        对立事件     & $\overline A$   & $\Omega-A$                                             \\
        概率         & $\pa$           &                                                        \\
        \hline
    \end{tblr}
\end{center}

\subsection{条件概率}

已知$A$事件发生,发生$B$事件的概率($\pa>0$)

\[P\p{B|A}=\frac{P\p{AB}}{\pa}\]

\subsection{乘法公式}

$\pa>0$

\[P\p{AB}=\pa P\p{B|A}\]

\subsection{古典概型}

\[\pa=\frac{A\text{所含样本点个数}}{\Omega\text{样本点个数}}\]

\subsection{几何概型}

\[\pa=\frac{A\text{的几何测度}}{\Omega\text{的几何测度}}\]

\subsection{完备事件组}

\[\bigcup A_i=\Omega;A_i\cap A_j=\emptyset\]

\subsection{全概率公式}

\begin{enumerate}
    \item [$\br{A_i}$] 完备事件组
\end{enumerate}

\[P\p{B}=\sum P\p{A_i}P\p{B|A_i}\]

\subsubsection{全概率条件公式}

\[P\p{C|B}=\sum P\p{A_i|B}P\p{C|A_iB}\]

\subsection{贝叶斯公式}

\[P\p{A_j|B}=\frac{P\p{A_jB}}{P\p{B}}=\frac{P\p{A_j}P\p{B|A_j}}{\sum P\p{A_i}P\p{B|A_i}}\]

\subsection{独立事件}

\[\begin{aligned}
        A,B\text{相互独立} & \Longleftrightarrow P\p{AB}=\pa P\p{B}                    \\
                           & \Longleftrightarrow P\p{A|B}=\pa                          \\
                           & \Longleftrightarrow\overline A,B\text{相互独立}           \\
                           & \Longleftrightarrow A,\overline B\text{相互独立}          \\
                           & \Longleftrightarrow\overline A,\overline B\text{相互独立} \\
                           & \Longleftrightarrow P\p{A|B}=P\p{A|\overline B}           \\
    \end{aligned}\]

\[\br{A_i}\text{相互独立}\Longleftrightarrow P\p{\bigcap A_i}=\prod P\p{A_i}\]

\subsection{伯努利概型}

定义:
\begin{enumerate}
    \item 每次试验对应样本空间相同
    \item 各次试验结果相对独立
    \item 只考虑两种结果
\end{enumerate}

$n$重伯努利试验中,$A$事件恰好发生$k$次的概率为$C_n^kp^k\p{1-p}^{n-k}$

\section{连续型随机变量及其分布函数}

\subsection{密度函数(概率密度)}

$f\p{x}$、$f\p{x,y}$等

\paragraph{定义}

\[F\p{x}=P\br{X=x}\p{x\in\sr}\]

\paragraph{性质}

\[f\p{x}\geqslant0\]

\[\impint f\p{x}\d x=1\]

\subsection{分布函数}

$F\p{x}$、$F\p{x,y}$等

\paragraph{定义}

\[F\p{x}=P\br{X\leqslant x}=\int\limits_{-\infty}^xf\p{t}\d t\p{x\in\sr}\]

\[F\p{x,y}=P\br{X\leqslant x,Y\leqslant y}=\int\limits_{-\infty}^x\int\limits_{-\infty}^yf\p{u,v}\d u\d v\p{x,y\in\sr}\]

\paragraph{性质}

\[F\p{x}\geqslant0\]

\[\lim_{x\to+\infty}F\p{x}=1\]

\section{分布}

\subsection{边缘分布}

\[F_X\p{x}=\lim_{y\to+\infty}F\p{x,y}=F\p{x,+\infty}\p{x\in\sr}\]

\[F_Y\p{y}=\lim_{x\to+\infty}F\p{x,y}=F\p{+\infty,y}\p{y\in\sr}\]

\subsubsection{边缘分布律}

\[P\br{X=x_i}=\sum_jP\br{X=x_i,Y=y_j}\]

\[P\br{Y=y_j}=\sum_iP\br{X=x_i,Y=y_j}\]

\subsubsection{边缘密度函数}

\[f_X\p{x}=\impint f\p{x,y}\d y\p{x\in\sr}\]

\[f_Y\p{y}=\impint f\p{x,y}\d x\p{y\in\sr}\]

\subsection{条件分布}

\[F_{X|Y}\p{x|y}=P\br{X\leqslant x|Y=y}\p{x\in\sr}\]

\[F_{Y|X}\p{y|x}=P\br{Y\leqslant y|X=x}\p{y\in\sr}\]

\subsubsection{条件分布律}

\[P\br{X=x_i|Y=y_j}=\frac{P\br{X=x_i,Y=y_j}}{\sum\limits_jP\br{X=x_i,Y=y_j}}\]

\[P\br{Y=y_j|X=x_i}=\frac{P\br{X=x_i,Y=y_j}}{\sum\limits_iP\br{X=x_i,Y=y_j}}\]

\subsubsection{条件分布密度函数}

\[f_{X|Y}\p{x|y}=\frac{f\p{x,y}}{f_Y\p{y}}\p{x\in\sr}\]

\[f_{Y|X}\p{y|x}=\frac{f\p{x,y}}{f_X\p{x}}\p{y\in\sr}\]

\subsection{独立性}

\paragraph{充要条件}

\[F\p{x,y}=F_X\p{x}F_Y\p{y}\]

\section{一维离散型随机变量及其分布律}

以下都有

\[p\in\p{0,1}\]

\subsection{两点分布}

$X\sim B\p{1,p}$

\begin{enumerate}
    \item [$k$] $k\in\br{0,1}$
\end{enumerate}

\[P\br{X=k}=p^k{\p{1-p}}^{1-k}\]

\subsection{二项(伯努利)分布}

$X\sim B\p{n,p}$

\begin{enumerate}
    \item [$n$] $n\in\snp$
    \item [$k$] $n\geqslant k\in\sn$
\end{enumerate}

\[P\br{X=k}=C_n^kp^k{\p{1-p}}^{n-k}\]

\subsection{泊松分布}

$X\sim P\p{\lambda}$

\begin{enumerate}
    \item [$\lambda$] $\lambda>0$
    \item [$k$] $k\in\sn$
\end{enumerate}

\[P\br{X=k}=\frac{\lambda^k}{k!}\e^{-\lambda}\]

\subsubsection{泊松定理}

$n$重伯努利试验中,事件发生概率$p_n\in\p{0,1}$与试验次数有关,
若$\lim\limits_{n\to\infty}np_n=\lambda>0$,则

\[\lim_{n\to\infty}C_n^kp_n^k\p{1-p_n}^{n-k}
    =\dfrac{\lambda^k}{k!}\e^{-\lambda}\]

\subsection{几何分布}

$X\sim G\p{p}$

\begin{enumerate}
    \item [$k$] 前$k-1$次都失败,第$k$次成功$k\in\snp$
\end{enumerate}

\[P\br{X=k}={\p{1-p}}^{k-1}p\]

\subsection{超几何分布}

$X\sim H\p{M,N,n}$

\begin{enumerate}
    \item [$N$] 总样本数$N>1$
    \item [$n$] 抽取样本数$n\leqslant N$
    \item [$M$] 指定样本数$M\leqslant N$
    \item [$k$] 抽到指定样本数$k\in\sn\cap\left[\max\br{0,M+n-N},\min\br{M,n}\right]$
\end{enumerate}

\[P\br{X=k}=\frac{C_M^kC_{N-M}^{n-k}}{C_N^n}\]

\section{一维连续型随机变量及其密度函数}

\subsection{均匀分布}

$X\sim U\left[a,b\right]$

\[f\p{x}=\left\{\begin{aligned}
         & \frac1{b-a} &  & x\in\left[a,b\right]               \\
         & 0           &  & x\in\p{-\infty,a}\cup\p{b,+\infty}
    \end{aligned}\right.\]

\subsection{指数分布}

$X\sim E\p{\lambda}$

\[f\p{x}=\left\{\begin{aligned}
         & \lambda\e^{-\lambda x} &  & x\geqslant0 \\
         & 0                      &  & x<0
    \end{aligned}\right.\]

\subsection{正态(高斯)分布}

$X\sim N\p{\mu,\sigma^2}$

\begin{enumerate}
    \item [$\mu$] $\mu\in\sr$ 期望
    \item [$\sigma$] $\sigma>0$ 标准差
\end{enumerate}

\[f\p{x}=\frac1{\sqrt{2\pi}\sigma}\exp\p{-\frac{{\p{x-\mu}}^2}{2\sigma^2}}\]

\subsubsection{标准正态分布}

$X\sim N\p{0,1}$

\[\varphi\p{x}=\frac1{\sqrt{2\pi}}\exp\p{-\frac{x^2}2}\]

\subsubsection{$3\sigma$原则}

\[\left\{\begin{aligned}
        P\br{\abs{X-\mu}<\sigma}=0.6826  \\
        P\br{\abs{X-\mu}<2\sigma}=0.9544 \\
        P\br{\abs{X-\mu}<3\sigma}=0.9974
    \end{aligned}\right.\]

\subsection{换元}

\subsubsection{离散型}

\[Y=g\p{X}\]

\subsubsection{连续型}

\[\left\{\begin{aligned}
         & y=g\p{x} &  & x\in\sr \\
         & Y=g\p{X} &  &
    \end{aligned}\right.\]

\[f_Y\p{y}=\left\{\begin{aligned}
         & f_X\p{g^{-1}\p{y}}\abs{g^{-1\prime}\p{y}} &  & y\in\p{\min\br{g\p{-\infty},g\p{+\infty}},\max\br{g\p{-\infty},g\p{+\infty}}} \\
         & 0                                         &  & \text{其他}
    \end{aligned}\right.\]

\section{二维连续型随机变量及其密度函数}

\subsection{均匀分布}

$\p{X,Y}\sim U\p{D}$

\[f\p{x,y}=\left\{\begin{aligned}
         & \frac1{S_D} &  & \p{x,y}\in D    \\
         & 0           &  & \p{x,y}\notin D \\
    \end{aligned}\right.\]

\subsection{正态分布}

$\p{X,Y}\sim N\p{\mu_X,\mu_Y,\sigma_X^2,\sigma_Y^2,\rho}$

\begin{enumerate}
    \item [$\mu$] $\mu\in\sr$ 期望
    \item [$\sigma$] $\sigma>0$ 标准差
    \item [$\rho$] $\rho\in\p{-1,1}$,相关系数($\rho=0$时$X$、$Y$独立)
\end{enumerate}

\[f\p{x,y}=\frac1{2\pi\sigma_X\sigma_Y\sqrt{1-\rho^2}}
    \exp\left[-\frac1{2\p{1-\rho^2}}
        \p{\frac{{\p{x-\mu_X}}^2}{\sigma_X^2}
            -2\rho\frac{\p{x-\mu_X}\p{y-\mu_Y}}{\sigma_X\sigma_Y}
            +\frac{{\p{y-\mu_Y}}^2}{\sigma_Y^2}}\right]\]

若$X$、$Y$独立

\[\left.\begin{aligned}
        X\sim N\p{\mu_X,\sigma_X^2} \\
        Y\sim N\p{\mu_Y,\sigma_Y^2}
    \end{aligned}\right\}\implies
    \p{X,Y}\sim N\p{\mu_X,\mu_Y,\sigma_X^2,\sigma_Y^2,0}\]


\[Z=aX+bY\sim N\p{a\mu_X+b\mu_Y,a^2\sigma_X^2+b^2\sigma_Y^2}\]

\subsection{换元}

$Z=X+Y$

\[f_Z\p{z}
    =\int_{-\infty}^{+\infty}f\p{x,z-x}\d x
    =\int_{-\infty}^{+\infty}f\p{z-y,y}\d y\]

\subsubsection{卷积定理}

若上式$X$、$Y$独立

\[f_Z\p{z}
    =\impint f_X\p{x}f_Y\p{z-x}\d x
    =\impint f_X\p{z-y}f_Y\p{y}\d y
    =f_X\p{z}\ast f_Y\p{z}\]

\subsubsection{最大最小值}

\[M=\max\br{X,Y},N=\min\br{X,Y}\]

则

\[F_M=F_XF_Y\]

\[1-F_N=\p{1-F_X}\p{1-F_Y}\]

\[f_M=F^\prime_M=f_XF_Y+f_YF_X\]

\[f_N=F^\prime_N=f_X\p{1-F_Y}+f_Y\p{1-F_X}\]

\section{数字特征}

\subsection{期望}

\subsubsection{离散型}

\[E\p{X}=\sum x_ip_i\]

\subsubsection{连续型}

\[E\p{X}=\impint xf\p{x}\d x\]

\subsubsection{换元}

\paragraph{一维}

\[Y=g\p{X}\]

\subparagraph{离散}

\[E\p{Y}=\sum g\p{x_i}p_i\]

\subparagraph{连续}

\[E\p{Y}=\impint g\p{x}f\p{x}\d x\]

\paragraph{二维}

\[Z=g\p{X,Y}\]

\subparagraph{离散}

\[E\p{Z}=\sum\sum g\p{x_i,y_j}p_{ij}\]

\subparagraph{连续}

\[E\p{Z}=\impint\impint g\p{x,y}f\p{x,y}\d x\d y\]

\subsubsection{性质}

\paragraph{线性}

\[E\p{kX+c}=kE\p{X}+c\]

\[E\p{X\pm Y}=E\p{X}\pm E\p{Y}\]

\paragraph{独立}

若$X$、$Y$独立

\[E\p{XY}=E\p{X}E\p{Y}\]

\paragraph{平均}

\begin{enumerate}
    \item [$\bar X$]$\dfrac1n\sum X_i$
\end{enumerate}

\[E\p{\bar X}=\frac1n\sum E\p{X_i}\]

\subsection{方差}

\[D\p{X}=E\p{X^2}-{\p{E\p{X}}}^2\]

\subsubsection{标准差}

\[\sqrt{D\p{X}}\]

\subsubsection{离散型}

\[D\p{X}=\sum{\p{x_i-E\p{X}}}^2p_i\]

\subsubsection{连续型}

\[D\p{X}=\impint{\p{x_i-E\p{X}}}^2f\p{x}\d x\]

\subsubsection{性质}

\paragraph{线性}

\[D\p{kX+c}=k^2D\p{X}\]

\paragraph{独立}

若$X$、$Y$独立

\[D\p{aX+bY}=a^2D\p{X}+b^2D\p{Y}\]

\[D\p{XY}=D\p{X}D\p{Y}\]

\paragraph{平均}

\begin{enumerate}
    \item [$\bar X$]$\dfrac1n\sum X_i$
\end{enumerate}

\[D\p{\bar X}=\frac1{n^2}\sum D\p{X_i}\]

\subsection{标准化随机变量}

\[X^\ast=\frac{X-E\p{X}}{\sqrt{D\p{X}}}\]

\subsection{常见分布期望方差}

\[\begin{tblr}{colspec={c|c|c},hlines}
        X\sim F\p{X}      & E\p{X}         & D\p{X}                               \\
        B\p{n,p}          & np             & np\p{1-p}                            \\
        P\p{\lambda}      & \lambda        & \lambda                              \\
        G\p{p}            & \dfrac1p       & \dfrac{\p{1-p}}{p^2}                 \\
        H\p{M,N,n}        & \dfrac{nM}N    & \dfrac{nM\p{N-n}\p{N-M}}{N^2\p{N-1}} \\
        U\left[a,b\right] & \dfrac{a+b}2   & \dfrac{{\p{b-a}}^2}{12}              \\
        E\p{\lambda}      & \dfrac1\lambda & \dfrac1{\lambda^2}                   \\
        N\p{\mu,\sigma^2} & \mu            & \sigma^2                             \\
    \end{tblr}\]

\subsection{协方差}

\[\cov XY=E\left[\p{X-E\p{X}}\p{Y-E\p{Y}}\right]\]

\subsection{性质}

\[D\p{X\pm Y}=D\p{X}+D\p{Y}\pm2\cov XY\]

\[\cov XY=E\p{XY}-E\p{X}E\p{Y}\]

\[\cov XX=D\p{X}\]

\paragraph{交换律}

\[\cov XY=\cov YX\]

\paragraph{线性}

\[\cov Xc=0\]

\[\cov{\sum a_iX_i}{\sum b_jY_j}=\sum\sum a_ib_j\cov{X_i}{Y_j}\]

\subsection{(线性)相关系数}

\subsubsection{(线性)均方误差}

用$aX+b$去拟合$Y$

\begin{enumerate}
    \item [$e\p{a,b}$] 越小表明线性关系越强,越大越弱
    \item [$e\p{a_0,b_0}$] 最小均方误差
    \item [$\p{a_0,b_0}$] 驻点
    \item [$a_0$] $\dfrac{\cov XY}{D\p{X}}$
    \item [$b_0$] $E\p{Y}-a_0E\p{X}$
\end{enumerate}

\[e\p{a,b}=E\left[{\p{Y-\p{aX+b}}}^2\right]\]

\[e\p{a_0,b_0}=\p{1-\rho^2_{XY}}D\p{Y}\]

\subsubsection{定义}

\[\rho_{XY}=\frac{\cov XY}{\sqrt{D\p{X}}\sqrt{D\p{Y}}}\]

\subsubsection{性质}

\[\rho\in\left[-1,1\right]\]

\[\abs{\rho_{XY}}=1\Longleftrightarrow P\br{Y=aX+b}=1\p{a\neq0}
    \left\{\begin{aligned}
        a>0 &  & \text{(正相关)} & \rho_{XY}= & 1  \\
        a<0 &  & \text{(负相关)} & \rho_{XY}= & -1 \\
    \end{aligned}\right.\]

\[\rho_{\p{aX}\p{bY}}=\frac{ab}{\abs{ab}}\rho_{XY}\]

\subsubsection{不(线性)相关}

\[\begin{aligned}
                            & \rho=0                    \\
        \Longleftrightarrow & \cov XY=0                 \\
        \Longleftrightarrow & E\p{XY}=E\p{X}E\p{Y}      \\
        \Longleftrightarrow & D\p{X\pm Y}=D\p{X}+D\p{Y}
    \end{aligned}\]

独立$\implies$不相关

\section{统计量}

\subsection{样本均值}

\[\bar X=\frac1n\sumin X_i\]

\subsection{样本方差}

\[S^2=\frac1{n-1}\sumin{\p{X_i-\bar X}}^2=\frac1{n-1}\p{\sumin X_i^2-n\bar X^2}\]

\subsection{样本标准差}

\[S=\sqrt{S^2}\]

\subsection{其他}

\[E\p{\bar X}=\mu\]

\[D\p{\bar X}=\frac{\sigma^2}n\]

\[E\p{S^2}=\sigma^2\]

\subsection{矩}

\begin{enumerate}
    \item [$k$] $\in\snp$
\end{enumerate}

\subsubsection{k阶原点矩}

\begin{enumerate}
    \item [$k=1$] 1阶原点矩$=E\p{X}$
\end{enumerate}

\[E\p{X^k}\]

\subsubsection{k阶中心矩}

\[E\left[{\p{X-E\p{X}}}^k\right]\]

\begin{enumerate}
    \item [$k=1$] 1阶中心矩$=0$
    \item [$k=2$] 2阶中心矩$=D\p{X}$
\end{enumerate}

\subsubsection{k+l阶混合原点矩}

\[E\p{X^kY^l}\]

\subsubsection{k+l阶混合中心矩}

\[E\left[{\p{X-E\p{X}}}^k{\p{Y-E\p{Y}}}^l\right]\]

\begin{enumerate}
    \item [$k=l=1$] 1+1阶混合中心矩$=\cov XY$
\end{enumerate}

\section{抽样分布}

$x\sim\Gamma\p{\alpha,\beta}$

\subsection{伽马分布}

\subsubsection{伽马函数}

\[\Gamma\p{x}=\int\limits_0^{+\infty}t^{x-1}\e^{-t}\d t\p{x>0}\]

\subsubsection{密度函数}

\[f\p{x,\alpha,\beta}=
    \left\{\begin{aligned}
         & \frac{\beta^\alpha}{\Gamma\p{\alpha}}x^{\alpha-1}\e^{-\beta x} &  & x>0         \\
         & 0                                                              &  & x\leqslant0
    \end{aligned}\right.\]

\subsubsection{性质}

\paragraph{再生性}

\begin{enumerate}
    \item [$X_1$] $\sim\Gamma\p{\alpha_1,\beta}$
    \item [$X_2$] $\sim\Gamma\p{\alpha_2,\beta}$
    \item [$X_3$] $\sim\Gamma\p{\alpha_1+\alpha_2,\beta}$
\end{enumerate}

\[X_1+X_2=X_3\]

\subsection{卡方分布}

$\chi^2\sim\chi^2\p{n}=\Gamma\p{\dfrac n2,\dfrac12}$

\begin{enumerate}
    \item [$n$] 自由度
    \item [$X_i$] $\sim N\p{0,1}$
\end{enumerate}

\[\chi^2=\sumin X_i^2\]

\subsubsection{非中心的卡方分布}

\begin{enumerate}
    \item [$X_i$] $\sim N\p{\mu_i,1}$
    \item [$\delta$] 非中心参数
\end{enumerate}

\[\delta=\sqrt{\sumin\mu_i^2}\]

\[\chi_{n,\delta}^2=\sumin X_i^2\]

\subsubsection{密度函数}

\[f\p{x,n}=
    \left\{\begin{aligned}
         & \frac{x^{\frac n2-1}\e^{-\frac x2}}{2^{\frac n2}\Gamma\p{\dfrac n2}} &  & x>0         \\
         & 0                                                                    &  & x\leqslant0
    \end{aligned}\right.\]

\subsubsection{性质}

\paragraph{再生性}

\begin{enumerate}
    \item [$\chi_1^2$] $\sim\chi^2\p{m}$
    \item [$\chi_2^2$] $\sim\chi^2\p{n}$
    \item [$\chi_3^2$] $\sim\chi^2\p{m+n}$
\end{enumerate}

\[\chi_1^2+\chi_2^2=\chi_3^2\]

\subsection{t分布}

$T\sim t\p{n}$

\begin{enumerate}
    \item [$n$] 自由度
    \item [$X$] $\sim N\p{0,1}$
    \item [$Y$] $\sim\chi^2\p{n}$
\end{enumerate}

\[T=\frac{X}{\sqrt{Y/n}}\]

\subsubsection{密度函数}

\[f\p{x,n}=\frac{\Gamma\p{\dfrac{n+1}2}}{\sqrt{n\pi}\Gamma\p{\dfrac n2}}{\p{1+\frac{x^2}n}}^{-\dfrac{n+1}2}\p{x\in\sr}\]

\subsubsection{性质}

$n=1$时,为柯西分布

$n$充分大时,为标准正态分布

\subsection{F分布}

$F\sim F\p{m,n}$

\begin{enumerate}
    \item [$m$] 第一自由度
    \item [$n$] 第二自由度
    \item [$X$] $\sim\chi^2\p{m}$
    \item [$Y$] $\sim\chi^2\p{n}$
\end{enumerate}

\[F=\frac{X/m}{Y/n}\]

\subsubsection{密度函数}

\[f\p{x,m,n}=
    \left\{\begin{aligned}
         & \frac{\Gamma\p{\dfrac{m+n}2}}{\Gamma\p{\dfrac m2}\Gamma\p{\dfrac n2}}{\p{\frac mn}}^{\dfrac m2}x^{\frac m2-1}{\p{1+\frac mnx}}^{-\dfrac{m+n}2} &  & x>0         \\
         & 0                                                                                                                                              &  & x\leqslant0 \\
    \end{aligned}\right.\]

\subsubsection{性质}

若$T\sim t\p{n}$,则$T^2\sim F\p{1,n}$

若$F\sim F\p{m,n}$,则$\dfrac1F\sim F\p{n,m}$

\subsection{常见分布期望方差}

\[\begin{tblr}{colspec={c|c|c|c},hlines}
        X      & \sim F\p{X}            & E\p{X}               & D\p{X}                                            \\
        X      & \Gamma\p{\alpha,\beta} & \dfrac\alpha\beta    & \dfrac\alpha{\beta^2}                             \\
        \chi^2 & \chi^2\p{n}            & n                    & 2n                                                \\
        T      & t\p{n}                 & 0\p{n>1}             & \dfrac n{n-2}\p{n>2}                              \\
        F      & F\p{m,n}               & \dfrac n{n-2}\p{n>2} & \dfrac{2n^2\p{m+n-2}}{m{\p{n-2}}^2\p{n-4}}\p{n>4} \\
    \end{tblr}\]

\subsection{上侧分位点}

$x_p$

\[P\br{X\geqslant x_p}=p\p{p\in\p{0,1}}\]

\subsection{单正态总体样本均值和样本方差的分布}

\[\begin{aligned}
        \chi^2=\frac{\displaystyle\sumin{\p{X_i-\mu}}^2}{\sigma^2}                                & \sim\chi^2\p{n}                                                     \\
        \chi^2=\frac{\p{n-1}S^2}{\sigma^2}=\frac{\displaystyle\sumin{\p{X_i-\bar X}}^2}{\sigma^2} & \sim\chi^2\p{n-1}              & \text{且}\bar X,S^2\text{相互独立} \\
        \bar X                                                                                    & \sim N\p{\mu,\frac{\sigma^2}n}                                      \\
        U=\frac{\bar X-\mu}{\sigma/\sqrt n}                                                       & \sim N\p{0,1}                                                       \\
        T=\frac{\bar X-\mu}{S/\sqrt n}                                                            & \sim t\p{n-1}                                                       \\
    \end{aligned}\]

\section{参数估计}

\subsection{矩估计}

\subsubsection{基本思想}

样本矩代替总体矩,建立$k$个方程,从中解出$k$个未知参数的矩估计量(低阶矩优先)

\begin{enumerate}
    \item [$k=1$] 一般采用$\bar X=E\p{X}$
    \item [$k=2$]
          一般采用$\left\{\begin{aligned}
                  \bar X                          & =E\p{X} \\
                  \frac1n\sumin{\p{X_i-\bar X}}^2 & =D\p{X} \\
              \end{aligned}\right.$

          也可以用$\left\{\begin{aligned}
                  \bar X              & =E\p{X}   \\
                  \frac1n\sumin X_i^2 & =E\p{X^2} \\
              \end{aligned}\right.$
\end{enumerate}

\subsection{极大似然估计}

\begin{enumerate}
    \item [$p\p{x,\theta}$] 分布律或者密度函数
\end{enumerate}

\[L\p{x_1,x_2,\cdots;\theta}=\prod_{i=1}^np\p{x_i,\theta}\]

\[L\p{x_1,x_2,\cdots;\hat\theta}=\max_{\theta\in\Theta}\br{L\p{x_1,x_2,\cdots;\theta}}\]

一般解法:求$\dfrac{\d\p{\ln L\p{\theta}}}{\d\theta}=0$的驻点

\subsection{估计量评价标准}

\subsubsection{均方误差}

\[E\left[\p{\hat\theta-\theta}^2\right]=D\p{\hat\theta}+{\p{\theta-E\p{\hat\theta}}}^2\]

\subsubsection{无偏性}

\paragraph{无偏估计}$E\p{\hat\theta}=\theta$

否则为有偏估计

\paragraph{渐进无偏估计}$\lim\limits_{n\to\infty}E\p{\hat\theta}=\theta$

\paragraph{性质}

$\bar X$是$\mu$的无偏估计,即$E\p{\bar X}=\mu$

$S^2$是$\sigma^2$的无偏估计,即$E\p{S^2}=\sigma^2$

\subsubsection{有效性}

$\hat\theta_1,\hat\theta_2$均为$\theta$的无偏估计,均方误差准则就是方差准则,
若$D\p{\hat\theta_1}<D\p{\hat\theta_2}$,称$\hat\theta_1$比$\hat\theta_2$有效

\end{document}
