\documentclass{article}

\usepackage{ctex}
\usepackage{amsfonts}
\usepackage{amsmath}
\usepackage{amsthm}
\usepackage{graphicx}
\usepackage{float}
\usepackage{hyperref}
\usepackage{mathabx}
\usepackage{datetime}
\usepackage{tabularray}
\usepackage{mathrsfs}
\usepackage{geometry}

\title{概率论与数理统计}
\author{}
\date{\today}

\geometry{a4paper,scale=0.8}

\begin{document}

\hypersetup{
    hidelinks,
    %colorlinks = true,
    allcolors = black,
    %pdfstartview = Fit,
    breaklinks = true
}

\newtheorem{definition}{Definition}[subsubsection]
\newtheorem{theorem}{Theorem}[subsubsection]
\newtheorem{corollary}{Corollary}[theorem]
\renewcommand{\proofname}{\indent\bf Proof}
\numberwithin{equation}{subsection}

\def\e{\mathrm e}
\def\i{\mathrm i}
\def\j{\mathrm j}
\def\d{\mathrm d}
\def\C{\mathrm C}
\def\sr{\mathbb R}
\def\sn{\mathbb N}
\def\snp{\mathbb N^+}
\def\sc{\mathbb C}
\def\sz{\mathbb Z}
\def\impint{\int\limits_{-\infty}^{+\infty}}

\newcommand{\abs}[1]{\left|#1\right|}
\newcommand{\p}[1]{\left(#1\right)}
\newcommand{\br}[1]{\left\{#1\right\}}
\newcommand{\conditionset}[2]{\br{#1|#2}}

\def\pa{P\p{A}}

\begin{titlepage}
    \maketitle
\end{titlepage}

\tableofcontents
\newpage

\section{随机事件及其概率}

\subsection{符号}

\begin{center}
    \begin{tblr}{c|c|c}
        \hline
        名词         & 符号            & 注释                                              \\
        \hline
        随机实验     & $E$             &                                                   \\
        样本点       & $\omega$        &                                                   \\
        样本空间     & $\Omega$        &                                                   \\
        交(积)事件 & $A\cap B$或$AB$ &                                                   \\
        并事件       & $A\cup B$       &                                                   \\
        差事件       & $A-B$           & $\conditionset\omega{\omega\in A;\omega\notin B}$ \\
        互斥事件     &                 & $A\cap B=\emptyset$                               \\
        对立事件     & $\overline A$   & $\Omega-A$                                        \\
        概率         & $\pa$           &                                                   \\
        \hline
    \end{tblr}
\end{center}

\subsection{条件概率}

已知$A$事件发生,发生$B$事件的概率($\pa>0$)

\[P\p{B|A}=\frac{P\p{AB}}{\pa}\]

\subsection{乘法公式}

$\pa>0$

\[P\p{AB}=\pa P\p{B|A}\]

\subsection{古典概型}

\[\pa=\frac{A\text{所含样本点个数}}{\Omega\text{样本点个数}}\]

\subsection{几何概型}

\[\pa=\frac{A\text{的几何测度}}{\Omega\text{的几何测度}}\]

\subsection{完备事件组}

\[\bigcup A_i=\Omega;A_i\cap A_j=\emptyset\]

\subsection{全概率公式}

\begin{enumerate}
    \item [$A_i$] 完备事件组
\end{enumerate}

\[P\p{B}=\sum P\p{A_i}P\p{B|A_i}\]

\subsubsection{全概率条件公式}

\[P\p{C|B}=\sum P\p{A_i|B}P\p{C|A_iB}\]

\subsection{贝叶斯公式}

\begin{enumerate}
    \item [$j$] $j\in\snp$
\end{enumerate}


\[P\p{A_j|B}=\frac{P\p{A_jB}}{P\p{B}}=\frac{P\p{A_j}P\p{B|A_j}}{\sum P\p{A_i}P\p{B|A_i}}\]

\subsection{独立事件}

\[\begin{aligned}
        A,B\text{相互独立} & \Longleftrightarrow P\p{AB}=\pa P\p{B}                    \\
                           & \Longleftrightarrow P\p{A|B}=\pa                          \\
                           & \Longleftrightarrow\overline A,B\text{相互独立}           \\
                           & \Longleftrightarrow A,\overline B\text{相互独立}          \\
                           & \Longleftrightarrow\overline A,\overline B\text{相互独立} \\
                           & \Longleftrightarrow P\p{A|B}=P\p{A|\overline B}           \\
    \end{aligned}\]

\[\br{A_i}\text{相互独立}\Longleftrightarrow P\p{\bigcap A_i}=\prod P\p{A_i}\]

\subsection{伯努利概型}

定义:
\begin{enumerate}
    \item 每次试验对应样本空间相同
    \item 各次试验结果相对独立
    \item 只考虑两种结果
\end{enumerate}

$n$重伯努利试验中,$A$事件恰好发生$k$次的概率为$C_n^kp^k\p{1-p}^{n-k}$

\section{连续型随机变量及其分布函数}

\subsection{密度函数(概率密度)}

$f\p{x}$、$f\p{x,y}$等

\paragraph{定义}

\[F\p{x}=P\br{X=x}\p{x\in\sr}\]

\paragraph{性质}

\[f\p{x}\geqslant0\]

\[\int\limits_{-\infty}^{+\infty}f\p{x}\d x=1\]

\subsection{分布函数}

$F\p{x}$、$F\p{x,y}$等

\paragraph{定义}

\[F\p{x}=P\br{X\leqslant x}=\int\limits_{-\infty}^xf\p{t}\d t\p{x\in\sr}\]

\[F\p{x,y}=P\br{X\leqslant x,Y\leqslant y}=\int\limits_{-\infty}^x\int\limits_{-\infty}^yf\p{u,v}\d u\d v\p{x,y\in\sr}\]

\paragraph{性质}

\[F\p{x}\geqslant0\]

\[\lim_{x\to+\infty}F\p{x}=1\]

\section{一维离散型随机变量及其分布律}

以下都有

\[p\in\p{0,1}\]

\subsection{两点分布}

$X\sim B\p{1,p}$

\begin{enumerate}
    \item [$k$] $k\in\br{0,1}$
\end{enumerate}

\[P\br{X=k}=p^k{\p{1-p}}^{1-k}\]

\subsection{二项(伯努利)分布}

$X\sim B\p{n,p}$

\begin{enumerate}
    \item [$n$] $n\in\snp$
    \item [$k$] $n\geqslant k\in\sn$
\end{enumerate}

\[P\br{X=k}=C_n^kp^k{\p{1-p}}^{n-k}\]

\subsection{泊松分布}

$X\sim P\p{\lambda}$

\begin{enumerate}
    \item [$\lambda$] $\lambda>0$
    \item [$k$] $k\in\sn$
\end{enumerate}

\[P\br{X=k}=\frac{\lambda^k}{k!}\e^{-\lambda}\]

\subsubsection{泊松定理}

$n$重伯努利试验中,事件发生概率$p_n\in\p{0,1}$与试验次数有关,
若$\lim\limits_{n\to\infty}np_n=\lambda>0$,则

\[\lim_{n\to\infty}C_n^kp_n^k\p{1-p_n}^{n-k}
    =\dfrac{\lambda^k}{k!}\e^{-\lambda}\]

\subsection{几何分布}

$X\sim G\p{p}$

\begin{enumerate}
    \item [$k$] 前$k-1$次都失败,第$k$次成功$p\in\snp$
\end{enumerate}

\[P\br{X=k}={\p{1-p}}^{k-1}p\]

\subsection{超几何分布}

$X\sim H\p{M,N,n}$

\begin{enumerate}
    \item [$N$] 总样本数$N>1$
    \item [$n$] 抽取样本数$n\leqslant N$
    \item [$M$] 指定样本数$M\leqslant N$
    \item [$k$] 抽到指定样本数$k\in\sn\cap\left[\max\br{0,M+n-N},\min\br{M,n}\right]$
\end{enumerate}

\[P\br{X=k}=\frac{C_M^kC_{N-M}^{n-k}}{C_N^n}\]

\section{一维连续型随机变量及其密度函数}

\subsection{均匀分布}

$X\sim U\left[a,b\right]$

\[f\p{x}=\left\{\begin{aligned}
         & \frac1{b-a} &  & x\in\left[a,b\right]               \\
         & 0           &  & x\in\p{-\infty,a}\cup\p{b,+\infty}
    \end{aligned}\right.\]

\subsection{指数分布}

$X\sim E\p{\lambda}$

\[f\p{x}=\left\{\begin{aligned}
         & \lambda\e^{-\lambda x} &  & x\geqslant0 \\
         & 0                      &  & x<0
    \end{aligned}\right.\]

\subsection{正态(高斯)分布}

$X\sim N\p{\mu,\sigma^2}$

\begin{enumerate}
    \item [$\mu$] $\mu\in\sr$
    \item [$\sigma$] $\sigma>0$
\end{enumerate}

\[f\p{x}=\frac1{\sqrt{2\pi}\sigma}\exp\p{-\frac{{\p{x-\mu}}^2}{2\sigma^2}}\]

\subsubsection{标准正态分布}

$X\sim N\p{0,1}$

\[\varphi\p{x}=\frac1{\sqrt{2\pi}}\exp\p{-\frac{x^2}2}\]

\subsubsection{$3\sigma$原则}

\[\left\{\begin{aligned}
        P\br{\abs{X-\mu}<\sigma}=0.6826  \\
        P\br{\abs{X-\mu}<2\sigma}=0.9544 \\
        P\br{\abs{X-\mu}<3\sigma}=0.9974
    \end{aligned}\right.\]

\subsection{换元}

\subsubsection{离散型}

\[Y=g\p{X}\]

\subsubsection{连续型}

\[\left\{\begin{aligned}
         & y=g\p{x} &  & x\in\sr \\
         & Y=g\p{X} &  &
    \end{aligned}\right.\]

\[f_Y\p{y}=\left\{\begin{aligned}
         & f_X\p{g^{-1}\p{y}}\abs{h^\prime\p{y}} &  & y\in\p{\min\br{g\p{-\infty},g\p{+\infty}},\max\br{g\p{-\infty},g\p{+\infty}}} \\
         & 0                                     &  & \text{其他}
    \end{aligned}\right.\]

\section{二维连续型随机变量及其密度函数}

\subsection{均匀分布}

$\p{X,Y}\sim U\p{D}$

\[f\p{x,y}=\left\{\begin{aligned}
         & \frac1{S_D} &  & \p{x,y}\in D    \\
         & 0           &  & \p{x,y}\notin D \\
    \end{aligned}\right.\]

\subsection{正态分布}

$\p{X,Y}\sim N\p{\mu_X,\mu_Y,\sigma_X^2,\sigma_Y^2,\rho}$

\[f\p{x,y}=\frac1{2\pi\sigma_X\sigma_Y\sqrt{1-\rho^2}}
    \exp\left[-\frac1{2\p{1-\rho^2}}
        \p{\frac{{\p{x-\mu_X}}^2}{\sigma_X^2}
            -2\rho\frac{\p{x-\mu_X}\p{y-\mu_Y}}{\sigma_X\sigma_Y}
            +\frac{{\p{y-\mu_Y}}^2}{\sigma_Y^2}}\right]\]

\section{边缘分布}

\[F_X\p{x}=\lim_{y\to+\infty}F\p{x,y}=F\p{x,+\infty}\p{x\in\sr}\]

\[F_Y\p{y}=\lim_{x\to+\infty}F\p{x,y}=F\p{+\infty,y}\p{y\in\sr}\]

\subsection{边缘分布律}

\[P\br{X=x_i}=\sum_jP\br{X=x_i,Y=y_j}\]

\[P\br{Y=y_j}=\sum_iP\br{X=x_i,Y=y_j}\]

\subsection{边缘密度函数}

\[f_X\p{x}=\int\limits_{-\infty}^{+\infty}f\p{x,y}\d y\p{x\in\sr}\]

\[f_Y\p{y}=\int\limits_{-\infty}^{+\infty}f\p{x,y}\d x\p{y\in\sr}\]

\section{条件分布}

\[F_{X|Y}\p{x|y}=P\br{X\leqslant x|Y=y}\p{x\in\sr}\]

\[F_{Y|X}\p{y|x}=P\br{Y\leqslant y|X=x}\p{y\in\sr}\]

\subsection{条件分布律}

\[P\br{X=x_i|Y=y_j}=\frac{P\br{X=x_i,Y=y_j}}{\sum\limits_jP\br{X=x_i,Y=y_j}}\]

\[P\br{Y=y_j|X=x_i}=\frac{P\br{X=x_i,Y=y_j}}{\sum\limits_iP\br{X=x_i,Y=y_j}}\]

\subsection{条件分布密度函数}

\[f_{X|Y}\p{x|y}=\frac{f\p{x,y}}{f_Y\p{y}}\p{x\in\sr}\]

\[f_{Y|X}\p{y|x}=\frac{f\p{x,y}}{f_X\p{x}}\p{y\in\sr}\]


\end{document}
