\documentclass{article}

\usepackage{ctex}
\usepackage{amsfonts}
\usepackage{amsmath}
\usepackage{amsthm}
\usepackage{graphicx}
\usepackage{float}
\usepackage{hyperref}
\usepackage{mathabx}
\usepackage{datetime}
\usepackage{tabularray}
\usepackage{mathrsfs}
\usepackage{geometry}

\title{概率论与数理统计}
\author{}
\date{\today}

\geometry{a4paper,scale=0.8}

\begin{document}

\hypersetup{
    hidelinks,
    %colorlinks = true,
    allcolors = black,
    %pdfstartview = Fit,
    breaklinks = true
}

\newtheorem{definition}{Definition}[subsection]
\newtheorem{theorem}{Theorem}[subsection]
\newtheorem{corollary}{Corollary}[theorem]
\renewcommand{\proofname}{\indent\bf Proof}
\numberwithin{equation}{section}

\def\e{\mathrm e}
\def\i{\mathrm i}
\def\j{\mathrm j}
\def\d{\mathrm d}
\def\C{\mathrm C}
\def\sr{\mathbb R}
\def\sn{\mathbb N}
\def\snp{\mathbb N^+}
\def\sc{\mathbb C}
\def\sz{\mathbb Z}
\def\impint{\int\limits_{-\infty}^{+\infty}}

\newcommand{\abs}[1]{\left|#1\right|}
\newcommand{\pare}[1]{\left(#1\right)}
\newcommand{\conditionset}[2]{\left\{#1|#2\right\}}

\def\pa{P\pare{A}}

\begin{titlepage}
    \maketitle
\end{titlepage}

\tableofcontents
\newpage

\part{随机事件及其概率}

\section{符号}

\begin{center}
    \begin{tblr}{c|c|c}
        \hline
        名词         & 符号            & 注释                                              \\
        \hline
        随机实验     & $E$             &                                                   \\
        样本点       & $\omega$        &                                                   \\
        样本空间     & $\Omega$        &                                                   \\
        交(积)事件 & $A\cap B$或$AB$ &                                                   \\
        并事件       & $A\cup B$       &                                                   \\
        差事件       & $A-B$           & $\conditionset\omega{\omega\in A;\omega\notin B}$ \\
        互斥事件     &                 & $A\cap B=\emptyset$                               \\
        对立事件     & $\overline A$   & $\Omega-A$                                        \\
        概率         & $\pa$           &                                                   \\
        \hline
    \end{tblr}
\end{center}

\section{条件概率}

已知$A$事件发生,发生$B$事件的概率($\pa>0$)

\[P\pare{B|A}=\frac{P\pare{AB}}{\pa}\]

\section{乘法公式}

$\pa>0$

\[P\pare{AB}=\pa P\pare{B|A}\]

\section{古典概型}

\[\pa=\frac{A\text{所含样本点个数}}{\Omega\text{样本点个数}}\]

\section{几何概型}

\[\pa=\frac{A\text{的几何测度}}{\Omega\text{的几何测度}}\]

\section{超几何分布}

$r$:抽到指定样本数

$n$:抽取样本数

$M$:指定样本数

$N$:总样本数

\[\pa=\frac{C_M^rC_{N-M}^{n-r}}{C_N^n}\]

\section{完备事件组}

\[\bigcup A_i=\Omega;A_i\cap A_j=\emptyset\]

\section{全概率公式}

$A_i$:完备事件组

\[P\pare{B}=\sum P\pare{A_i}P\pare{B|A_i}\]

\paragraph{全概率条件公式}

\[P\pare{C|B}=\sum P\pare{A_i|B}P\pare{C|A_iB}\]

\section{贝叶斯公式}

$j\in\snp$

\[P\pare{A_j|B}=\frac{P\pare{A_jB}}{P\pare{B}}=\frac{P\pare{A_j}P\pare{B|A_j}}{\sum P\pare{A_i}P\pare{B|A_i}}\]

\end{document}
