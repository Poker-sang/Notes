\documentclass{article}

\usepackage{ctex}
\usepackage{amsfonts}
\usepackage{amsmath}
\usepackage{amsthm}
\usepackage{graphicx}
\usepackage{float}
\usepackage{hyperref}
\usepackage{mathabx}
\usepackage{datetime}
\usepackage{tabularray}
\usepackage{mathrsfs}
\usepackage{geometry}

\title{汇编语言程序设计}
\author{}
\date{\today}

\geometry{a4paper,scale=0.8}

\begin{document}

\hypersetup{
    hidelinks,
    %colorlinks=true,
    allcolors=black,
    %pdfstartview=Fit,
    breaklinks=true
}

\newtheorem{definition}{Definition}[subsection]
\newtheorem{theorem}{Theorem}[subsection]
\newtheorem{corollary}{Corollary}[theorem]
\renewcommand{\proofname}{\indent\bf Proof}
\numberwithin{equation}{section}

\def\e{\mathrm e}
\def\i{\mathrm i}
\def\j{\mathrm j}
\def\d{\mathrm d}
\def\C{\mathrm C}
\def\sr{\mathbb R}
\def\sn{\mathbb N}
\def\snp{\mathbb N^+}
\def\sc{\mathbb C}
\def\sz{\mathbb Z}
\def\impint{\int\limits_{-\infty}^{+\infty}}

\newcommand{\abs}[1]{\left|#1\right|}
\newcommand{\pare}[1]{\left(#1\right)}

\begin{titlepage}
    \maketitle
\end{titlepage}

\tableofcontents
\newpage

\part{寄存器}

\section{通用寄存器}

\subsection{数据寄存器}

\subsubsection{累加寄存器 AX (Accumulator)}

可分为AH和AL两个8位寄存器

\subsubsection{基址寄存器 BX (Base)}

可分为BH和BL两个8位寄存器


\subsubsection{计数寄存器 CX (Count)}

可分为CH和CL两个8位寄存器


\subsubsection{数据寄存器 DX (Data)}

可分为DH和DL两个8位寄存器


\subsection{指针寄存器\label{Pointer}}

与SS(\ref{SS})联合使用

\subsubsection{栈指针寄存器 SP (Stack Pointer)}

栈顶的偏移地址

\subsubsection{基指针寄存器 BP (Base Pointer)}

数据在堆栈段中的基地址

\subsection{变址寄存器}

\subsubsection{源变址寄存器 SI (Source Index)}

\subsubsection{目的变址寄存器 DI (Destination Index)}

\section{专用寄存器}

\subsection{指令指针寄存器 IP (Instruction Pointer)\label{IP}}

代码段中指令的偏移地址,与CS(\ref{CS})联合使用

\subsection{标志寄存器 FLAG}

\section{段寄存器}

用来确定该段在内存中的起始地址。用途特定,不可分开使用。段的长度不超过$2^{16}=64K$

\subsection{代码段寄存器 CS (Code Segment)\label{CS}}

与IP(\ref{IP})联合使用

\subsection{数据段寄存器 DS (Data Segment)}

\subsection{堆栈段寄存器 SS (Stack Segment)\label{SS}}

与指针寄存器(\ref{Pointer})联合使用

\subsection{附加段寄存器 ES (Extra Segment)}

\part{标志位}

DEBUG模式下的表示

\begin{center}
    \begin{tblr}{colspec={c|c|c},hlines}
        标志位 & 1  & 0  \\
        OF     & OV & NV \\
        DF     & DN & UP \\
        IF     & EI & DI \\
        SF     & NG & PL \\
        ZF     & ZR & NZ \\
        AF     & AC & NA \\
        PF     & PE & PO \\
        CF     & CY & NC \\
    \end{tblr}
\end{center}

\section{状态标志位}

记录当前运算结果的状态信息,是CPU“自动”完成的

\subsection{0 进位标志 CF(Carry Flag)}

当运算结果的最高有效位有进位(加法)或借位(减法)时,CF=1

\subsection{6 零标志 ZF(Zero Flag)}

若运算结果为0,ZF=1

\subsection{2 奇偶标志 PF(Parity Flag)}

当运算结果最低8位中“1”的个数为偶数时,PF=1

\subsection{4 辅助进位标志 AF(Auxiliary Carry Flag)}

运算时低4位有进位或借位时,AF=1

\subsection{7 符号标志 SF(Sign Flag)}

运算结果最高位为1,SF=1

\subsection{11 溢出标志 OF(Overflow Flag)}

若算术运算的结果有溢出,OF=1(对无符号数而言,OF=1并不意味着结果出错)

\section{控制标志位}

存放控制CPU工作方式的标志信息

\subsection{10 方向标志 DF(Direction Flag)}

用于串操作指令中,控制地址的变化方向:

设置DF=0,串操作的存储器地址自动增加;

设置DF=1,串操作的存储器地址自动减少。

\subsection{9 中断允许标志 IF(Interrupt-enable Flag)}

用于控制外部可屏蔽中断是否可以被处理器响应:

设置IF=1,则允许中断;

设置IF=0,则禁止中断。

\subsection{8 陷阱标志 TF(Trap Flag)}

用于控制处理器是否进入单步操作方式:

设置TF=0,处理器正常工作;

设置TF=1,处理器单步执行指令。

单步执行指令:处理器在每条指令执行结束时,便产生一个编号为1的内部中断

\section{内存操作数寻址方式}

括号内只能有这四种BX、BP、SI、DI四种寄存器

(段寄存器):(基址)+(变址)+(位移量)(三种寻址至少有其一)

(CS或DS或SS或ES:)([BX或BP])([SI或DI])([8bits或16bits])

若未指定段寄存器:

若出现BP,默认操作CS;否则默认操作SS

\subsection{直接寻址方式(direct addressing)}

[8bits或16bits]

\subsection{寄存器间接寻址方式(register indirect)}

[BX或BP或SI或DI]

\subsection{寄存器相对寻址方式(register relative)}

[BX或BP或SI或DI][8bits或16bits]

\subsection{基址变址寻址方式(based indexed)}

[BX或BP][SI或DI]

\subsection{相对基址变址方式(relative based indexed)}

[BX或BP][SI或DI][8bits或16bits]

\end{document}
