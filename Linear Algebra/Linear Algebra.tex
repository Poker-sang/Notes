\documentclass{article}

\usepackage{ctex}
\usepackage{amsfonts}
\usepackage{amsmath}
\usepackage{amsthm}
\usepackage{amssymb}
\usepackage{graphicx}
\usepackage{float}
\usepackage{hyperref}
\usepackage{mathabx}
\usepackage{datetime}
\usepackage{tabularray}
\usepackage{mathrsfs}
\usepackage{geometry}
\usepackage[dvipsnames]{xcolor}

\title{线性代数}
\author{}
\date{\today}

\geometry{a4paper,scale=0.8}

\begin{document}

\hypersetup{
    hidelinks,
    %colorlinks = true,
    allcolors = black,
    %pdfstartview = Fit,
    breaklinks = true
}

\newtheorem{definition}{Definition}[subsection]
\newtheorem{theorem}{Theorem}[subsection]
\newtheorem{corollary}{Corollary}[theorem]
\renewcommand{\proofname}{\indent\bf Proof}
\numberwithin{equation}{section}

\def\v{\boldsymbol v}
\def\w{\boldsymbol w}
\def\ev{\boldsymbol e}
\def\pv{\boldsymbol p}

\newcommand{\abs}[1]{\left|#1\right|}
\newcommand{\p}[1]{\left(#1\right)}

\begin{titlepage}
    \maketitle
\end{titlepage}

\tableofcontents
\newpage

\section{行列式(方阵)}

\begin{definition}
    余子式:$M_{ij}$

    代数余子式:$A_{ij}$
\end{definition}

\[\begin{gathered}
        \det A=\det A^T\\
        \det kA=k^n\det A\\
        \det AB=\det A\det B
    \end{gathered}\]

\[\det A=\sum_ia_{ij}A_{ij}={\textcolor{blue}{\sum_j}}\p{-1}^{i+j}\textcolor{red}{a_{ij}}\textcolor{green}{M_{ij}}\]

\[\begin{vmatrix}
        \textcolor{green}{a_{11}}    & \textcolor{green}{\cdots} & \textcolor{green}{a_{1,j-1}}   & a_{1j}            & \textcolor{green}{a_{1,j+1}}   & \textcolor{green}{\cdots} & \textcolor{green}{a_{1n}}    \\
        \textcolor{green}{\vdots}    & \textcolor{green}{\ddots} & \textcolor{green}{\vdots}      & \vdots            & \textcolor{green}{\vdots}      & \textcolor{green}{\ddots} & \textcolor{green}{\vdots}    \\
        \textcolor{green}{a_{i-1,1}} & \textcolor{green}{\cdots} & \textcolor{green}{a_{i-1,j-1}} & a_{i-1,j}         & \textcolor{green}{a_{i-1,j+1}} & \textcolor{green}{\cdots} & \textcolor{green}{a_{i-1,n}} \\
        \textcolor{blue}{a_{i,1}}    & \textcolor{blue}{\cdots}  & \textcolor{blue}{a_{i,j-1}}    & \color{red}a_{ij} & \textcolor{blue}{a_{i,j+1}}    & \textcolor{blue}{\cdots}  & \textcolor{blue}{a_{i,n}}    \\
        \textcolor{green}{a_{i+1,1}} & \textcolor{green}{\cdots} & \textcolor{green}{a_{i+1,j-1}} & a_{i+1,j}         & \textcolor{green}{a_{i+1,j+1}} & \textcolor{green}{\cdots} & \textcolor{green}{a_{i+1,n}} \\
        \textcolor{green}{\vdots}    & \textcolor{green}{\ddots} & \textcolor{green}{\vdots}      & \vdots            & \textcolor{green}{\vdots}      & \textcolor{green}{\ddots} & \textcolor{green}{\vdots}    \\
        \textcolor{green}{a_{n1}}    & \textcolor{green}{\cdots} & \textcolor{green}{a_{n,j-1}}   & a_{nj}            & \textcolor{green}{a_{n,j+1}}   & \textcolor{green}{\cdots} & \textcolor{green}{a_{nn}}    \\
    \end{vmatrix}\]

\subsection{克拉默法则}

线性方程组:

\[\sum_ix_i{\boldsymbol{\mathrm\alpha}}_i=y{\boldsymbol{\mathrm\beta}}\]

则有增广矩阵$\bar A_{n+1,n}$:

\[\def\av{\boldsymbol{\mathrm\alpha}}
    \bar A=
    \left[A|\boldsymbol{\mathrm\beta}\right]=
    \begin{bmatrix}
        \bigg| & \bigg| & \bigg| & \bigg| & \bigg|                    \\
        \av_1  & \av_2  & \cdots & \av_n  & \boldsymbol{\mathrm\beta} \\
        \bigg| & \bigg| & \bigg| & \bigg| & \bigg|
    \end{bmatrix}\]

线性方程解:

\[\def\av{\boldsymbol{\mathrm\alpha}}
    \def\bv{\boldsymbol{\mathrm\beta}}
    x_i=\frac{\det A_i}{\det A}=\frac{
        \begin{vmatrix}
            \bigg| & \bigg| & \bigg|    & \bigg| & \bigg|    & \bigg| & \bigg| \\
            \av_1  & \cdots & \av_{i-1} & \bv    & \av_{i+1} & \cdots & \av_n  \\
            \bigg| & \bigg| & \bigg|    & \bigg| & \bigg|    & \bigg| & \bigg|
        \end{vmatrix}}{\det A}\]

\subsection{范德蒙德行列式}

\[\begin{vmatrix}
        1         & 1         & 1         & \cdots & 1         \\
        x_1       & x_2       & x_3       & \cdots & x_n       \\
        x_1^2     & x_2^2     & x_3^2     & \cdots & x_n^2     \\
        \vdots    & \vdots    & \vdots    & \ddots & \vdots    \\
        x_1^{n-1} & x_2^{n-1} & x_3^{n-1} & \cdots & x_n^{n-1} \\
    \end{vmatrix}
    =\prod_{1\leqslant j<i\leqslant n}\p{x_i-x_j}\]

\section{旋转矩阵}

\[\begin{bmatrix}
        \cos\theta & -\sin\theta \\
        \sin\theta & \cos\theta
    \end{bmatrix}\]

\section{运算律}

\paragraph{加法交换律}$A+B=B+A$

\paragraph{加法结合律}$\p{A+B}+C=A+\p{B+C}$

\paragraph{减法}$A-B=A+\p{-B}$

\paragraph{数乘}

\[\begin{gathered}
        \p{kl}A=k\p{lA}=l\p{kA}\\
        \p{k+l}A=kA+lA\\
        k\p{A+B}=kA+kB
    \end{gathered}\]

\paragraph{零元}$A+O=A$

\paragraph{幺元}$AE=EA=A$

\paragraph{外积}

\[C_{m,p}=A_{m,n}B_{n,p}\]

\[c_{ij}=\sum_{k=1}^na_{ik}b_{kj}\]

\[\begin{bmatrix}
        \textcolor{red}{c_{11}} & c_{12}                    & \cdots & c_{1p} \\
        c_{21}                  & \textcolor{green}{c_{22}} & \cdots & c_{2p} \\
        \vdots                  & \vdots                    & \ddots & \vdots \\
        c_{m1}                  & c_{m2}                    & \cdots & c_{mp} \\
    \end{bmatrix}=
    \begin{bmatrix}
        \textcolor{red}{a_{11}}   & \textcolor{red}{a_{12}}   & \textcolor{red}{\cdots}   & \textcolor{red}{a_{1n}}   \\
        \textcolor{green}{a_{21}} & \textcolor{green}{a_{22}} & \textcolor{green}{\cdots} & \textcolor{green}{a_{2n}} \\
        \vdots                    & \vdots                    & \ddots                    & \vdots                    \\
        a_{m1}                    & a_{m2}                    & \cdots                    & a_{mn}                    \\
    \end{bmatrix}
    \begin{bmatrix}
        \textcolor{red}{b_{11}} & \textcolor{green}{b_{12}} & \cdots & b_{1p} \\
        \textcolor{red}{b_{21}} & \textcolor{green}{b_{22}} & \cdots & b_{2p} \\
        \textcolor{red}{\vdots} & \textcolor{green}{\vdots} & \ddots & \vdots \\
        \textcolor{red}{b_{n1}} & \textcolor{green}{b_{n2}} & \cdots & b_{np} \\
    \end{bmatrix}\]

\subsection{数幂(方阵)}

\[\begin{gathered}
        A^0=E\\
        A^k=AA^{k-1}\\
        A^kA^l=A^{k+l}\\
        \p{A^k}^l=A^{kl}
    \end{gathered}\]

\subsection{内积}

\begin{definition}
    \[A\cdot B=\sum_i\sum_ja_{ij}b_{ij}\]
\end{definition}

\paragraph{交换律}$A\cdot B=B\cdot A$

\paragraph{数乘}$\p{\lambda A}\cdot B=\lambda\p{A\cdot B}$

\paragraph{分配律}$\p{A+B}\cdot C=A\cdot C+B\cdot C$

\subsubsection{柯西-施瓦茨不等式}

(积和方$\leqslant$方和积)

\[\p{A\cdot B}^2\leqslant\p{A\cdot A}\p{B\cdot B}\]

\section{行(列)矩阵(向量)}

\subsection{范数(模长)}

\[\left\Vert\v\right\Vert=\sqrt{\v\cdot\v}\]

\subsection{内积(点乘)}

\[\begin{aligned}
        \v\cdot\w & =\sum_i v_iw_i\text{(}v_iw_i\text{为向量各元素)}            \\
                  & =\left\Vert\v\right\Vert\left\Vert\w\right\Vert
        \cos\left<\v,\w\right>                                            \\
                  & =\left\{\begin{aligned}
                                 & \v^T\w=\w^T\v &  & \v\text{、}\w\text{为列向量} \\
                                 & \v\w^T=\w\v^T &  & \v\text{、}\w\text{为行向量}
                            \end{aligned}\right.
    \end{aligned}\]

\subsection{外积(叉乘)}

\[\begin{aligned}
        \v\times\w                      & =
        \begin{bmatrix}\v_x\\\v_y\\\v_z\end{bmatrix}\times
        \begin{bmatrix}\w_x\\\w_y\\\w_z\end{bmatrix}=
        \begin{vmatrix}
            \hat\imath & \v_x & \w_x \\
            \hat\jmath & \v_y & \w_y \\
            \hat k     & \v_z & \w_z
        \end{vmatrix}
        \\
        \left\Vert\v\times\w\right\Vert & =
        \left\Vert\v\right\Vert\left\Vert\w\right\Vert
        \sin\left<\v,\w\right>
    \end{aligned}\]

\section{转置}

\begin{definition}
    \[\p{A^T}^T=A\]
\end{definition}

\[\begin{gathered}
        \p{A+B}^T=A^T+B^T\\
        \p{kA}^T=kA^T\\
        \p{AB}^T=B^TA^T\\
        \p{A^k}^T=\p{A^T}^k\\
    \end{gathered}\]

\section{逆(方阵)}

\begin{definition}[经过矩阵$A$变换,变换后的线性空间可以通过$A^{-1}$变换回原线性空间]
    \[AA^{-1}=A^{-1}A=E\]
\end{definition}

\[\begin{gathered}
        \p{A^{-1}}^{-1}=A\\
        \p{kA}^{-1}=\frac1k A^{-1}\\
        \p{AB}^{-1}=B^{-1}A^{-1}\\
        \exists A^{-1}\implies\exists\p{A^T}^{-1} \\
        \p{A^T}^{-1}=\p{A^{-1}}^T\\
        \det A^{-1}=\frac1{\det A}\\
    \end{gathered}\]

若矩阵$A$变换压缩了维度,则无法通过逆矩阵变换回来:

\[\exists A^{-1}\iff r\p{A_n}=n\iff\det A\neq0\]

\section{伴随(方阵)}

\[A^*=\begin{bmatrix}
        A_{11} & A_{21} & \cdots & A_{n1} \\
        A_{12} & A_{22} & \cdots & A_{n2} \\
        \vdots & \vdots & \ddots & \vdots \\
        A_{1n} & A_{2n} & \cdots & A_{nn}
    \end{bmatrix}\]

\[\begin{gathered}
        AA^*=A^*A=\p{\det A}E\\
        \p{kA}^*=k^{n-1}A^*\\
        \det A\neq0\implies A^*=\p{\det A}A^{-1}\\
        \p{A^*}^{-1}=\p{A^{-1}}^*\\
        \det A^*=\p{\det A}^{n-1}
    \end{gathered}\]

\section{分块矩阵}

运算与普通矩阵相同

\section{初等变换(等价)}

\begin{definition}
    行:$r_i$,列:$c_i$

    1. 对换两行(列):$r_i\leftrightarrow r_j$

    2. $k$乘某行(列):$kr_i$或$r_i\times k\p{k\neq0}$

    3. 加某行(列)$k$倍:$r_i+kr_j$
\end{definition}

\paragraph{反身性}$A\cong A$

\paragraph{对称性}$A\cong B\implies B\cong A$

\paragraph{传递性}$A\cong B,B\cong C\implies A\cong C$

若对$A$初等行/列变换,可先对$E$作,即可得到$P/Q$

\[PA=\p{PE}A,AQ=A\p{EQ}\]

初等变换不改变秩,故$P$、$Q$必然满秩/可逆

\[\begin{aligned}
        A\overset{r}\to B & \iff PA=B  \\
        A\overset{c}\to B & \iff AQ=B  \\
        A\to B            & \iff PAQ=B
    \end{aligned}\]

\[\begin{aligned}
        \begin{bmatrix}A&B\end{bmatrix}
         & \overset{r}\to
        \begin{bmatrix}E&A^{-1}B\end{bmatrix} \\
        \begin{bmatrix}A\\B\end{bmatrix}
         & \overset{c}\to\begin
        {bmatrix}B                            \\A^{-1}\end{bmatrix}
    \end{aligned}\]

\section{秩}

\begin{definition}[经过矩阵$A$变换,变换后的线性空间的维度是$r\p{A}$]
    \[r\begin{bmatrix}
            \bigg| & \bigg| & \bigg| & \bigg| \\
            \v_1   & \v_2   & \cdots & \v_n   \\
            \bigg| & \bigg| & \bigg| & \bigg|
        \end{bmatrix}
        =\mathrm{span}\left[\v_1,\v_2,\cdots,\v_n\right]\]
\end{definition}

\[\begin{gathered}
        A\cong B\implies r\p{A}=r\p{B}\\
        \max\left\{r\p{A},r\p{B}\right\}\leqslant r\p{A,B}\leqslant r\p{A}+r\p{B}\\
        r\p{A+B}\leqslant r\p{A}+r\p{B}\\
        r\p{AB}\leqslant\min\{r\p A,r\p B\}\\
        \exists P^{-1},Q^{-1}\implies r\p{A}=r\p{PAQ}\\
        r\p{A_{m,n}}+r\p{B_{n,s}}\leqslant n
    \end{gathered}\]

\[\left\{\begin{aligned}
        r\p{A} & =n   & \implies r\p{A^*}=n \\
        r\p{A} & =n-1 & \implies r\p{A^*}=1 \\
        r\p{A} & <n-1 & \implies r\p{A^*}=0
    \end{aligned}\right.\]

满秩(方阵):$r\p{A_n}=n$

奇异矩阵:不满秩的方阵

非奇异矩阵:满秩方阵

\section{正交矩阵(方阵)}

\begin{definition}[矩阵行(列)向量组两两正交,且都为单位向量]
    \[AA^T=E\]
\end{definition}

\[\begin{gathered}
        A^{-1}=A^T\iff AA^T=A^TA=E\\
        \det A=\pm 1\\
        \left.\begin{aligned}
            AA^T & =E \\
            BB^T & =E
        \end{aligned}\right\}\implies\p{AB}\p{AB}^T=E
    \end{gathered}\]

\section{迹(方阵)}

\[\mathrm{tr}A=\sum_i a_{ii}\]

\section{特征(方阵)}

\begin{definition}
    特征多项式:

    \[f\p{\lambda}=\det\p{A_n-\lambda E}\]

    特征值$\lambda$(特征多项式为$0$的根,包括重根,共$n$个)

    \[f\p{\lambda}=0\]

    特征向量$\boldsymbol p$($1\leqslant\lambda$对应线性无关$\boldsymbol p$数$\leqslant\lambda$重数)

    \[\p{A_n-\lambda E}\boldsymbol p=\boldsymbol 0\]
\end{definition}

\[\begin{gathered}
        \mathrm{tr}A=\sum_i\lambda_i\\
        \det A=\prod_i\lambda_i\\
        \lambda\text{是}A\text{的特征值}\implies
        \left\{\begin{aligned}
             & \frac1\lambda\text{是}A^{-1}\text{的特征值}                        \\
             & \frac{\det A}\lambda\text{是}A^*\text{的特征值}                    \\
             & \sum_{i=0}^ma_i\lambda^i\text{是}\sum_{i=0}^ma_iA^i\text{的特征值}
        \end{aligned}\right.\\
        \begin{aligned}
            f\p{\lambda}
             & =\p{-1}^n\lambda^n+\p{-1}^{n-1}\mathrm{tr}A\cdot\lambda^{n-1}+\cdots+\det A \\
             & \overset{n=2}{=\!=}\lambda^2-\mathrm{tr}A\cdot\lambda+\det A
        \end{aligned}
    \end{gathered}\]

\section{相似(方阵)}

\begin{definition}
    \[P^{-1}AP=B\iff A\sim B\]
\end{definition}

\paragraph{反身性}$A\sim A$

\paragraph{对称性}$A\sim B\implies B\sim A$

\paragraph{传递性}$A\sim B,B\sim C\implies A\sim C$

\[A\sim B\implies
    \left\{\begin{aligned}
         & \det\p{A-\lambda_A E}=\det\p{B-\lambda_B E}
        \implies\left\{\begin{aligned}
                           \lambda_A=\lambda_B \\
                           \mathrm{tr}A=\mathrm{tr}B
                       \end{aligned}\right.        \\
         & \det A=\det B                               \\
         & r\p{A}=r\p{B}                               \\
         & A^{-1}\sim B^{-1}\text{(如果都可逆)}
    \end{aligned}\right.\]

\section{可对角化(方阵)}

\begin{definition}
    \[A_n\sim\Lambda\]
\end{definition}

$r:\lambda$重数

$n-r\p{A-\lambda E}:\lambda$对应线性无关${\boldsymbol p}$数

\[\begin{aligned}
        A\sim\Lambda & \iff n-r\p{A-\lambda E}=r      \\
                     & \iff\text{全体线性无关}\pv\text{数}=n \\
                     & \iff P^{-1}AP=\Lambda
        \left\{\begin{aligned}
                   \Lambda & =
                   \begin{bmatrix}
                \lambda_1 &           &        &           \\
                          & \lambda_2 &        &           \\
                          &           & \ddots &           \\
                          &           &        & \lambda_n
            \end{bmatrix}                      \\
                   P       & =\begin{bmatrix}\pv_1&\pv_2&\cdots&\pv_n\end{bmatrix}
               \end{aligned}\right.
    \end{aligned}\]

\section{对称矩阵(方阵)}

\begin{definition}
    对称矩阵:

    \[A=A^T\]

    反对称矩阵:

    \[A=-A^T\]
\end{definition}

实数范围内(即$A$为实矩阵):

\[\left.\begin{aligned}
        A         & =A^T            \\
        A\pv_1    & =\lambda_1\pv_1 \\
        A\pv_2    & =\lambda_2\pv_2 \\
        \lambda_1 & \neq\lambda_2
    \end{aligned}\right\}
    \implies \pv_1\cdot\pv_2=0\]

实对称矩阵必可对角化,且

\[\text{实对称矩阵}A\implies
    \exists\text{正交矩阵}P,P^{-1}AP=P^TAP=\Lambda\]

\section{合同(方阵)}

\begin{definition}
    \[B=C^TAC,\exists C^{-1}\iff A\simeq B\]
\end{definition}

\paragraph{反身性}$A\simeq A$

\paragraph{对称性}$A\simeq B\implies B\simeq A$

\paragraph{传递性}$A\simeq B,B\simeq C\implies A\simeq C$

\[\begin{gathered}
        A\simeq B\implies r\p{A}=r\p{B}\\
        A\simeq B\iff A,B\text{的特征值中,正、负、零的个数相同}
    \end{gathered}\]

\section{二次型(方阵、对称矩阵)}

\begin{definition}
    \[f=\sum_{i,j=1}^na_{ij}x_ix_j\p{a_{ij}=a_{ji}}
        =\boldsymbol x^TA\boldsymbol x\]
\end{definition}

\subsection{标准型(对角矩阵)}

\[f=\sum_i^na_{ii}x_i^2\]

\subsection{二次型转标准型}

二次型$f_A\simeq$标准型$g_\Lambda$

\subsubsection{正交变换法}

1. 令$\det\p{A-\lambda E}=0$,解得$n$个特征值$\{\lambda_n\}$

2. 令$\p{A-\lambda_iE}\boldsymbol p=\boldsymbol 0$,解得线性无关特征向量组$\{\pv_n\}$

3. 用格拉姆-施密特正交单位化$\p{\ref{Orthogonalization}}$,解得正交单位特征向量组$\{\ev_n\}$

4. 用正交单位特征向量组构建正交矩阵$P=\begin{bmatrix}\ev_1&\ev_2&\cdots&\ev_n\end{bmatrix}$

5. $\boldsymbol x=P\boldsymbol y$即可将$f$化为标准型$g$

\subsubsection{拉格朗日配方法}

1. 先配$x_1$,再依次往后配;配完的变量后面不能再出现

2. 若只有交叉项,没有平方项,则令$\left\{\begin{aligned}
        x_1 & =y_1+y_2 \\
        x_2 & =y_1-y_2 \\
        x_3 & =y_3     \\
            & \vdots   \\
        x_n & =y_n
    \end{aligned}\right.$,替换后按$y$配方

3. 配完后得:$f=k_1\p{\sum\limits_{i=1}^nk_{1i}x_i}^2+
    k_2\p{\sum\limits_{i=2}^nk_{2i}x_i}^2+\cdots+
    k_n\p{k_{n1}x_n}^2$可替换每一个平方项为一个变量$z$,即:$\boldsymbol z=K\boldsymbol x:\left\{\begin{aligned}
        z_1 & =\sum_{i=1}^nk_{1i}x_i \\
        z_2 & =\sum_{i=2}^nk_{2i}x_i \\
            & \vdots                 \\
        z_n & =k_{n1}x_n
    \end{aligned}\right.$,则原二次型已转为标准型$g=k_1z_1^2+k_2z_2^2+\cdots+k_nz_n^2$

4. 作倒代换得$\boldsymbol x=C\boldsymbol z:\left\{\begin{aligned}
        x_1 & =\sum_{i=1}^nc_{1i}x_i \\
        x_2 & =\sum_{i=2}^nc_{2i}x_i \\
            & \vdots                 \\
        x_n & =c_{n1}x_n
    \end{aligned}\right.$,此处$C=K^{-1}$即为$f$变为标准型$g$的变换矩阵

\subsubsection{初等变换法}

$\begin{bmatrix}A\\E\end{bmatrix}\xrightarrow[\text{对}A\text{只作对应行变换}]{\text{对整个初等列变换}}\begin{bmatrix}\Lambda\\C\end{bmatrix}$

\paragraph{对应行变换}

\subparagraph{将$a$列与$b$列交换}将$a$行与$b$行交换

\subparagraph{将$a$列乘以$k$}将$a$行乘以$k$

\subparagraph{将$a$列加到$b$列}将$a$行列加到$b$行

\subsection{规范型}

\begin{definition}[只有对角元素且元素只包含$1$、$-1$和$0$的二次型,称为规范型]
    \[f=\sum_{i=1}^py_i^2-\sum_{i=p+1}^{r\p{A}}y_i^2\]

    \[\text{实二次型矩阵}A\simeq\begin{bmatrix}
            E_p &               &   \\
                & -E_{r\p{A}-p} &   \\
                &               & O
        \end{bmatrix}\]

    其中$p$为$A$正特征值个数(正惯性指数)(重根按重数展开算),即$r\p{A}-p$为负特征值个数(负惯性指数)
\end{definition}

\[A\simeq B\iff A,B\text{惯性指数相同}\]

\section{正定二次型(方阵)}

\begin{definition}[只有正数特征值的二次型]
    \[A_n\simeq E_n\iff A\text{为正定矩阵(正定二次型)}\]
\end{definition}

\[\begin{aligned}
        A\text{正定}\iff & A\text{特征值全为正}                            \\
        \iff           & A\text{正惯性指数}=n                           \\
        \iff           & A\text{各阶顺序主子式}>0                         \\
                       & \begin{vmatrix}
                             a_{11} & a_{12} & \cdots & a_{1i} \\
                             a_{21} & a_{22} & \cdots & a_{2i} \\
                             \vdots & \vdots & \ddots & \vdots \\
                             a_{i1} & a_{i2} & \cdots & a_{ii}
                         \end{vmatrix}_{1\leqslant i\leqslant n}>0 \\
        \implies       & \det A>0
    \end{aligned}\]

\section{向量组}

\subsection{线性相关}

\begin{definition}
    \[\left.\begin{aligned}
            \sum_i a_i\v_i= \textbf{0} \\
            \prod_i a_i\neq 0
        \end{aligned}\right\}
        \iff\v_i\text{线性相关}\]
\end{definition}

\[\begin{vmatrix}
        \bigg| & \bigg| & \bigg| & \bigg| \\
        \v_1   & \v_2   & \cdots & \v_n   \\
        \bigg| & \bigg| & \bigg| & \bigg|
    \end{vmatrix}
    \neq0\iff\v_i\text{线性相关}\]

\subsection{格拉姆-施密特正交单位化}\label{Orthogonalization}

有线性无关组:

\[\v_1,\v_2,\cdots,\v_r\]

则有正交单位向量组:

\[\begin{aligned}
        \w_1  & =\v_1                                                         \\
        \w_2  & =\v_2-\frac{\w_1\cdot\v_2}{\w_1\cdot\w_1}\w_1                 \\
              & \vdots                                                        \\
        \w_r  & =\v_r-\sum_{i=1}^{r-1}\frac{\w_i\cdot\v_r}{\w_i\cdot\w_i}\w_i \\
        \ev_r & =\w_r^0
    \end{aligned}\]

\end{document}
