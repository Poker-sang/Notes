\documentclass{article}

\usepackage{ctex}
\usepackage{amsfonts}
\usepackage{amsmath}
\usepackage{amsthm}
\usepackage{amssymb}
\usepackage{graphicx}
\usepackage{float}
\usepackage{hyperref}
\usepackage{mathabx}
\usepackage{datetime}
\usepackage{tabularray}
\usepackage{mathrsfs}
\usepackage{geometry}
\usepackage{centernot}
\usepackage{nicematrix}
\usepackage{tikz}
\usetikzlibrary{calc}
\usepackage[dvipsnames]{xcolor}

\title{线性代数}
\author{}
\date{\today}

\geometry{a4paper,scale=0.8}

\begin{document}

\hypersetup{
    hidelinks,
    %colorlinks = true,
    allcolors = black,
    %pdfstartview = Fit,
    breaklinks = true
}

\newtheorem{definition}{Definition}[subsection]
\newtheorem{theorem}{Theorem}[subsection]
\newtheorem{corollary}{Corollary}[theorem]
\renewcommand{\proofname}{\indent\bf Proof}
\numberwithin{equation}{section}

\def\sr{\mathbb R}
\def\v{\boldsymbol v}
\def\w{\boldsymbol w}
\def\ev{\boldsymbol e}
\def\pv{\boldsymbol p}
\def\av{\boldsymbol\alpha}
\def\bv{\boldsymbol\beta}
\def\xv{\boldsymbol x}
\def\yv{\boldsymbol y}

\newcommand{\abs}[1]{\left|#1\right|}
\newcommand{\norm}[1]{\left\Vert#1\right\Vert}
\newcommand{\p}[1]{\left(#1\right)}
\newcommand{\B}[1]{\left\{#1\right\}}

\begin{titlepage}
    \maketitle
\end{titlepage}

\tableofcontents
\newpage

\section{行列式(方阵)}

\begin{definition}
    \[\det A=\sum_ia_{ij}A_{ij}={\textcolor{blue}{\sum_j}}\p{-1}^{i+j}\textcolor{red}{a_{ij}}\textcolor{green}{M_{ij}}\]

    \[\begin{vmatrix}
            \textcolor{green}{a_{11}}    & \textcolor{green}{\cdots} & \textcolor{green}{a_{1,j-1}}   & a_{1j}              & \textcolor{green}{a_{1,j+1}}   & \textcolor{green}{\cdots} & \textcolor{green}{a_{1n}}    \\
            \textcolor{green}{\vdots}    & \textcolor{green}{\ddots} & \textcolor{green}{\vdots}      & \vdots              & \textcolor{green}{\vdots}      & \textcolor{green}{\ddots} & \textcolor{green}{\vdots}    \\
            \textcolor{green}{a_{i-1,1}} & \textcolor{green}{\cdots} & \textcolor{green}{a_{i-1,j-1}} & a_{i-1,j}           & \textcolor{green}{a_{i-1,j+1}} & \textcolor{green}{\cdots} & \textcolor{green}{a_{i-1,n}} \\
            \textcolor{blue}{a_{i,1}}    & \textcolor{blue}{\cdots}  & \textcolor{blue}{a_{i,j-1}}    & \color{red}{a_{ij}} & \textcolor{blue}{a_{i,j+1}}    & \textcolor{blue}{\cdots}  & \textcolor{blue}{a_{i,n}}    \\
            \textcolor{green}{a_{i+1,1}} & \textcolor{green}{\cdots} & \textcolor{green}{a_{i+1,j-1}} & a_{i+1,j}           & \textcolor{green}{a_{i+1,j+1}} & \textcolor{green}{\cdots} & \textcolor{green}{a_{i+1,n}} \\
            \textcolor{green}{\vdots}    & \textcolor{green}{\ddots} & \textcolor{green}{\vdots}      & \vdots              & \textcolor{green}{\vdots}      & \textcolor{green}{\ddots} & \textcolor{green}{\vdots}    \\
            \textcolor{green}{a_{n1}}    & \textcolor{green}{\cdots} & \textcolor{green}{a_{n,j-1}}   & a_{nj}              & \textcolor{green}{a_{n,j+1}}   & \textcolor{green}{\cdots} & \textcolor{green}{a_{nn}}    \\
        \end{vmatrix}\]
\end{definition}

\subsection{运算关系}

\[\begin{gathered}
        \det A=\det A^T\\
        \det kA=k^n\det A\\
        \det AB=\det A\det B
    \end{gathered}\]

\subsection{K阶子式}

选取行列式$A$的$K$行$R=\B{r_1,r_2,\cdots,r_K}$

选取行列式$A$的$K$列$C=\B{c_1,c_2,\cdots,c_K}$

将处在选取行列中的元素$\B{a_{ij}\p{i\in R,j\in C}}$组成新的行列式$B_{kk}$,称为行列式$A$的$K$阶子式

\subsection{K阶主子式}

在$K$阶子式基础上,要求选取的行、列序数相同:

选取行列式$A$的$K$行/列$L=\B{l_1,l_2,\cdots,l_K}$

将处在选取行列中的元素$\B{a_{ij}\p{i\in L,j\in L}}$组成新的行列式$B_{kk}$,称为行列式$A$的$K$阶主子式

\subsection{K阶顺序主子式}

在$K$阶子式基础上,只能选取前$K$行、列:

选取行列式$A$的前$K$行/列$L=\B{1,2,\cdots,K}$

将处在选取行列中的元素$\B{a_{ij}\p{i\in L,j\in L}}$组成新的行列式$B_{kk}$,称为行列式$A$的$K$阶顺序主子式

\[\begin{vmatrix}
        a_{11} & a_{12} & \cdots & a_{1K} \\
        a_{21} & a_{22} & \cdots & a_{2K} \\
        \vdots & \vdots & \ddots & \vdots \\
        a_{K1} & a_{K2} & \cdots & a_{KK}
    \end{vmatrix}\]

\subsection{余子式}

原行列式去掉第$i$行第$j$列,得到的行列式称为余子式,记作$M_{ij}$

\[M_{ij}=\begin{vmatrix}
        a_{11}    & \cdots & a_{1,j-1}   & a_{1,j+1}   & \cdots & a_{1n}    \\
        \vdots    & \ddots & \vdots      & \vdots      & \ddots & \vdots    \\
        a_{i-1,1} & \cdots & a_{i-1,j-1} & a_{i-1,j+1} & \cdots & a_{i-1,n} \\
        a_{i+1,1} & \cdots & a_{i+1,j-1} & a_{i+1,j+1} & \cdots & a_{i+1,n} \\
        \vdots    & \ddots & \vdots      & \vdots      & \ddots & \vdots    \\
        a_{n1}    & \cdots & a_{n,j-1}   & a_{n,j+1}   & \cdots & a_{nn}
    \end{vmatrix}\]

\subsection{代数余子式}

余子式乘以$\p{-1}^{i+j}$,记作$A_{ij}$

\[A_{ij}=\p{-1}^{i+j}M_{ij}\]

\subsection{线性}

可将行列式某行(列)分解为两个行列式之和

\[\begin{vmatrix}
        a_{11}  & \cdots & \color{red}{a_{1j}+b_{1j}} & \cdots & a_{1n}  \\
        \vdots  & \ddots & \color{red}{\vdots}        & \ddots & \vdots  \\
        a_{i,1} & \cdots & \color{red}{a_{ij}+b_{ij}} & \cdots & a_{i,n} \\
        \vdots  & \ddots & \color{red}{\vdots}        & \ddots & \vdots  \\
        a_{n1}  & \cdots & \color{red}{a_{nj}+b_{nj}} & \cdots & a_{nn}  \\
    \end{vmatrix}=
    \begin{vmatrix}
        a_{11}  & \cdots & \color{red}{a_{1j}} & \cdots & a_{1n}  \\
        \vdots  & \ddots & \color{red}{\vdots} & \ddots & \vdots  \\
        a_{i,1} & \cdots & \color{red}{a_{ij}} & \cdots & a_{i,n} \\
        \vdots  & \ddots & \color{red}{\vdots} & \ddots & \vdots  \\
        a_{n1}  & \cdots & \color{red}{a_{nj}} & \cdots & a_{nn}  \\
    \end{vmatrix}+
    \begin{vmatrix}
        a_{11}  & \cdots & \color{red}{b_{1j}} & \cdots & a_{1n}  \\
        \vdots  & \ddots & \color{red}{\vdots} & \ddots & \vdots  \\
        a_{i,1} & \cdots & \color{red}{b_{ij}} & \cdots & a_{i,n} \\
        \vdots  & \ddots & \color{red}{\vdots} & \ddots & \vdots  \\
        a_{n1}  & \cdots & \color{red}{b_{nj}} & \cdots & a_{nn}  \\
    \end{vmatrix}\]

\subsection{初等变换}

对行列式进行初等变换$\p{\ref{ElementaryTransformation}}$

\begin{enumerate}
    \item 第一类初等变换
          $\det A\xrightarrow{r_i\leftrightarrow r_j}-\det A^\prime$
    \item 第二类初等变换
          $\det A\xrightarrow{kr_i\rightarrow r_i\p{k\neq0}}k\det A^\prime$
    \item 第三类初等变换
          $\det A\xrightarrow{r_i+kr_j\rightarrow r_i}\det A^\prime$
\end{enumerate}

\subsection{范德蒙德行列式}

\[\begin{vmatrix}
        1         & 1         & 1         & \cdots & 1         \\
        x_1       & x_2       & x_3       & \cdots & x_n       \\
        x_1^2     & x_2^2     & x_3^2     & \cdots & x_n^2     \\
        \vdots    & \vdots    & \vdots    & \ddots & \vdots    \\
        x_1^{n-1} & x_2^{n-1} & x_3^{n-1} & \cdots & x_n^{n-1} \\
    \end{vmatrix}
    =\prod_{1\leqslant j<i\leqslant n}\p{x_i-x_j}\]

\section{运算律}

\subsection{加法}

\paragraph{幺元}$A+O=A$

\paragraph{交换律}$A+B=B+A$

\paragraph{结合律}$\p{A+B}+C=A+\p{B+C}$

\paragraph{减法}$A-B=A+\p{-B}$

\paragraph{数乘}

\[\begin{gathered}
        \p{kl}A=k\p{lA}=l\p{kA}\\
        \p{k+l}A=kA+lA\\
        k\p{A+B}=kA+kB
    \end{gathered}\]

\subsection{内积\label{InnerProduct}}

\begin{definition}
    \[A\cdot B=\sum_i\sum_ja_{ij}b_{ij}\]
\end{definition}

\paragraph{无零元}

\paragraph{无幺元}

\paragraph{无逆元}

\paragraph{数乘}$\p{\lambda A}\cdot B=\lambda\p{A\cdot B}$

\paragraph{交换律}$A\cdot B=B\cdot A$

\paragraph{分配律}$\p{A+B}\cdot C=A\cdot C+B\cdot C$

\paragraph{不满足结合律}$A\cdot\p{B\cdot C}\neq\p{A\cdot B}\cdot C$

\paragraph{不满足消去律}$A\cdot B=A\cdot C\centernot\implies B=C$

\subsubsection{柯西-施瓦茨不等式}

(积和方$\leqslant$方和积)

\[\p{A\cdot B}^2\leqslant\p{A\cdot A}\p{B\cdot B}\]

\subsection{外积\label{OuterProduct}}

\begin{definition}

    \[C_{m,p}=A_{m,n}B_{n,p}\]

    \[c_{ij}=\sum_{k=1}^na_{ik}b_{kj}\]

    \[\begin{bmatrix}
            \textcolor{red}{c_{11}} & c_{12}                    & \cdots & c_{1p} \\
            c_{21}                  & \textcolor{green}{c_{22}} & \cdots & c_{2p} \\
            \vdots                  & \vdots                    & \ddots & \vdots \\
            c_{m1}                  & c_{m2}                    & \cdots & c_{mp} \\
        \end{bmatrix}=
        \begin{bmatrix}
            \textcolor{red}{a_{11}}   & \textcolor{red}{a_{12}}   & \textcolor{red}{\cdots}   & \textcolor{red}{a_{1n}}   \\
            \textcolor{green}{a_{21}} & \textcolor{green}{a_{22}} & \textcolor{green}{\cdots} & \textcolor{green}{a_{2n}} \\
            \vdots                    & \vdots                    & \ddots                    & \vdots                    \\
            a_{m1}                    & a_{m2}                    & \cdots                    & a_{mn}                    \\
        \end{bmatrix}
        \begin{bmatrix}
            \textcolor{red}{b_{11}} & \textcolor{green}{b_{12}} & \cdots & b_{1p} \\
            \textcolor{red}{b_{21}} & \textcolor{green}{b_{22}} & \cdots & b_{2p} \\
            \textcolor{red}{\vdots} & \textcolor{green}{\vdots} & \ddots & \vdots \\
            \textcolor{red}{b_{n1}} & \textcolor{green}{b_{n2}} & \cdots & b_{np} \\
        \end{bmatrix}\]

\end{definition}

\paragraph{零元}$AO=OA=O$

\paragraph{幺元}$AE=EA=A$

\paragraph{无逆元}仅满秩矩阵拥有逆矩阵

\paragraph{数乘}$\p{\lambda A}B=\lambda\p{AB}$

\paragraph{结合律}$A\p{BC}=\p{AB}C$

\paragraph{分配律}$A\p{B+C}=AB+AC$

\paragraph{不满足交换律}$AB\neq BA$

\paragraph{不满足消去律}$AB=AC\centernot\implies B=C$

\subsection{数幂(方阵)}

\begin{definition}

    \[A^k=AA^{k-1}\]

\end{definition}


\[\begin{gathered}
        A^0=E\\
        A^kA^l=A^{k+l}\\
        \p{A^k}^l=A^{kl}
    \end{gathered}\]

\section{行(列)矩阵(向量)}

\subsection{范数(模长)}

\[\norm\v=\sqrt{\v\cdot\v}\]

\subsection{内积(点乘)}

\[\begin{aligned}
        \v\cdot\w & =\sum_i v_iw_i\text{(}v_iw_i\text{为向量各元素)}            \\
                  & =\norm\v\norm\w\cos\left<\v,\w\right>                 \\
                  & =\left\{\begin{aligned}
                                 & \v^T\w=\w^T\v &  & \v\text{、}\w\text{为列向量} \\
                                 & \v\w^T=\w\v^T &  & \v\text{、}\w\text{为行向量}
                            \end{aligned}\right.
    \end{aligned}\]

运算律同内积(\ref{InnerProduct})

\subsection{外积(叉乘)}

\[\begin{gathered}
        \v\times\w=
        \begin{bmatrix}\v_x\\\v_y\\\v_z\end{bmatrix}\times
        \begin{bmatrix}\w_x\\\w_y\\\w_z\end{bmatrix}=
        \begin{vmatrix}
            \hat\imath & v_x & w_x \\
            \hat\jmath & v_y & w_y \\
            \hat k     & v_z & w_z
        \end{vmatrix}\\
        \norm\v\times\norm\w=\norm\v\norm\w\sin\left<\v,\w\right>
    \end{gathered}\]

运算律同外积(\ref{OuterProduct})

\section{转置}

\begin{definition}
    \[\p{A^T}^T=A\]
\end{definition}

\[\begin{gathered}
        \p{A+B}^T=A^T+B^T\\
        \p{kA}^T=kA^T\\
        \p{AB}^T=B^TA^T\\
        \p{A^k}^T=\p{A^T}^k\\
    \end{gathered}\]

\section{逆(方阵)}

\begin{definition}[经过矩阵$A$变换,变换后的线性空间可以通过$A^{-1}$变换回原线性空间]
    \[AA^{-1}=A^{-1}A=E\]
\end{definition}

\[\begin{gathered}
        \p{A^{-1}}^{-1}=A\\
        \p{kA}^{-1}=\frac1k A^{-1}\\
        \p{AB}^{-1}=B^{-1}A^{-1}\\
        \exists A^{-1}\implies\exists\p{A^T}^{-1} \\
        \p{A^T}^{-1}=\p{A^{-1}}^T\\
        \det A^{-1}=\frac1{\det A}\\
    \end{gathered}\]

若矩阵$A$变换压缩了维度,则无法通过逆矩阵变换回来:

\[\exists A^{-1}\iff r\p{A_n}=n\iff\det A\neq0\]

\section{伴随(方阵)}

\[A^*=\begin{bmatrix}
        A_{11} & A_{21} & \cdots & A_{n1} \\
        A_{12} & A_{22} & \cdots & A_{n2} \\
        \vdots & \vdots & \ddots & \vdots \\
        A_{1n} & A_{2n} & \cdots & A_{nn}
    \end{bmatrix}\]

\[\begin{gathered}
        AA^*=A^*A=\p{\det A}E\\
        \p{kA}^*=k^{n-1}A^*\\
        \det A\neq0\implies A^*=\p{\det A}A^{-1}\\
        \p{A^*}^{-1}=\p{A^{-1}}^*\\
        \det A^*=\p{\det A}^{n-1}
    \end{gathered}\]

\section{分块矩阵}

运算与普通矩阵相同

\section{初等变换(等价)\label{ElementaryTransformation}}

行:$r_i$,列:$c_i$

\begin{enumerate}
    \item 对换:第一类初等变换

          对换两行(列):$r_i\leftrightarrow r_j$
    \item 倍乘:第二类初等变换

          $k$乘某行(列):$kr_i\rightarrow r_i\p{k\neq0}$
    \item 倍加:第三类初等变换

          加某行(列)$k$倍:$r_i+kr_j\rightarrow r_i$
\end{enumerate}

\paragraph{反身性}$A\cong A$

\paragraph{对称性}$A\cong B\implies B\cong A$

\paragraph{传递性}$A\cong B,B\cong C\implies A\cong C$

\[A\cong B\iff A,B\text{同型且}r\p{A}=r\p{B}\]

若对$A$初等行变换(行等价),可先对$E$作相同初等行变换,即可得到$P$,再让$A$左乘$P$

若对$A$初等列变换(列等价),可先对$E$作相同初等列变换,即可得到$Q$,再让$A$右乘$Q$

初等变换不改变秩,故$P$、$Q$必然满秩(可逆)

\[\begin{aligned}
        A\to B            & \iff E\to P,Q,          &  & PAQ=B & \iff & A,B\text{同型且}r\p{A}=r\p{B}                                   \\
        A\overset{r}\to B & \iff E\overset{r}\to Q, &  & PA=B  & \iff & A,B\text{同型且}r\p{A}=r\p{B}=r\begin{bmatrix}A\\B\end{bmatrix} \\
        A\overset{c}\to B & \iff E\overset{c}\to Q, &  & AQ=B  & \iff & A,B\text{同型且}r\p{A}=r\p{B}=r\begin{bmatrix}A&B\end{bmatrix}
    \end{aligned}\]

\subsection{初等变换求逆矩阵}

\[\begin{aligned}
        \begin{bmatrix}A&E\end{bmatrix}
         & \overset{r}\to
        \begin{bmatrix}E&A^{-1}\end{bmatrix} \\
        \begin{bmatrix}A\\E\end{bmatrix}
         & \overset{c}\to
        \begin{bmatrix}E\\A^{-1}\end{bmatrix}
    \end{aligned}\]

\subsection{广义初等变换(舒尔公式)}

广义倍加(左行右列):

\begin{enumerate}
    \item 分块矩阵某行左乘一矩阵加到另一行,秩不变
    \item 分块矩阵某列右乘一矩阵加到另一列,秩不变
\end{enumerate}

\section{秩}

\begin{definition}[经过矩阵$A$变换,变换后的线性空间的维度是$r\p{A}$]
    \[r\begin{bmatrix}
            \big| & \big| &        & \big| \\
            \v_1  & \v_2  & \cdots & \v_n  \\
            \big| & \big| &        & \big|
        \end{bmatrix}
        =\dim\mathrm{span}\left[\v_1,\v_2,\cdots,\v_n\right]\]
\end{definition}

\[\begin{gathered}
        A\cong B\implies r\p{A}=r\p{B}\\
        \max\B{r\p{A},r\p{B}}\leqslant r\p{A,B}\leqslant r\p{A}+r\p{B}\\
        r\p{A+B}\leqslant r\p{A}+r\p{B}\\
        r\p{AB}\leqslant\min\B{r\p{A},r\p{B}}\\
        \exists P^{-1},Q^{-1}\implies r\p{A}=r\p{PAQ}\\
        A_{m,n}B_{n,s}=O\implies r\p{A_{m,n}}+r\p{B_{n,s}}\leqslant n
    \end{gathered}\]

\[\left\{\begin{aligned}
        r\p{A} & =n   & \implies r\p{A^*}=n \\
        r\p{A} & =n-1 & \implies r\p{A^*}=1 \\
        r\p{A} & <n-1 & \implies r\p{A^*}=0
    \end{aligned}\right.\]

满秩(方阵):$r\p{A_n}=n$

奇异矩阵:不满秩的方阵

非奇异矩阵:满秩方阵

\section{向量组}

\subsection{线性表示}

\[\w=\sum_{i=1}^nk_i\v_i;k_i\in\sr\]

\subsection{线性相(无)关组}

\subsubsection{线性表示法}

常用于线性相关组判定

向量组$\B{\v_n}$中,存在一个向量可由其他向量线性表示,则为线性相关组

\[\exists\v_m;1\leqslant m\leqslant n\to
    \v_m=\sum_{\substack{i=1 \\ i\neq m}}^n k_i\v_i;k_i\in\sr\]

若不存在该向量则为线性无关组

\subsubsection{齐次方程组法}

常用于线性无关组判定

向量组$\B{\v_n}$中,向量组对应的齐次线性方程组只有零解,则为线性无关组

\[\sum_{i=1}^nk_i\v_i=0;k_i\in\sr\to k_i=0\]

若有非零解则为相关组

\subsection{极大无关组}

向量组$\B{\v_n}$中,存在一个线性无关子集$\B{\w_m}$,
能线性表示向量组内所有向量,则称为向量组的一个极大无关组(不唯一,除非$m=n$)

\[\forall\v_j;1\leqslant j\leqslant n\to
    \v_j=\sum_{i=1}^m k_i\w_i;k_i\in\sr\]

\[r\B{\v_n}=r\B{\w_m}=m\]

\subsubsection{求法}

将向量写为矩阵的列向量形式:

\[\begin{bmatrix}
        \big| & \big| &        & \big| \\
        \v_1  & \v_2  & \cdots & \v_n  \\
        \big| & \big| &        & \big|
    \end{bmatrix}\]

通过初等行变换化为行阶梯矩阵(无需最简):

\[\begin{bNiceMatrix}[baseline=c,extra-margin=.5em]
        \CodeBefore
        \tikz\draw[dotted](1-|1)|-(2-|3)|-(3-|4)|-(4-|6);
        \Body
        \textcolor{red}{\av_1} & \times & \textcolor{red}{\times} & \textcolor{red}{\times} & \times \\
        0                      & 0      & \textcolor{red}{\av_2}  & \textcolor{red}{\times} & \times \\
        0                      & 0      & 0                       & \textcolor{red}{\av_3}  & \times \\
        0                      & 0      & 0                       & 0                       & 0
    \end{bNiceMatrix}\]

台角所在列的集合即为一个极大无关组

\subsection{等价向量组}

列向量组等价相当于矩阵的列等价

\[\B{\v_m}\cong\B{\w_n}\]
\[\iff
    r\B{\v_m}=r\B{\w_n}=
    r\p{\B{\v_m}\cup\B{\w_n}}\]

\subsection{基}

设向量组$\B{\v_n}$为线性无关组,则称其为$\sr^n$的一个基,它可以张成线性空间$\sr^n$

\subsubsection{过渡矩阵}

设有基$\B{\v_n}$和基$\B{\w_n}$,则称其为同一线性空间的两组基

若有:

\[\begin{bmatrix}
        \big| & \big| &        & \big| \\
        \v_1  & \v_2  & \cdots & \v_n  \\
        \big| & \big| &        & \big|
    \end{bmatrix}C=
    \begin{bmatrix}
        \big| & \big| &        & \big| \\
        \w_1  & \w_2  & \cdots & \w_n  \\
        \big| & \big| &        & \big|
    \end{bmatrix}\]

则称$C$为由基$\B{\v_n}$到基$\B{\w_n}$过渡矩阵

\subsection{格拉姆-施密特正交化}\label{Orthogonalization}

有线性无关组:

\[\B{\v_r}\]

则有正交向量组:

\[\begin{aligned}
        \w_1 & =\v_1                                                         \\
        \w_2 & =\v_2-\frac{\w_1\cdot\v_2}{\w_1\cdot\w_1}\w_1                 \\
             & \vdots                                                        \\
        \w_r & =\v_r-\sum_{i=1}^{r-1}\frac{\w_i\cdot\v_r}{\w_i\cdot\w_i}\w_i
    \end{aligned}\]

\subsection{单位化}

\[\ev_i=\w_i^0\]

\section{线性方程组}

\subsection{齐次线性方程组}

\[\def\av{\boldsymbol{\mathrm\alpha}}
    A\xv=
    \begin{bmatrix}
        \big| & \big| &        & \big| \\
        \av_1 & \av_2 & \cdots & \av_n \\
        \big| & \big| &        & \big|
    \end{bmatrix}
    \begin{bmatrix}
        x_1    \\
        x_2    \\
        \vdots \\
        x_n
    \end{bmatrix}=
    \boldsymbol0\]

\subsection{非齐次线性方程组}

\[\def\av{\boldsymbol{\mathrm\alpha}}
    A\xv=
    \begin{bmatrix}
        \big| & \big| &        & \big| \\
        \av_1 & \av_2 & \cdots & \av_n \\
        \big| & \big| &        & \big|
    \end{bmatrix}
    \begin{bmatrix}
        x_1    \\
        x_2    \\
        \vdots \\
        x_n
    \end{bmatrix}=
    \boldsymbol\beta\]

导出组:非齐次线性方程组对应的齐次线性方程组

\subsection{增广矩阵}

\[\bar A=
    \left[A|\boldsymbol{\mathrm\beta}\right]=
    \begin{bNiceMatrix}[baseline=c,extra-margin=.5em]
        \CodeBefore
        \tikz\draw[dashed](1-|5)|-(5-|5);
        \Body
        \big| & \big| &        & \big| & \big| \\
        \av_1 & \av_2 & \cdots & \av_n & \bv   \\
        \big| & \big| &        & \big| & \big|
    \end{bNiceMatrix}\]

\subsection{克拉默法则}

\[x_i=\frac{\det A_i}{\det A}=\frac{
        \begin{vmatrix}
            \big| &        & \big|     & \big| & \big|     &        & \big| \\
            \av_1 & \cdots & \av_{i-1} & \bv   & \av_{i+1} & \cdots & \av_n \\
            \big| &        & \big|     & \big| & \big|     &        & \big|
        \end{vmatrix}}{\det A}\]

\subsection{增广矩阵解(非)齐次方程组}

将方程组写为增广矩阵形式,通过初等行变换化为行最简阶梯型矩阵:

\[\begin{bNiceMatrix}[baseline=c,extra-margin=.5em]
        \CodeBefore
        \tikz\draw[dotted](1-|1)|-(2-|3)|-(3-|4)|-(4-|6);
        \tikz\draw[dashed](1-|6)|-(5-|6);
        \Body
        1 & \textcolor{red}{\alpha_1} & 0 & 0 & \textcolor{red}{\alpha_2} & \beta_1 \\
        0 & 0                         & 1 & 0 & \textcolor{red}{\alpha_3} & \beta_2 \\
        0 & 0                         & 0 & 1 & \textcolor{red}{\alpha_4} & \beta_3 \\
        0 & 0                         & 0 & 0 & 0                         & 0
    \end{bNiceMatrix}\]

其中台角(即值为1)所在列为主变量,其余列为\textcolor{red}{自由变量}

构建方程组解系,其中基础解系数量同\textcolor{red}{自由变量}数,特解1个

对\textcolor{red}{自由变量}取单位阵(每行依次将一个\textcolor{red}{自由变量}设为\textcolor{red}{1},其他自由变量设为\textcolor{red}{0}),
不考虑非齐次项($\bv_i$),向量内积解出主变量;

将\textcolor{red}{自由变量}全取\textcolor{red}{0},考虑非齐次项($\bv_i$),向量内积得到特解:

\[\begin{tblr}{r c c c c c l}
        \boldsymbol\xi_1=[ & -\alpha_1 & \textcolor{red}{1} & 0         & 0         & \textcolor{red}{0} & ]^T \\
        \boldsymbol\xi_2=[ & -\alpha_2 & \textcolor{red}{0} & -\alpha_3 & -\alpha_4 & \textcolor{red}{1} & ]^T \\
        \boldsymbol\eta=[  & \beta_1   & \textcolor{red}{0} & \beta_2   & \beta_3   & \textcolor{red}{0} & ]^T \\
    \end{tblr}\]


\[\xv=\boldsymbol\eta+k_1\boldsymbol\xi_1+k_2\boldsymbol\xi_2;k_1,k_2\in\sr\]

\subsection{秩和解的关系}

\begin{enumerate}
    \item[$n$] 未知数数量(即$A$的列数)
    \item[$r$] 系数矩阵$A$的秩(即$r\p{A}$,显然$r\leqslant n$)
    \item[$\boldsymbol\eta$] 线性方程组特解(齐次方程组中恒为$\boldsymbol0$)
    \item[$\boldsymbol\xi_i$] 线性方程组基础解系
\end{enumerate}

线性方程组通解形式(共$n-r$个基础解系):

\[\xv=\boldsymbol\eta+\sum_{i=1}^{n-r}k_i\boldsymbol\xi_i;k_i\in\sr\]

\begin{itemize}
    \item 当$r+1=r\p{\bar A}$时,表示无解。否则有解情况:
    \item 当$r<n$时,表示有无穷解
    \item 当$r=n$时,表示只有唯一解(即特解)
\end{itemize}

注:齐次方程组中,$\boldsymbol{\mathrm\beta}$恒为$\boldsymbol0$;则显然$r\p{\bar A}=r\p{A}\p{=r}\leqslant n$。
故齐次方程组最少只会有零解(特解),不会无解

\subsection{齐次线性方程组同解充要条件}

\[A\xv=\boldsymbol0,B\xv=\boldsymbol0\text{同解}
    \iff A,B\text{行等价}
    \iff r\p{A}=r\p{B}=r\begin{bmatrix}A\\B\end{bmatrix}\]

\subsection{齐次线性方程组有非零公共解充要条件}

\[A=A_{m,l},B=B_{n,l}\]

\[A\xv=\boldsymbol0\text{基础解系是}\B{\boldsymbol\xi_i},B\xv=\boldsymbol0\text{基础解系是}\B{\boldsymbol\eta_j}\]

\[\begin{aligned}
         & A\xv=\boldsymbol0,B\xv=\boldsymbol0\text{有非零公共解}                       \\
         & \iff\begin{bmatrix}A\\B\end{bmatrix}\xv=\boldsymbol0\text{有非零解}
        \iff r\begin{bmatrix}A\\B\end{bmatrix}<n                                  \\
         & \iff\B{\boldsymbol\xi_i},\B{\boldsymbol\eta_i}\text{线性相关}              \\
         & \iff\B{A\boldsymbol\eta_i}\text{线性相关或}\B{B\boldsymbol\xi_i}\text{线性相关}
    \end{aligned}\]

\section{正交矩阵(方阵)}

\begin{definition}[矩阵行(列)向量组两两正交,且都为单位向量]
    \[AA^T=E\]
\end{definition}

\[\begin{gathered}
        A^{-1}=A^T\iff AA^T=A^TA=E\\
        \det A=\pm 1\\
        \left.\begin{aligned}
            AA^T & =E \\
            BB^T & =E
        \end{aligned}\right\}\implies\p{AB}\p{AB}^T=E
    \end{gathered}\]

\section{迹(方阵)}

\[\mathrm{tr}A=\sum_ia_{ii}\]

\section{特征(方阵)}

\subsection{特征矩阵}

\[A-\lambda E\]

\subsection{特征多项式}

\[f\p{\lambda}=\det\p{A-\lambda E}\]

\subsection{特征值}

特征多项式为$0$的根,重根计算重数,共$n$个

\[f\p{\lambda}=0\]

一个矩阵的不同特征值对应的特征向量一定线性无关;同一特征值的特征向量不一定线性无关

\subsubsection{代数重数}

特征值$\lambda_i$的重根重数

\[f\p{\lambda}=\sum_i\p{\lambda-\lambda_i}^{r_i}\]

$r_i$是特征值$\lambda_i$的代数重数,其中$\sum r_i=n$

\subsubsection{几何重数}

特征值$\lambda_i$对应的线性无关特征向量数

\[r^\prime_i=n-r\p{A-\lambda_iE}\]

$r^\prime_i$是特征值$\lambda_i$的几何重数

几何重数$r^\prime_i\leqslant$代数重数$r_i$

\subsection{特征向量}

\[\p{A-\lambda_iE}\pv_i=\boldsymbol0\]

$\pv_i$是特征值$\lambda_i$的特征向量(不唯一)

\subsubsection{特征空间}

特征值$\lambda_i$对应的特征向量所张成的空间

\[r^\prime_i=\dim\mathrm{span}\left[\pv_1,\pv_2,\cdots,\pv_{r_i}\right]\]

特征空间维度等于几何重数$r^\prime_i$

\subsection{性质}

\[\begin{gathered}
        r\p{A}\geqslant A\text{的非零特征值个数}\\
        \mathrm{tr}A=\sum\lambda_i\\
        \det A=\prod\lambda_i\\
        \begin{aligned}
            \lambda\text{是}A\text{的特征值} & \implies g\p{\lambda}\text{是}g\p{A}\text{的特征值(}g\text{为任意函数)}              \\
                                        & \implies\frac1\lambda\text{是}A^{-1}\text{的特征值}                             \\
                                        & \implies\frac{\det A}\lambda\text{是}A^*\text{的特征值}                         \\
                                        & \implies\sum_ia_i\lambda^{k_i}+b\text{是}\sum_ia_iA^{k_i}+bE\text{的特征值(线性)}
        \end{aligned}\\
        \begin{aligned}
            f\p{\lambda}
             & =\sum_{i=0}^n\p{-1}^i\sum_\mu A_{\mu\mu}^{\p{i}}\lambda^{n-i} \\
             & \overset{n=2}{=\!=}\lambda^2-\mathrm{tr}A\cdot\lambda+\det A
        \end{aligned}
    \end{gathered}\]

\section{相似(方阵)}

\begin{definition}
    \[P^{-1}AP=B\iff A\sim B\]
\end{definition}

\paragraph{反身性}$A\sim A$

\paragraph{对称性}$A\sim B\implies B\sim A$

\paragraph{传递性}$A\sim B;B\sim C\implies A\sim C$


\[\begin{aligned}
        A\sim B & \iff\p{A-\lambda E}\cong\p{B-\lambda E}             \\
                & \implies A^T\sim B^T                                \\
                & \implies A^{-1}\sim B^{-1}\text{(如果都可逆)}            \\
                & \implies A\cong B\implies r\p{A}=r\p{B}             \\
                & \implies\det\p{A-\lambda_A E}=\det\p{B-\lambda_B E}
        \implies\left\{\begin{aligned}
                           \lambda_A    & =\lambda_B    \\
                           \mathrm{tr}A & =\mathrm{tr}B
                       \end{aligned}\right.                   \\
                & \implies\det A=\det B
    \end{aligned}\]

\subsection{相似对角化\label{SimilarityDiagonalization}}

\begin{definition}
    \[A_n\sim\Lambda\]
\end{definition}

$r_i:\lambda_i$代数重数

$r^\prime_i=n-r\p{A-\lambda_iE}:\lambda_i$几何重数

\[\begin{aligned}
        A\sim\Lambda & \iff r^\prime_i=r_i   \\
                     & \iff\sum r^\prime_i=n \\
                     & \iff P^{-1}AP=\Lambda
        \left\{\begin{aligned}
                   \Lambda & =
                   \begin{bmatrix}
                \lambda_1 &           &        &           \\
                          & \lambda_2 &        &           \\
                          &           & \ddots &           \\
                          &           &        & \lambda_n
            \end{bmatrix}                      \\
                   P       & =\begin{bmatrix}\pv_1&\pv_2&\cdots&\pv_n\end{bmatrix}
               \end{aligned}\right.
    \end{aligned}\]

对于线性无关的特征向量组$\B{\pv_n}$,对其格拉姆-施密特正交化$\p{\ref{Orthogonalization}}$并单位化,仍为特征向量

$A\sim\Lambda,B\sim\Lambda\implies A\sim B$

\section{对称矩阵(方阵)}

\begin{definition}
    对称矩阵:

    \[A=A^T\]

    反对称矩阵:

    \[A=-A^T\]
\end{definition}

实数范围内(即$A$为实矩阵):

实对称矩阵的不同特征值对应的特征向量一定正交

\[\left.\begin{aligned}
        A         & =A^T            \\
        A\pv_1    & =\lambda_1\pv_1 \\
        A\pv_2    & =\lambda_2\pv_2 \\
        \lambda_1 & \neq\lambda_2
    \end{aligned}\right\}
    \implies \pv_1\cdot\pv_2=0\]

实对称矩阵必可相似对角化,且若相似必定合同(反之不一定成立)

\[\text{实对称矩阵}A\implies
    \exists\text{正交矩阵}P,P^{-1}AP=P^TAP=\Lambda\]

\subsection{*谱分解定理}

$A$为实对称矩阵,$\ev_i$为$\lambda_i$对应单位特征向量

\[A=\sum\lambda_i\ev_i\ev^T_i\]

\section{合同(方阵)}

\begin{definition}
    \[B=C^TAC;\exists C^{-1}\iff A\simeq B\]
\end{definition}

\paragraph{反身性}$A\simeq A$

\paragraph{对称性}$A\simeq B\implies B\simeq A$

\paragraph{传递性}$A\simeq B;B\simeq C\implies A\simeq C$

\[\begin{aligned}
        A\simeq B & \implies A^T\simeq B^T                     \\
                  & \implies A^{-1}\simeq B^{-1}\text{(如果都可逆)} \\
                  & \implies A\cong B\implies r\p{A}=r\p{B}    \\
                  & \implies A,B\text{对称性相同(都对称或非对称)}
    \end{aligned}\]

\subsection{西尔维斯特惯性定理}

\[\text{实对称矩阵}A,B;A\simeq B\iff A,B\text{的特征值中,正、负、零的个数相同(正负惯性指数相同)}\]

\section{二次型(方阵、对称矩阵)}

\begin{definition}
    \[f=\sum_{i,j=1}^na_{ij}x_ix_j\p{a_{ij}=a_{ji}}
        =\xv^TA\xv\]
\end{definition}

\subsection{标准型(对角矩阵)}

\[f=\sum_{i=1}^na_{ii}x_i^2\]

\subsection{二次型转标准型(合同对角化)}

二次型$f_A\simeq$标准型$g_\Lambda$

\[\begin{gathered}
        f=\xv^TA\xv\\
        g=\yv^T\Lambda\yv
    \end{gathered}\]

求变换矩阵$C$,并$\exists C^{-1}$,使得:

\[\begin{gathered}
        \xv=C\yv\\
        C^TAC=\Lambda\\
        f=\xv^TA\xv=\yv^TC^TAC\yv=\yv^T\Lambda\yv=g
    \end{gathered}\]

\subsubsection{正交变换法}

本法要求$A$必为对称矩阵;$C$正交,$g^\prime$系数不一定为特征值,$g$系数为特征值

(由于实对称矩阵的相似则必合同,该法同相似对角化$\p{\ref{SimilarityDiagonalization}}$)

特征向量矩阵$P$即为变换矩阵,可通过$\xv=P\yv$将$f$化为标准型$g^\prime=\sum\lambda_ik_iy_i^2$

注:其中$k_i$为非零常系数,且$\prod k_i=\p{\det P}^2$

构造正交矩阵:

1. 对特征向量组$\B{\pv_n}$使用格拉姆-施密特正交化$\p{\ref{Orthogonalization}}$并单位化,解得正交单位特征向量组$\B{\ev_n}$

2. 用正交单位特征向量组构建正交矩阵$C=\begin{bmatrix}\ev_1&\ev_2&\cdots&\ev_n\end{bmatrix}$,可通过$\xv=C\yv$将$f$化为标准型$g=\sum\lambda_iy_i^2$

\subsubsection{拉格朗日配方法}

$C$可逆,$g$系数不一定为特征值

1. 先配$x_1$,再依次往后配;配完的变量后面不能再出现

2. 若只有交叉项,没有平方项,则令$\left\{\begin{aligned}
        x_1 & =z_1+z_2 \\
        x_2 & =z_1-z_2 \\
        x_3 & =z_3     \\
            & \vdots   \\
        x_n & =z_n
    \end{aligned}\right.$,替换后按$z$配方

3. 配完后得:$f=k_1\p{\sum\limits_{i=1}^nk_{1i}x_i}^2+
    k_2\p{\sum\limits_{i=2}^nk_{2i}x_i}^2+\cdots+
    k_n\p{k_{n1}x_n}^2$可替换每一个平方项为一个变量$y$,即:$\boldsymbol y=K\xv:y_j=\sum\limits_{i=j}^nk_{ji}x_i\left\{\begin{aligned}
        y_1 & =\sum_{i=1}^nk_{1i}x_i \\
        y_2 & =\sum_{i=2}^nk_{2i}x_i \\
            & \vdots                 \\
        y_n & =k_{n1}x_n
    \end{aligned}\right.$(若$y_j=0$可令$y_j=x_j$,以便$K$可逆),则原二次型已转为标准型$g=k_1y_1^2+k_2y_2^2+\cdots+k_ny_n^2$

4. 作倒代换得$\xv=C\boldsymbol y:x_j=\sum\limits_{i=j}^nc_{ji}y_i\left\{\begin{aligned}
        x_1 & =\sum_{i=1}^nc_{1i}y_i \\
        x_2 & =\sum_{i=2}^nc_{2i}y_i \\
            & \vdots                 \\
        x_n & =c_{n1}y_n
    \end{aligned}\right.$,此处$C=K^{-1}$即为$f$变为标准型$g$的变换矩阵

\subsubsection{初等变换法}

$C$可逆,$g$系数不一定为特征值

\[\begin{bmatrix}A\\E\end{bmatrix}\xrightarrow[\text{对}A\text{只作对应行变换}]{\text{对整体初等列变换}}\begin{bmatrix}\Lambda\\C\end{bmatrix}\]

\paragraph{对应行变换}

\subparagraph{将$a$列与$b$列交换}将$a$行与$b$行交换

\subparagraph{将$a$列乘以$k$}将$a$行乘以$k$

\subparagraph{将$a$列加到$b$列}将$a$行列加到$b$行

\subsection{规范型}

\begin{definition}[只有对角元素且元素只包含$1$、$-1$和$0$的二次型,称为规范型]
    \[f=\sum_{i=1}^py_i^2-\sum_{i=p+1}^{r\p{A}}y_i^2\]

    \[\text{实对称矩阵}A\simeq\begin{bmatrix}
            E_p &               &   \\
                & -E_{r\p{A}-p} &   \\
                &               & O
        \end{bmatrix}\]

    其中$p$为$A$正特征值个数(正惯性指数)(重根按重数展开算),即$r\p{A}-p$为负特征值个数(负惯性指数)
\end{definition}

\[\text{实对称矩阵}A,B;A\simeq B\iff A,B\text{正负惯性指数相同}\]

\subsection{正定二次型}

\begin{definition}[只有正数特征值的二次型]
    \[A\simeq E\iff A\text{为正定矩阵(正定二次型)}\]
\end{definition}

\[\begin{aligned}
        A_n\text{正定}\iff & A\text{特征值全为正}                      \\
        \iff             & A\text{正惯性指数}=n                     \\
        \iff             & A\text{各阶顺序主子式都}>0\implies\det A>0  \\
        \iff             & \exists B;B^{-1}\to A=B^TB          \\
        \iff             & \forall\xv\neq\boldsymbol0\to f_A>0
    \end{aligned}\]

\end{document}
