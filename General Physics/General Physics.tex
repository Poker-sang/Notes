\documentclass{article}

\usepackage{ctex}
\usepackage{amsfonts}
\usepackage{amsmath}
\usepackage{amsthm}
\usepackage{graphicx}
\usepackage{float}
\usepackage{hyperref}
\usepackage{mathabx}
\usepackage{datetime}
\usepackage{tabularray}
\usepackage{mathrsfs}
\usepackage{geometry}

\title{普通物理学}
\author{}
\date{\today}

\geometry{a4paper,scale=0.8}

\begin{document}

\hypersetup{
    hidelinks,
    %colorlinks = true,
    allcolors = black,
    %pdfstartview = Fit,
    breaklinks = true
}

\newtheorem{definition}{Definition}[subsubsection]
\newtheorem{theorem}{Theorem}[subsubsection]
\newtheorem{corollary}{Corollary}[theorem]
\renewcommand{\proofname}{\indent\bf Proof}
\numberwithin{equation}{subsection}

\def\e{\mathrm e}
\def\i{\mathrm i}
\def\j{\mathrm j}
\def\d{\mathrm d}
\def\C{\mathrm C}
\def\sr{\mathbb R}
\def\sn{\mathbb N}
\def\snp{\mathbb N^+}
\def\sc{\mathbb C}
\def\sz{\mathbb Z}
\def\impint{\int\limits_{-\infty}^{+\infty}}

\newcommand{\abs}[1]{\left|#1\right|}
\newcommand{\pare}[1]{\left(#1\right)}

\begin{titlepage}
    \maketitle
\end{titlepage}

\tableofcontents
\newpage

\section{常数}

\begin{center}
    \begin{tblr}{c|c|c|c}
        \hline
        常量                     & 符号  & 值                           & 量纲                  \\
        \hline
        地球重力加速度           & $g$   & $9.8$                        & $m/s^2$               \\
        圆周率                   & $\pi$ & $3.1415926\cdots$            &                       \\
        真空光速                 & $c$   & $2.99792458\cdot{10}^8$      & $m/s$                 \\
        绝对零度                 &       & $0\pare{-273.15^{\circ}C}$   & $K$                   \\
        元电荷                   & $e$   & $1.602117733\cdot{10}^{-19}$ & $C$                   \\
        引力常量                 & $G$   & $6.672\cdot{10}^{-11}$       & $N\cdot m^2/{kg}^2$   \\
        静电力常量               & $k$   & $8.987551\cdot{10}^9$        & $N\cdot m^2/C^2$      \\
        阿伏伽德罗常数           & $N_A$ & $6.0221367\cdot{10}^{23}$    & $mol^{-1}$            \\
        普朗克常数               & $h$   & $6.62607015\cdot{10}^{-34}$  & $J\cdot s$            \\
        里德伯常量               & $R$   & $1.097373157\cdot{10}^7$     & $m^{-1}$              \\
        气体摩尔体积(标准情况) & $n$   & $22.4$                       & $L/mol$               \\
        普适气体常数             & $R$   & $8.31$                       & $J/\pare{mol\cdot K}$ \\
        玻尔兹曼常数             & $k_B$ & $1.380649\cdot{10}^{-23}$    & $J/K$                 \\
        \hline
    \end{tblr}
\end{center}


\section{平动与转动}

$r$:为某点到参考点的位矢

\begin{center}
    \begin{longtblr}[
            caption = {平动与转动},
            note{$\ast$} = {指标量,本表中矢量省略了箭头},
        ]{c|c|c||c|c|c}
        \hline
        \SetCell[c=3]{c}平动      &       &                      & \SetCell[c=3]{c}转动      &          &                             \\
        \hline
        位移                      & $x$   &                      & 角度                      & $\theta$ &                             \\
        速度                      & $v$   & $\dfrac{\d x}{\d t}$ & 角速度                    & $\omega$ & $\dfrac{\d\theta}{\d t}$    \\
        加速度                    & $a$   & $\dfrac{\d v}{\d t}$ & 角加速度                  & $\alpha$ & $\dfrac{\d\omega}{\d t}$    \\
        质量\TblrNote{$\ast$}     & $m$   &                      & 转动惯量\TblrNote{$\ast$} & $J$      & $\displaystyle\int r^2\d m$ \\
        力                        & $F$   & $ma$                 & 力矩                      & $M$      & $J\alpha=r\times F$         \\
        动量                      & $p$   & $mv$                 & 角动量                    & $L$      & $J\omega=r\times p$         \\
        冲量                      & $I$   & $Ft=p-p_0$           & 冲量矩                    & $H$      & $Mt=L-L_0$                  \\
        平动动能\TblrNote{$\ast$} & $E_k$ & $\dfrac12mv^2$       & 转动动能\TblrNote{$\ast$} & $E_k$    & $\dfrac12J\omega^2$         \\
        功\TblrNote{$\ast$}       & $A$   & $Fx$                 & 功\TblrNote{$\ast$}       & $A$      & $M\theta$                   \\
        \hline
    \end{longtblr}
\end{center}

\section{能量}

\subsection{圆周运动}

$a_t$:切向加速度

$a_n$:法向加速度

\[v=R\omega\]

\[a_t=R\alpha\]

\[a_n=\frac{v^2}R=v\omega=R\omega^2\]

\subsection{保守力}

\[\oint\vec F\d\vec r=0\]

\begin{center}
    \begin{tblr}{c|c|c}
        \hline
        保守力类型 & 力$F/N$            & 势能$E_p/J$    \\
        \hline
        重力       & $mg$               & $mgh$          \\
        弹力       & $kx$               & $\dfrac12kx^2$ \\
        引力       & $G\dfrac{Mm}{R^2}$ & $G\dfrac{Mm}R$ \\
        \hline
    \end{tblr}
\end{center}

\subsection{动能}

\[E_k=\frac12mv^2=\frac{p^2}{2m}\]

\section{相对论}

$u$:参考系移动速度,沿$Ox$方向

\subsection{洛伦兹因子}

\[\gamma=\frac1{\sqrt{1-\pare{\dfrac vc}^2}}\]

\subsection{相对论效应}

\begin{center}
    \begin{tblr}{c|c}
        \hline
        效应     & 表达式                \\
        \hline
        尺缩效应 & $l=\dfrac{l_0}\gamma$ \\
        钟慢效应 & $t=t_0\gamma$         \\
        质增效应 & $m=m_0\gamma$         \\
        \hline
    \end{tblr}
\end{center}

\subsection{质能方程}

\[E=mc^2\]

\subsection{静能}

$m_0$:静质量

\[E_0=m_0c^2\]

\subsection{洛伦兹变换}

\[\left\{\begin{aligned}
        x^\prime & =\pare{x-ut}\gamma            \\
        t^\prime & =\pare{t-\frac u{c^2}x}\gamma
    \end{aligned}\right.\]

\subsection{一维速度叠加}

\[v_x^\prime=\frac{v_x-u}{1-\dfrac u{c^2}v}\]

\subsection{相对论动量和能量关系式}

\[E^2=c^2p^2+E_0^2\]

\section{气体动理论}

\subsection{理想气体物态方程}

$p$:气体压强

$V$:气体体积

$T$:气体热力学温度

$R$:普适气体常数

$M$:气体摩尔质量

$m$:气体质量

\[\frac{pV}T=\frac mMR\]

$n$:单位体积内的气体分子数

$k_B$:玻尔兹曼常数,$k_B=\dfrac R{N_A}$

\[p=nk_BT\]

\subsection{压强}

$\bar\varepsilon_{tk}$:分子平均平动动能

\[p=\frac23n\bar\varepsilon_{tk}\]

\subsection{气体分子方均根速率}

\[v_{rms}=\sqrt{\bar{v^2}}=\sqrt{\frac{3k_BT}{m_0}}=\sqrt{\frac{3RT}M}\]

\subsection{能量按自由度均分定理}

$i$:气体分子自由度

\[\bar\varepsilon_k=\frac i2k_BT\]

\begin{center}
    \begin{tblr}{c|c|c|c}
        \hline
        分子类型       & 平动自由度$i_t$ & 转动自由度$i_r$ & 分子平均总动能$\bar\varepsilon_k$ \\
        \hline
        单原子分子     & 3               & 0               & $\dfrac 32k_BT$                   \\
        刚性双原子分子 & 3               & 2               & $\dfrac 52k_BT$                   \\
        刚性多原子分子 & 3               & 3               & $3k_BT$                           \\
        非刚性分子     & 3               & >1              & 还有振动自由度等                  \\
        \hline
    \end{tblr}
\end{center}

\subsection{内能}

\[E=\frac mM\frac i2RT=\frac {mN_A}M\bar\varepsilon_k\]

\subsection{气体分子速率分布函数}

\[f\pare{v}=\frac{\d N}{N\d v}\]

\subsubsection{归一化条件}

\[\int\limits_0^{+\infty}f\pare{v}\d v=1\]

可据此求出

\[\Delta N=N\int_{v_1}^{v_2}f\pare{v}\d v\]

\[\Delta\bar v=\dfrac{\int_{v_1}^{v_2}vNf\pare{v}\d v}{\Delta N}\]

\[\bar v=\int\limits_0^{+\infty}vf\pare{v}\d v\]

\[\bar{v^2}=\int\limits_0^{+\infty}v^2f\pare{v}\d v\]

\subsubsection{麦克斯韦速率分布律}

$m_0$:单个气体分子质量

\[f\pare{v}=4\pi{\pare{\frac{m_0}{2\pi k_BT}}}^{\frac32}\exp\pare{-\frac{m_0v^2}{2k_BT}}v^2\]

\subsection{平均自由程}

$\bar v$:算术平均速率

$\bar Z$:碰撞频率

$d$:分子的作用球半径

$n$:单位体积内的气体分子数

\[\bar\lambda=\frac{\bar v}{\bar Z}=\frac1{\sqrt2\pi d^2n}\]

\section{热力学}

\subsection{热力学第一定律}

$Q$:吸收热量

$\Delta E$:内能增量

$A$:对外做功

\[Q=\Delta E+A\]

\subsection{准静态过程}

$C_m$:摩尔热容

$C_{V,m}$:气体的摩尔定容热容,$C_{V,m}=\dfrac i2R$

$C_{p,m}$:气体的摩尔定压热容(迈耶公式),$C_{p,m}=C_{V,m}+R$

$C_{p,m}$:气体的摩尔多方热容,$C_{,m}=C_{V,m}+R$

$\gamma$:摩尔热容比,$\gamma=\dfrac{C_{p,m}}{C_{V,m}}=\dfrac{i+2}i$

$n$:多方指数,$1<n<\gamma$,且:
\[n=\left\{\begin{aligned}
        0      &  & \text{等压过程} \\
        \infty &  & \text{等容过程} \\
        1      &  & \text{等温过程} \\
        \gamma &  & \text{绝热过程} \\
    \end{aligned}\right.\]

\begin{longtblr}[
        caption = {热力学},
        remark{注} = {在微分形式下$\Delta$变为微分算子$\d$}
    ]{colspec={c|c|c|c|c|c|c},hlines}
    过程 & 常量       & 过程方程(常量)              & $A$                                                                              & $Q$                                         & $\Delta E$                 & $C_m$     \\
    等容 & $V$        & $pT^{-1}$                     & 0                                                                                & \SetCell[c=2]{c} $\dfrac mMC_{V,m}\Delta T$ &                            & $C_{V,m}$ \\
    等压 & $p$        & $VT^{-1}$                     & $p\Delta V=\dfrac mMR\Delta T$                                                   & $\dfrac mMC_{p,m}\Delta T$                  & $\dfrac mMC_{V,m}\Delta T$ & $C_{p,m}$ \\
    等温 & $T$        & $pV$                          & \SetCell[c=2]{c} $\dfrac mMRT\ln\dfrac{V_2}{V_1}=\dfrac mMRT\ln\dfrac{p_1}{p_2}$ &                                             & $0$                        & $\infty$  \\
    绝热 & $\delta Q$ & $\begin{aligned}
                                  & pV^\gamma               \\
                                  & V^{\gamma-1}T           \\
                                  & p^{\gamma-1}T^{-\gamma}
                             \end{aligned}$ & $\begin{aligned}
                                                     & -\dfrac mMC_{V,m}\Delta T          \\
                                                   = & -\dfrac{\Delta\pare{pV}}{\gamma-1}
                                               \end{aligned}$                             & $0$                                         & $\dfrac mMC_{V,m}\Delta T$ & $0$                                                          \\
    多方 &            & $\begin{aligned}
                                  & pV^n          \\
                                  & V^{n-1}T      \\
                                  & p^{n-1}T^{-n}
                             \end{aligned}$           & $\begin{aligned}
                                                              & \dfrac mMRT\ln\dfrac{V_2}{V_1}=                  \\
                                                              & \dfrac mMRT\ln\dfrac{p_1}{p_2}  & \pare{n=1}     \\
                                                              & -\dfrac{\Delta\pare{pV}}{n-1}   & \pare{n\neq 1}
                                                         \end{aligned}$                & $A+\Delta E$                                & $\dfrac mMC_{V,m}\Delta T$ & $C_{n,m}$                                                       \\
\end{longtblr}

\subsection{循环过程}

若$p-V$坐标轴上有默认顺时针的循环过程$C$,即

\[\Delta E=0\]

\[A=Q_1-Q_2=\oint_C\d V\]

\paragraph{热机效率}

\[\eta=\frac A{Q_1}=1-\frac{Q_2}{Q_1}\]

若为逆时针即为制冷(积分为负)

\[A=Q_1-Q_2=\oint_C\d V\]

\paragraph{制冷系数}

\[w=\frac {Q_2}A=\frac{Q_2}{Q_1-Q_2}\]

\subsection{卡诺循环}

两个绝热和两个等温过程组成的循环

\[\eta_C=1-\frac{Q_2}{Q_1}=1-\frac{T_2}{T_1}\]

\end{document}
