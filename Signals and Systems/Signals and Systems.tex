\documentclass{article}

\usepackage{ctex}
\usepackage{amsfonts}
\usepackage{amsmath}
\usepackage{amsthm}
\usepackage{graphicx}
\usepackage{float}
\usepackage{hyperref}
\usepackage{mathabx}
\usepackage{datetime}
\usepackage{tabularray}
\usepackage{mathrsfs}

\title{Signals and Systems}
\author{}
\date{\today}

\begin{document}

\hypersetup{
    hidelinks,
    %colorlinks = true,
    allcolors = black,
    %pdfstartview = Fit,
    breaklinks = true
}

\newtheorem{definition}{Definition}[subsection]
\newtheorem{theorem}{Theorem}[subsection]
\newtheorem{corollary}{Corollary}[theorem]
\renewcommand{\proofname}{\indent\bf Proof}
\numberwithin{equation}{subsection}

\def\e{\mathrm e}
\def\i{\mathrm i}
\def\j{\mathrm j}
\def\d{\mathrm d}
\def\C{\mathrm C}
\def\div{\mathrm{div}}
\def\rot{\mathrm{rot}}
\def\vecv{\vec{\mathrm v}}
\def\sr{\mathbb R}
\def\srp{\mathbb R^+}
\def\sn{\mathbb N}
\def\snp{\mathbb N^+}
\def\sc{\mathbb C}
\def\sz{\mathbb Z}
\edef\impint{\int\limits_{-\infty}^{+\infty}}
\def\sumsz{\sum_{n\in\sz}}

\newcommand{\abs}[1]{\left|#1\right|}
\newcommand{\pare}[1]{\left(#1\right)}
\newcommand{\fourier}[1]{\mathscr F\pare{#1}}
\newcommand{\ifourier}[1]{\mathscr F^{-1}\pare{#1}}
\newcommand{\laplace}[1]{\mathscr L\pare{#1}}
\newcommand{\ilaplace}[1]{\mathscr L^{-1}\pare{#1}}
\newcommand{\jacobi}[2]{\frac{\partial\pare{#1}}{\partial\pare{#2}}}

\def\ft{f\pare{t}}
\def\fw{F\pare{\omega}}
\def\fs{F\pare{s}}


\begin{titlepage}
    \maketitle
\end{titlepage}

\tableofcontents
\newpage

\part{绪论}

\section{能量与功率}

\subsection{能量信号}

能量为有限值的信号

\[E=\impint f^2\pare{t}\d t\]

\subsection{功率信号}

功率为有限值的信号

\[P=\lim_{T\to\infty}\frac1{2T}\int_{-T}^Tf^2\pare{t}\d t\]

\section{时间信号}

\subsection{指数信号}

默认$\alpha>0$

\[\ft=k\e^{-\alpha t}\]

\subsection{复指数信号}

\[s=\alpha+\j\omega\]
\[\ft=k\e^{st}=k\e^\alpha\pare{\cos\omega t+\j\sin\omega t}\]

\subsubsection{欧拉公式}

\[\e^{\j\omega t}=\cos\omega t+\j\sin\omega t\]
\[\left\{\begin{aligned}
        \sin\omega t=\frac1{2\j}\pare{\e^{\j\omega t}-\e^{-\j\omega t}} \\
        \cos\omega t=\frac12\pare{\e^{\j\omega t}+\e^{-\j\omega t}}     \\
    \end{aligned}\right.\]

\subsection{正弦信号}

\[\ft=k\sin\pare{\omega t+\theta}\]

\subsection{抽样信号}

\[Sa\pare{t}=\frac{\sin t}t\]

\[\impint Sa\pare{t}=\pi\]

\subsection{单位阶跃信号}

\[u\pare{t}=
    \left\{\begin{aligned}
        0 &  & t<0 \\
        1 &  & t>0
    \end{aligned}\right.\]

\subsection{单位矩形脉冲信号}

\[g_T\pare{t}=
    \left\{\begin{aligned}
        0 &  & \abs{t}>\frac T2 \\
        1 &  & \abs{t}<\frac T2
    \end{aligned}\right.\]

\[g_T\pare{t}=u\pare{t+\frac T2}-u\pare{t-\frac T2}\]

\subsection{符号函数}

\[\mathrm{sgn}\pare{t}=
    \left\{\begin{aligned}
        -1 &  & t<0 \\
        1  &  & t>0
    \end{aligned}\right.\]

\[\mathrm{sgn}\pare{t}=2u\pare{t}-1\]

\subsection{单位冲激信号(狄拉克函数)}

\[\delta\pare{t}=
    \left\{\begin{aligned}
        0       &  & t=0    \\
        +\infty &  & t\neq0
    \end{aligned}\right.\]

\[\int_{0_-}^{0_+}\delta\pare{t}\d t=1\]

\[\delta\pare{t}=\frac{d u\pare{t}}{\d t}\]

\[\delta\pare{at}=\frac1{\abs a}\delta\pare{t}\pare{a\neq0}\]

\[\delta\pare{t-t_0}\ast\ft=f\pare{t-t_0}\]

\part{LTI时域分析}

\section{电路基础}

\subsection{电阻}

\[u_t=Ri_t\]

\subsection{电感}

\[u_t=Li_t^\prime\]

\[u_C\pare{0_-}=u_C\pare{0_+}\]

\subsection{电容}

\[i_t=Cu_t^\prime\]

\[i_L\pare{0_-}=i_L\pare{0_+}\]

\section{微分方程}

\subsection{自由响应(固有响应)}

电路微分方程对应的齐次(通)解

\[r_h\pare{t}\]

\subsection{强迫响应}

电路微分方程对应的(非齐次)特解

\[r_p\pare{t}\]

\subsection{全响应}

电路微分方程对应的全解

\[r\pare{t}=r_h\pare{t}+r_p\pare{t}\]

\section{LTI电路}

\subsection{起始状态}

\[r^{\pare{k}}\pare{0_-}
    \pare{k\in\{0,1,\cdots,n-1\}}\]

\subsection{初始状态}

\[r^{\pare{k}}\pare{0_+}
    \pare{k\in\{0,1,\cdots,n-1\}}\]

\subsection{零输入响应}

\[r_{zi}^{\pare{k}}\pare{0_+}
    =r_{zi}^{\pare{k}}\pare{0_-}
    =r^{\pare{k}}\pare{0_-}
    \pare{k\in\{0,1,\cdots,n-1\}}\]

\subsection{零状态响应}

\[r_{zs}^{\pare{k}}\pare{0_-}=0
    \pare{k\in\{0,1,\cdots,n-1\}}\]

\subsection{全响应}

\[r\pare{t}=r_{zi}\pare{t}+r_{zs}\pare{t}\]
\[\frac{s-2}{s^2+s+3}\]
\section{全响应分解}

\[\begin{aligned}
        r\pare{t}
         & =\underbrace{\sum_{k=1}^nA_k\e^{\alpha_kt}}_\text{自由响应}
        +\underbrace{B\pare{t}}_\text{强迫响应}                              \\
         & =\underbrace{\sum_{k=1}^nA_{zik}\e^{\alpha_kt}}_\text{零输入响应}
        +\underbrace{\sum_{k=1}^nA_{zsk}\e^{\alpha_kt}+B\pare{t}}_\text{零状态响应}
    \end{aligned}\]

\section{卷积}

\[\ft\ast g\pare{t}
    =\impint
    f\pare{\tau}g\pare{t-\tau}\d\tau\]

\subsection{交换律}

\[f\ast g=g\ast f\]

\subsection{结合律}

\[\pare{f\ast g}\ast h=f\ast\pare{g\ast h}\]

\subsection{分配律}

\[f\ast\pare{g+h}=f\ast g+f\ast h\]

\subsection{并联系统}

\[h=h_1+h_2\]

\subsection{串联系统}

\[h=h_1\ast h_2\]

\subsection{微积分性质}

\[s=f\ast g\]

\[\implies s^{\pare{i}}=f^{\pare{j}}\ast g^{\pare{i-j}}\]

\part{傅里叶变换}

\section{正交函数集}

\[\int_{t_1}^{t_2}\varphi_i\varphi_j\d t
    \left\{\begin{aligned}
        =0    &  & i\neq j \\
        \neq0 &  & i=j
    \end{aligned}\right.\]

\section{傅里叶级数}

\subsection{三角形式}

\[\begin{aligned}
        \ft
        = & \frac{a_0}2+\sum_{n\in\snp}\pare{a_n\cos n\omega t+b_n\sin n\omega t} \\
        = & \frac{c_0}2+\sum_{n\in\snp}c_n\cos\pare{n\omega t+\varphi_n}
    \end{aligned}\]

\paragraph{周期信号}\[\ft\]

\paragraph{周期}\[T\]

\paragraph{角频率}\[\omega=\frac{2\pi}T\]

\paragraph{频率}\[f=T^{-1}\]

\paragraph{直流分量(与书上不同)}\[\frac{a_0}2=\frac{c_0}2\pare{b_0=0}\]

\paragraph{余弦分量的幅度}

\[a_n=\frac2T\int_{t_0}^{t_0+T}\ft\cos n\omega t\d t\]

\paragraph{正弦分量的幅度}

\[b_n=\frac2T\int_{t_0}^{t_0+T}\ft\sin n\omega t\d t\]

\paragraph{基波分量}\[c_1\cos\pare{n\omega t+\varphi_1}\]

\paragraph{$n$次谐波分量}\[c_n\cos\pare{n\omega t+\varphi_n}\]

\paragraph{其他关系}

\[\begin{aligned}
        a_n       & =c_n\cos\varphi_n        \\
        b_n       & =-c_n\sin\varphi_n       \\
        c_n       & =\sqrt{a_n^2+b_n^2}      \\
        \varphi_n & =-\arctan\frac{b_n}{a_n}
    \end{aligned}\]

\paragraph{奇函数性质}\[\ft=-f\pare{-t}\implies a_n=0\]

\paragraph{偶函数性质}\[\ft=f\pare{-t}\implies b_n=0\]

\paragraph{奇谐函数性质}\[\ft=-f\pare{t\pm\frac T2}\]

\[\implies\left\{\begin{aligned}
        a_n & =
        \left\{\begin{aligned}
                    & 0                                             &  & n=2k   \\
                    & \frac4T\int_0^{\frac T2}\ft\cos n\omega t\d t &  & n=2k+1
               \end{aligned}\right. \\
        b_n & =
        \left\{\begin{aligned}
                    & 0                                             &  & n=2k   \\
                    & \frac4T\int_0^{\frac T2}\ft\sin n\omega t\d t &  & n=2k+1
               \end{aligned}\right.
    \end{aligned}\right.k\in\sz\]

\subsection{指数形式}

\[\ft=\sumsz F_n\e^{\j n\omega t}\]

\paragraph{复系数}

\[F_n=\frac1T\int_{t_0}^{t_0+T}\ft\e^{-\j n\omega t}\d t\]

\paragraph{幅度谱}

\[\abs{F_n}\sim n\omega\]

\paragraph{相位谱}

\[\varphi_n\sim n\omega\]

\paragraph{其他关系}

\[F_{\pm n}=\abs{F_{\pm n}}\e^{\pm\j\varphi_n}=\frac12\pare{a_n\mp\j b_n}\]

\[\abs{F_n}=\abs{F_{-n}}=\frac{c_n}2\]

\paragraph{功率}

\[P=\sumsz\abs{F_n}^2\]

\subsection{周期矩形脉冲信号}

\paragraph{一个周期内}

\[\ft=
    \left\{\begin{aligned}
        E &  & \abs{t}            & \leqslant\frac\tau2 \\
        0 &  & \frac\tau2<\abs{t} & \leqslant\frac T2
    \end{aligned}\right.\]

\[a_n=c_n=\frac{2E\tau}TSa\pare{\frac{n\omega\tau}2}\]

\[b_n=0\]

\[\begin{aligned}
        \ft & =\frac{E\tau}T+\frac{2E\tau}T\sum_{n\in\snp}Sa\pare{\frac{n\omega\tau}2}\cos n\omega t \\
            & =\frac{E\tau}T\sumsz Sa\pare{\frac{n\omega\tau}2}\e^{\j n\omega t}
    \end{aligned}\]

\section{傅里叶变换}

\paragraph{傅里叶变换}

\[\fw=\fourier\ft=\impint\ft\e^{-\j\omega t}\d t\]

\paragraph{傅里叶逆变换}

\[\ft=\ifourier{\fw}=\frac1{2\pi}\impint \fw\e^{\j\omega t}\d\omega\]

\paragraph{充分条件}

\[\impint\abs\ft\d t<+\infty\]

\paragraph{幅度谱}

\[\abs{\fw}\sim \omega\]

\paragraph{相位谱}

\[\abs{\varphi\pare{\omega}}\sim \omega\]

\subsection{频带宽度}

\paragraph{$\tau$:等效脉冲宽度}

\[f\pare{0}\tau=F\pare{0}=\impint\ft\d t\]

\paragraph{$B$:等效频带宽度}

\[F\pare{0}B_\omega=2\pi f\pare{0}=\impint \fw\d\omega\]

\[B_\omega=\frac{2\pi}\tau\]

\[B_f=\frac1\tau\]

\subsection{典型非周期信号傅里叶变换}

\subsubsection{单边指数信号}

\[\ft=\e^{-\alpha t}u\pare{t}\]

\[\left\{\begin{aligned}
        \fw                  & =\frac1{\alpha+\j\omega}          \\
        \abs{\fw}            & =\frac1{\sqrt{\alpha^2+\omega^2}} \\
        \varphi\pare{\omega} & =-\arctan\frac\omega\alpha
    \end{aligned}\right.\]

\subsubsection{双边指数信号}

\[\ft=\e^{-\alpha\abs t}\]

\[\left\{\begin{aligned}
        \fw                  & =\frac{2\alpha}{\alpha^2+\omega^2} \\
        \abs{\fw}            & =\frac{2\alpha}{\alpha^2+\omega^2} \\
        \varphi\pare{\omega} & =0
    \end{aligned}\right.\]

\subsubsection{矩形脉冲信号}

\[\ft=E\cdot g_T\pare{t}\]

\[\fw=E\tau\cdot Sa\pare{\frac{\omega\tau}2}\]

\[B_f=\frac1\tau\]

\subsubsection{符号函数}

\[\ft=\mathrm{sgn}\pare{t}\]

\[\left\{\begin{aligned}
        \fw                  & =\frac2{\j\omega}   \\
        \abs{\fw}            & =\frac2{\abs\omega} \\
        \varphi\pare{\omega} & =
        \left\{\begin{aligned}
                   \frac\pi2  &  & \omega<0 \\
                   -\frac\pi2 &  & \omega>0
               \end{aligned}\right.
    \end{aligned}\right.\]

\subsection{常用变换对}

\[\begin{tblr}{c|c}
        \hline
        \ft                     & \fw                                                                   \\
        \hline
        1                       & 2\pi\delta\pare{\omega }                                              \\
        \delta\pare{t}          & \j\omega                                                              \\
        u\pare{t}               & \pi\delta\pare{\omega }+\frac1{\j\omega}                              \\
        \e^{-\alpha t}u\pare{t} & \frac1{\alpha+\j\omega}                                               \\
        \e^{-\alpha\abs t}      & \frac{2\alpha}{\alpha^2+\omega^2}                                     \\
        g_\tau\pare{t}          & \tau\cdot Sa\pare{\frac{\omega\tau}2}                                 \\
        \mathrm{sgn}\pare{t}    & \frac2{\j\omega}                                                      \\
        \e^{\j\omega_0t}        & 2\pi\delta\pare{\omega-\omega_0}                                      \\
        \cos\omega_0 t          & \pi\pare{\delta\pare{\omega+\omega_0}+\delta\pare{\omega-\omega_0}}   \\
        \sin\omega_0 t          & \j\pi\pare{\delta\pare{\omega+\omega_0}-\delta\pare{\omega-\omega_0}} \\
        \hline
    \end{tblr}\]

\subsection{性质}

\[\fourier{\ft}=\fw\]

\subsubsection{线性}

\[\fourier{f+g}=\fourier f+\fourier g\]

\subsubsection{对称性}

\[\fourier{\fourier\ft}=2\pi f\pare{-\omega}\]

\subsubsection{奇偶虚实性}

略

\subsubsection{尺度变换特性}

\[\fourier{f\pare{at}}=\frac1{\abs a}F\pare{\frac\omega a}\]

\subsubsection{时移特性}

\[\fourier{f\pare{t+t_0}}=\e^{\j\omega t_0}\cdot \fw\]

\subsubsection{频移特性}

\[\fourier{\ft\e^{-\j\omega_0t}}=F\pare{\omega+\omega_0}\]

\paragraph{频谱搬移}

\[\fourier{\ft\cos\omega_0t}
    =\fourier{\frac\ft2\pare{\e^{\j\omega t}+\e^{-\j\omega t}}}
    =\frac12\pare{F\pare{\omega+\omega_0}+F\pare{\omega-\omega_0}}\]

下式不常用

\[\fourier{\ft\sin\omega_0t}
    =\fourier{\frac\ft{2\j}\pare{\e^{\j\omega t}-\e^{-\j\omega t}}}
    =\frac\j2\pare{F\pare{\omega+\omega_0}-F\pare{\omega-\omega_0}}\]

\subsubsection{时域微分特性}

\[\fourier{f^{\pare{n}}\pare{t}}=\pare{\j\omega}^n\cdot \fw\]

\subsubsection{频域微分特性}

\[\fourier{\pare{\j t}^n\cdot\ft}\pare{t}=F^{\pare{n}}\pare{\omega}\]

\subsubsection{时域积分特性}

\[\fourier{\int\limits_{-\infty}^tf\pare{\tau}\d\tau}=\frac{\fw}{\j\omega}+\pi F\pare{0}\delta\pare{\omega}\]

\subsubsection{频域积分特性}

略

\subsubsection{时域卷积特性}

\[\fourier{f\ast g}=\fourier f\cdot\fourier g\]

\subsubsection{频域卷积特性}

\[\fourier{f\cdot g}=\frac1{2\pi}\fourier f\ast\fourier g\]

\subsection{周期信号傅里叶变换}

$\ft$角频率为$\omega_0$,第一个周期的频域$F_0\pare{\omega}$

\[\fw=2\pi\sumsz F_n\delta\pare{\omega-n\omega_0}\]

\[F_n=\left.\frac1TF_0\pare{\omega}\right|_{\omega=n\omega_0}=\left.\frac1T\int_{-\frac T2}^{\frac T2}\ft\e^{-\j\omega t}\d t\right|_{\omega=n\omega_0}\]

\subsection{抽样信号的傅里叶变换}

抽样信号$=$连续信号$\cdot$抽样脉冲

\[f_s\pare{t}=\ft p\pare{t}\]

\[P\pare{\omega}=2\pi\sumsz P_n\delta\pare{\omega-n\omega_s}\]

\[F_s\pare{\omega}=\frac1{2\pi}\fw\ast P\pare{\omega}=\sumsz P_nF\pare{\omega-n\omega_s}\]

\subsubsection{均匀抽样脉冲}

$p\pare{t}$角频率(抽样频率)为$\omega_s$,周期(抽样周期)为$T_s$

\paragraph{冲激抽样}

\[\delta_T\pare{t}=\sumsz\delta\pare{t-nT_s}\]

\[\left\{\begin{aligned}
        P_n              & =\frac1T_s                                   \\
        \delta_T\pare{t} & =\frac1{T_s}\sumsz\e^{\j n\omega_s t}        \\
        P\pare{\omega}   & =\omega_s\sumsz\delta\pare{\omega-n\omega_s}
    \end{aligned}\right.\]

\[F_s\pare{\omega}=\frac1{T_s}\sumsz F\pare{\omega-n\omega_s}\]

\paragraph{矩形脉冲(自然)抽样}

\[\delta_T\pare{t}=\sumsz\delta\pare{t-nT_s}\]

\[P_n=\frac{E\tau}{T_s}Sa\pare{\frac{n\omega_s\tau}2}\]

\[F_s\pare{\omega}=\frac{E\tau}{T_s}\sumsz Sa\pare{\frac{n\omega_s\tau}2}F\pare{\omega-n\omega_s}\]

\subsubsection{时域抽样定理}

\paragraph{奈奎斯特频率}

\[f_s\geqslant2f_m\]

\paragraph{奈奎斯特周期}

\[T_s\leqslant\frac1{2f_m}\]

\subsubsection{LTI系统的频域分析}

系统冲激响应为$h\pare{t}$

$\ft$作用于LTI系统响应

\[y\pare{t}=\ifourier{f\pare{\omega}H\pare{\omega}}\]

\[Y\pare{\omega}=f\pare{\omega}H\pare{\omega}\]

\part{拉普拉斯变换}

\section{拉普拉斯变换}

\paragraph{双边拉普拉斯变换}

\[\fs=\laplace{\ft}=\impint\ft\e^{-st}\d t\pare{s=\sigma+\j\omega}\]

\paragraph{单边拉普拉斯变换(以下拉普拉斯变换都指单边)}

\[\fs=\laplace{\ft}=\int\limits_0^{+\infty}\ft\e^{-st}\d t\pare{s=\sigma+\j\omega}\]

\paragraph{拉普拉斯逆变换}

\[\ft=\ilaplace{\fs}=\frac1{2\pi\j}\int_{\sigma-\j\infty}^{\sigma+\j\infty}\fs\e^{st}\d s\]

\subsection{常用变换对}

定义域$D\subseteq\srp$,收敛域$\sigma>\sigma_0$

\[\begin{tblr}{c|c|c}
        \hline
        \ft                         & \fs                                           & \sigma_0 \\
        \hline
        t^n                         & \frac{n!}{s^{n+1}}                            & 0        \\
        \e^{-\lambda t}             & \frac1{s+\lambda}                             & -\lambda \\
        \e^{\j\omega_0t}            & \frac1{s-\j\omega_0}                          & 0        \\
        \cos\omega t                & \frac s{s^2+\omega^2}                         & 0        \\
        \sin\omega t                & \frac\omega{s^2+\omega^2}                     & 0        \\
        \delta^{\pare{n}}\pare{t}   & s^n                                           & -\infty  \\
        t^n\e^{-\lambda t}          & \frac{n!}{\pare{s+\lambda}^{n+1}}             & -\lambda \\
        \e^{-\lambda t}\cos\omega t & \frac{s+\lambda}{\pare{s+\lambda}^2+\omega^2} & -\lambda \\
        \e^{-\lambda t}\sin\omega t & \frac\omega{\pare{s+\lambda}^2+\omega^2}      & -\lambda \\
        t\cos\omega t               & \frac{s^2-\omega^2}{\pare{s^2+\omega^2}^2}    & 0        \\
        t\sin\omega t               & \frac{2\omega s}{\pare{s^2+\omega^2}^2}       & 0        \\
        \hline
    \end{tblr}\]

\subsection{性质}

以下函数收敛域

\[\begin{aligned}
        \laplace{f} & :\sigma>\sigma_1 \\
        \laplace{g} & :\sigma>\sigma_2
    \end{aligned}\]

\[\laplace{\ft}=\fs\]

\subsubsection{线性}

\[\laplace{f+g}=\laplace{f}+\laplace{g}\]

\[\sigma>\max\pare{\sigma_1,\sigma_2}\]

\subsubsection{尺度变换特性}

\[\laplace{f\pare{at}}=\frac1aF\pare{\frac sa}\]

\[\sigma>a\sigma_1\]

\subsubsection{时移特性}

\[\laplace{f\pare{t-t_0}}=e^{-st_0}\fs\pare{t_0\geqslant0}\]

\[\sigma>\sigma_0\]

\subsubsection{复频域平移特性}

\[\laplace{\e^{-\lambda t}\ft}=F\pare{s+\lambda}\]

\[\sigma>\sigma_1-\lambda\]

\subsubsection{时域微分特性}

\[\laplace{f^{\pare{n}}\pare{t}}=s^nF\pare{s}-\sum_{r=0}^{n-1}s^{n-r-1}f^r\pare{0}\]

\[\sigma>\sigma_1\]

\subsubsection{复频域微分特性}

\[\laplace{\pare{-t}^n\ft}=F^{\pare{n}}\pare{s}\]

\[\sigma>\sigma_1\]

\subsubsection{时域积分特性}

\[\laplace{\int\limits_{-\infty}^tf\pare{\tau}\d\tau}=
    \frac1s\pare{\fs+f^{-1}\pare{0}}\]

\[\sigma>\max\pare{\sigma_1,0}\]

\subsubsection{复频域积分特性}

略

\subsubsection{时域卷积特性}

\[\laplace{f\ast g}=\laplace{f}\cdot\laplace{f}\]

\[\sigma>\max\pare{\sigma_1,\sigma_2}\]

\subsubsection{复频域卷积特性}

\[\laplace{f\cdot g}=
    \frac1{2\pi\j}\pare{\laplace{f}\ast\laplace{g}}\]

\[\sigma>\sigma_1+\sigma_2\]

\subsubsection{初值定理}

\[\lim_{t\to0^+}\ft=f\pare{0^+}=\lim_{s\to+\infty}sF\pare{s}\]

\subsubsection{终值定理}

\[\lim_{t\to+\infty}\ft=f\pare{+\infty}=\lim_{s\to0}sF\pare{s}\]

\section{拉普拉斯逆变换}

\subsection{部分分式展开法}

\paragraph{有理分式}

\[\fs=\frac{\mathcal P_m\pare{s}}{\mathcal Q_n\pare{s}}
    =\frac{\sum\limits_{i=0}^mb_is^i}{\sum\limits_{i=0}^na_is^i}\]

\paragraph{若为有理真分式($m<n$)}

\[\fs=\frac{\mathcal P_m\pare{s}}{\prod\limits_{i=1}^m\pare{s-p_i}}\]

\subparagraph{若$k_1$项极点为一阶极点}

可分解为:

\[\fs=\sum_{i=1}^m\frac{k_i}{s-p_i}\]

\[k_i=\left.\pare{s-p_i}\fs\right|_{s=p_i}\pare{0\leqslant i\leqslant n}\]

\subparagraph{若$k_1$项极点为$r$重阶极点}

可分解为:

\[\fs=\frac{\mathcal P_m\pare{s}}{{\pare{s-p_1}}^r\cdot\prod\limits_{i=r+1}^m\pare{s-p_i}}
    =\sum_{j=1}^r\frac{k_j}{{\pare{s-p_1}}^j}+\sum_{i=r+1}^m\frac{k_i}{s-p_i}\]

\[\left\{\begin{aligned}
        k_j & =\left.\frac1{\pare{r-j}!}\left[{\pare{s-p_i}}^r\fs\right]^{\pare{r-j}}\right|_{s=p_1} &  & 0\leqslant j\leqslant r   \\
        k_i & =\left.\pare{s-p_i}\fs\right|_{s=p_i}                                                  &  & r+1\leqslant i\leqslant n
    \end{aligned}\right.\]

\paragraph{若为有理假分式($m\geqslant n$)}

\[\fs=\sum_{i=0}^{m-n}B_is^i+\frac{\mathcal P_{n-1}^\prime\pare{s}}{\mathcal Q_n\pare{s}}\]

后按有理真分式变换

\subsection{二阶LTI系统响应}

\[a_0y^{\prime\prime}\pare{t}+a_1y^\prime\pare{t}a_2y\pare{t}
    =b_0f^{\prime\prime}\pare{t}+b_1f^\prime\pare{t}+b_2\ft\]

\[\underbrace{a_0\left[{s^2Y\pare{s}-sy\pare{0_-}-y^\prime\pare{0_-}}\right]}_{a_0y^{\prime\prime}\pare{t}}
    +\underbrace{a_1\left[{sY\pare{s}-y\pare{0_-}}\right]}_{a_1y^\prime\pare{t}}
    +\underbrace{a_2Y\pare{s}}_{a_2y\pare{t}}\]

\[=\underbrace{b_0s^2\fs}_{b_0f^{\prime\prime}\pare{t}}
    +\underbrace{b_1s\fs}_{b_1f^\prime\pare{t}}
    +\underbrace{b_2\fs}_{b_2\ft}\]

\[Y\pare{s}
    =\underbrace{\frac{a_0\left[sy\pare{0_-}+y^\prime\pare{0_-}\right]+a_1y\pare{0_-}}{a_0s^2+a_1s+a_2}}_{Y_x\pare{s}}
    +\underbrace{\frac{b_0s^2+b_1s+b_2}{a_0s^2+a_1s+a_2}\fs}_{Y_f\pare{s}}\]

\paragraph{零输入响应}

\[y_x\pare{t}=\ilaplace{Y_x\pare{s}}\]

\paragraph{零状态响应}

\[y_f\pare{t}=\ilaplace{Y_f\pare{s}}\]

\subsection{电路的复频域模型}

\[\begin{tblr}{c|c|c}
        \hline
        \text{模型} & \text{时域}     & \text{复频域}                  \\
        \hline
        \text{电阻} & u_t=Ri_t        & U_s=RI_s                       \\
        \text{电感} & u_t=Li_t^\prime & U_s=L\pare{sI_s-i_t\pare{0_-}} \\
        \text{电容} & i_t=Cu_t^\prime & I_s=C\pare{sU_s-u_t\pare{0_-}} \\
        \hline
    \end{tblr}\]

\end{document}
