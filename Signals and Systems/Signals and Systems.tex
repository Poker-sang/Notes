\documentclass{article}

\usepackage{ctex}
\usepackage{amsfonts}
\usepackage{amsmath}
\usepackage{amsthm}
\usepackage{graphicx}
\usepackage{float}
\usepackage{hyperref}
\usepackage{mathabx}
\usepackage{datetime}
\usepackage{tabularray}
\usepackage{mathrsfs}

\title{Signals and Systems}
\author{}
\date{\today}

\begin{document}

\hypersetup{hidelinks,
    %colorlinks = true,
    allcolors = black,
    %pdfstartview = Fit,
    breaklinks = true
}

\newtheorem{definition}{Definition}[subsection]
\newtheorem{theorem}{Theorem}[subsection]
\newtheorem{corollary}{Corollary}[theorem]
\renewcommand{\proofname}{\indent\bf Proof}
\numberwithin{equation}{subsection}

\def\e{\mathrm e}
\def\i{\mathrm i}
\def\j{\mathrm j}
\def\d{\mathrm d}
\def\C{\mathrm C}
\def\div{\mathrm{div}}
\def\rot{\mathrm{rot}}
\def\vecv{\vec{\mathrm v}}
\def\sr{\mathbb R}
\def\sn{\mathbb N}
\def\snp{\mathbb N^+}
\def\sc{\mathbb C}
\def\sz{\mathbb Z}
\def\impint{\int\limits_{-\infty}^{+\infty}}

\newcommand{\abs}[1]{\left|#1\right|}
\newcommand{\pare}[1]{\left(#1\right)}
\newcommand{\fourier}[1]{\mathscr F\pare{#1}}
\newcommand{\tfourier}[1]{\mathscr F^{-1}\pare{#1}}
\newcommand{\jacobi}[2]{\frac{\partial\pare{#1}}{\partial\pare{#2}}}

\def\ft{f\pare{t}}

\begin{titlepage}
    \maketitle
\end{titlepage}

\tableofcontents
\newpage

\part{绪论}

\section{能量与功率}

\subsection{能量信号}

能量为有限值的信号

\[E=\impint f^2\pare{t}\d t\]

\subsection{功率信号}

功率为有限值的信号

\[P=\lim_{T\to\infty}\frac1{2T}\int_{-T}^Tf^2\pare{t}\d t\]

\section{时间信号}

\subsection{指数信号}

默认$\alpha>0$

\[\ft=k\e^{-\alpha t}\]

\subsection{复指数信号}

\[s=\alpha+\j\omega\]
\[\ft=k\e^{st}=k\e^\alpha\pare{\cos\omega t+\j\sin\omega t}\]

\subsubsection{欧拉公式}

\[\e^{\j\omega t}=\cos\omega t+\j\sin\omega t\]
\[\left\{\begin{aligned}
        \sin\omega t=\frac1{2\j}\pare{\e^{\j\omega t}-\e^{-\j\omega t}} \\
        \cos\omega t=\frac12\pare{\e^{\j\omega t}+\e^{-\j\omega t}}     \\
    \end{aligned}\right.\]

\subsection{正弦信号}

\[\ft=k\sin\pare{\omega t+\theta}\]

\subsection{抽样信号}

\[Sa\pare{t}=\frac{\sin t}t\]

\[\impint Sa\pare{t}=\pi\]

\subsection{单位阶跃信号}

\[u\pare{t}=
    \left\{\begin{aligned}
        0 &  & t<0 \\
        1 &  & t>0
    \end{aligned}\right.\]

\subsection{单位矩形脉冲信号}

\[g_T\pare{t}=
    \left\{\begin{aligned}
        0 &  & \abs{t}>\frac T2 \\
        1 &  & \abs{t}<\frac T2
    \end{aligned}\right.\]

\[g_T\pare{t}=u\pare{t+\frac T2}-u\pare{t-\frac T2}\]

\subsection{符号函数}

\[\mathrm{sgn}\pare{t}=
    \left\{\begin{aligned}
        -1 &  & t<0 \\
        1  &  & t>0
    \end{aligned}\right.\]

\[\mathrm{sgn}\pare{t}=2u\pare{t}-1\]

\subsection{单位冲激信号(狄拉克函数)}

\[\delta\pare{t}=
    \left\{\begin{aligned}
        0 &  & t=0    \\
        1 &  & t\neq0
    \end{aligned}\right.\]

\[\delta\pare{t}=\frac{d u\pare{t}}{\d t}\]

\part{LTI时域分析}

\section{电路基础}

\subsection{电阻}

\[i_t=\frac1Rv_t\]

\subsection{电感}

\[i_t=\frac1L\int_{-\infty}^tv_\tau\d\tau\]

\[u_C\pare{0_-}=u_C\pare{0_+}\]

\subsection{电容}

\[i_t=C\frac{\d v_t}{\d t}\]

\[i_L\pare{0_-}=i_L\pare{0_+}\]

\section{微分方程}

\subsection{自由响应(固有响应)}

电路微分方程对应的齐次(通)解

\[r_h\pare{t}\]

\subsection{强迫响应}

电路微分方程对应的(非齐次)特解

\[r_p\pare{t}\]

\subsection{全响应}

电路微分方程对应的全解

\[r\pare{t}=r_h\pare{t}+r_p\pare{t}\]

\section{LTI电路}

\subsection{起始状态}

\[r^{\pare{k}}\pare{0_-}
    \pare{k\in\{0,1,\cdots,n-1\}}\]

\subsection{初始状态}

\[r^{\pare{k}}\pare{0_+}
    \pare{k\in\{0,1,\cdots,n-1\}}\]

\subsection{零输入响应}

\[r_{zi}^{\pare{k}}\pare{0_+}
    =r_{zi}^{\pare{k}}\pare{0_-}
    =r^{\pare{k}}\pare{0_-}
    \pare{k\in\{0,1,\cdots,n-1\}}\]

\subsection{零状态响应}

\[r_{zs}^{\pare{k}}\pare{0_-}=0
    \pare{k\in\{0,1,\cdots,n-1\}}\]

\subsection{全响应}

\[r\pare{t}=r_{zi}\pare{t}+r_{zs}\pare{t}\]

\section{全响应分解}

\[\begin{aligned}
        r\pare{t}
         & =\underbrace{\sum_{k=1}^nA_k\e^{\alpha_kt}}_\text{自由响应}
        +\underbrace{B\pare{t}}_\text{强迫响应}                              \\
         & =\underbrace{\sum_{k=1}^nA_{zik}\e^{\alpha_kt}}_\text{零输入响应}
        +\underbrace{\sum_{k=1}^nA_{zsk}\e^{\alpha_kt}+B\pare{t}}_\text{零状态响应}
    \end{aligned}\]

\section{卷积}

\[\ft\ast g\pare{t}
    =\impint
    f\pare{\tau}g\pare{t-\tau}\d\tau\]

\subsection{交换律}

\[f\ast g=g\ast f\]

\subsection{结合律}

\[\pare{f\ast g}\ast h=f\ast\pare{g\ast h}\]

\subsection{分配律}

\[f\ast\pare{g+h}=f\ast g+f\ast h\]

\subsection{并联系统}

\[h=h_1+h_2\]

\subsection{串联系统}

\[h=h_1\ast h_2\]

\subsection{微积分性质}

\[s=f\ast g\]

\[\implies s^{\pare{i}}=f^{\pare{j}}\ast g^{\pare{i-j}}\]

\part{傅里叶变换}

\section{正交函数集}

\[\int_{t_1}^{t_2}\varphi_i\varphi_j\d t
    \left\{\begin{aligned}
        =0    &  & i\neq j \\
        \neq0 &  & i=j
    \end{aligned}\right.\]

\section{傅里叶级数}

\subsection{三角形式}

\[\begin{aligned}
        \ft
        = & \frac{a_0}2+\sum_{n\in\snp}\pare{a_n\cos n\omega t+b_n\sin n\omega t} \\
        = & \frac{c_0}2+\sum_{n\in\snp}c_n\cos\pare{n\omega t+\varphi_n}
    \end{aligned}\]

\paragraph{周期信号}\[\ft\]

\paragraph{周期}\[T\]

\paragraph{角频率}\[\omega=\frac{2\pi}T\]

\paragraph{频率}\[f=T^{-1}\]

\paragraph{直流分量(与书上不同)}\[\frac{a_0}2=\frac{c_0}2\pare{b_0=0}\]

\paragraph{余弦分量的幅度}

\[a_n=\frac2T\int_{t_0}^{t_0+T}\ft\cos n\omega t\d t\]

\paragraph{正弦分量的幅度}

\[b_n=\frac2T\int_{t_0}^{t_0+T}\ft\sin n\omega t\d t\]

\paragraph{基波分量}\[c_1\cos\pare{n\omega t+\varphi_n}\]

\paragraph{$n$次谐波分量}\[c_1\cos\pare{n\omega t+\varphi_n}\]

\paragraph{其他关系}

\[\begin{aligned}
        a_n       & =c_n\cos\varphi_n        \\
        b_n       & =-c_n\sin\varphi_n       \\
        c_n       & =\sqrt{a_n^2+b_n^2}      \\
        \varphi_n & =-\arctan\frac{b_n}{a_n}
    \end{aligned}\]

\paragraph{奇函数性质}\[\ft=-f\pare{-t}\implies a_n=0\]

\paragraph{偶函数性质}\[\ft=f\pare{-t}\implies b_n=0\]

\paragraph{奇谐函数性质}\[\ft=-f\pare{t\pm\frac T2}\]

\[\implies\left\{\begin{aligned}
        a_n & =
        \left\{\begin{aligned}
                    & 0                                             &  & n=2k   \\
                    & \frac4T\int_0^{\frac T2}\ft\cos n\omega t\d t &  & n=2k+1
               \end{aligned}\right. \\
        b_n & =
        \left\{\begin{aligned}
                    & 0                                             &  & n=2k   \\
                    & \frac4T\int_0^{\frac T2}\ft\sin n\omega t\d t &  & n=2k+1
               \end{aligned}\right.
    \end{aligned}\right.k\in\sz\]

\subsection{指数形式}

\[\ft=\sum_{n\in\sz}F_n\e^{\j n\omega t}\]

\paragraph{复系数}

\[F_n=\frac1T\int_{t_0}^{t_0+T}\ft\e^{-\j n\omega t}\d t\]

\paragraph{幅度谱}

\[\abs{F_n}\sim n\omega\]

\paragraph{相位谱}

\[\varphi_n\sim n\omega\]

\paragraph{其他关系}

\[F_{\pm n}=\abs{F_{\pm n}}\e^{\pm\j\varphi_n}=\frac12\pare{a_n\mp\j b_n}\]

\[\abs{F_n}=\abs{F_{-n}}=\frac{c_n}2\]

\paragraph{功率}

\[P=\sum_{n\in\sz}\abs{F_n}^2\]

\subsection{周期矩形脉冲信号}

\paragraph{一个周期内}

\[\ft=
    \left\{\begin{aligned}
        E &  & \abs{t}            & \leqslant\frac\tau2 \\
        0 &  & \frac\tau2<\abs{t} & \leqslant\frac T2
    \end{aligned}\right.\]

\[a_n=c_n=\frac{2E\tau}TSa\pare{\frac{n\omega\tau}2}\]

\[b_n=0\]

\[\begin{aligned}
        \ft & =\frac{E\tau}T+\frac{2E\tau}T\sum_{n\in\snp}Sa\pare{\frac{n\omega\tau}2}\cos n\omega t \\
            & =\frac{E\tau}T\sum_{n\in\sz}Sa\pare{\frac{n\omega\tau}2}\e^{\j n\omega t}
    \end{aligned}\]

\section{傅里叶变换}

\paragraph{傅里叶变换}

\[F\pare{\omega}=\fourier\ft=\impint\ft\e^{-\j\omega t}\d t\]

\paragraph{傅里叶逆变换}

\[\ft=\tfourier{F\pare{\omega}}=\frac1{2\pi}\impint F\pare{\omega}\e^{\j\omega t}\d\omega\]

\paragraph{充分条件}

\[\impint\abs\ft\d t<+\infty\]

\paragraph{幅度谱}

\[\abs{F\pare{\omega}}\sim \omega\]

\paragraph{相位谱}

\[\abs{\varphi\pare{\omega}}\sim \omega\]

\subsection{频带宽度}

\paragraph{$\tau$:等效脉冲宽度}

\[f\pare{0}\tau=F\pare{0}=\impint\ft\d t\]

\paragraph{$B$:等效频带宽度}

\[F\pare{0}B_\omega=2\pi f\pare{0}=\impint F\pare{\omega}\d\omega\]

\[B_\omega=\frac{2\pi}\tau\]

\[B_f=\frac1\tau\]

\subsection{典型非周期信号傅里叶变换}

\subsubsection{单边指数信号}

\[\ft=\e^{-\alpha t}u\pare{t}\]

\[\left\{\begin{aligned}
        F\pare{\omega}       & =\frac1{\alpha+\j\omega}          \\
        \abs{F\pare{\omega}} & =\frac1{\sqrt{\alpha^2+\omega^2}} \\
        \varphi\pare{\omega} & =-\arctan\frac\omega\alpha
    \end{aligned}\right.\]

\subsubsection{双边指数信号}

\[\ft=\e^{-\alpha\abs t}\]

\[\left\{\begin{aligned}
        F\pare{\omega}       & =\frac{2\alpha}{\alpha^2+\omega^2} \\
        \abs{F\pare{\omega}} & =\frac{2\alpha}{\alpha^2+\omega^2} \\
        \varphi\pare{\omega} & =0
    \end{aligned}\right.\]

\subsubsection{矩形脉冲信号}

\[\ft=E\cdot g_T\pare{t}\]

\[F\pare{\omega}=E\tau\cdot Sa\pare{\frac{\omega\tau}2}\]

\[B_f=\frac1\tau\]

\subsubsection{符号函数}

\[\ft=\mathrm{sgn}\pare{t}\]

\[\left\{\begin{aligned}
        F\pare{\omega}       & =\frac2{\j\omega}   \\
        \abs{F\pare{\omega}} & =\frac2{\abs\omega} \\
        \varphi\pare{\omega} & =
        \left\{\begin{aligned}
                   \frac\pi2  &  & \omega<0 \\
                   -\frac\pi2 &  & \omega>0
               \end{aligned}\right.
    \end{aligned}\right.\]

\subsection{常用变换对}

\[\begin{aligned}
        \ft                       & \overset{\mathscr F}{\underset{\mathscr F^{-1}}\rightleftharpoons}F\pare{\omega}         \\
        1                         & \rightleftharpoons 2\pi\delta\pare{t}                                                    \\
        \delta^{\pare{n}}\pare{t} & \rightleftharpoons \pare{\j\omega}^n\pare{n\geqslant0}                                   \\
        u\pare{t}                 & \rightleftharpoons \pi\delta\pare{t}+\frac1{\j\omega}                                    \\
        \e^{-\alpha t}u\pare{t}   & \rightleftharpoons \frac1{\alpha+\j\omega}                                               \\
        \e^{-\alpha\abs t}        & \rightleftharpoons \frac{2\alpha}{\alpha^2+\omega^2}                                     \\
        g_\tau\pare{t}            & \rightleftharpoons \tau\cdot Sa\pare{\frac{\omega\tau}2}                                 \\
        \mathrm{sgn}\pare{t}      & \rightleftharpoons \frac2{\j\omega}                                                      \\
        \e^{\j\omega_0t}          & \rightleftharpoons 2\pi\delta\pare{\omega-\omega_0}                                      \\
        \cos\pare{\omega_0 t}     & \rightleftharpoons \pi\pare{\delta\pare{\omega+\omega_0}+\delta\pare{\omega-\omega_0}}   \\
        \sin\pare{\omega_0 t}     & \rightleftharpoons \j\pi\pare{\delta\pare{\omega+\omega_0}-\delta\pare{\omega-\omega_0}}
    \end{aligned}\]

\subsection{性质}

\subsubsection{线性}

\[\fourier{f+g}=\fourier f+\fourier g\]

\subsubsection{对称性}

\[\fourier{\fourier\ft}=2\pi f\pare{-\omega}\]

\subsubsection{奇偶虚实性}

略

\subsubsection{尺度变换特性}

\[\fourier{f\pare{at}}=\frac1{\abs a}F\pare{\frac\omega a}\]

\subsubsection{时移特性}

\[\fourier{f\pare{t+t_0}}=\e^{\j\omega t_0}\cdot F\pare{\omega}\]

\subsubsection{频移特性}

\[\fourier{\ft\e^{-\j\omega_0t}}=F\pare{\omega+\omega_0}\]

\paragraph{频谱搬移}

\[\fourier{\ft\cos\omega_0t}
    =\fourier{\frac\ft2\pare{\e^{\j\omega t}+\e^{-\j\omega t}}}
    =\frac12\pare{F\pare{\omega+\omega_0}+F\pare{\omega-\omega_0}}\]

不常用

\[\fourier{\ft\sin\omega_0t}
    =\fourier{\frac\ft{2\j}\pare{\e^{\j\omega t}-\e^{-\j\omega t}}}
    =\frac\j2\pare{F\pare{\omega+\omega_0}-F\pare{\omega-\omega_0}}\]

\subsubsection{时域微分特性}

\[\fourier{f^{\pare{n}}\pare{t}}=\pare{\j\omega}^n\cdot F\pare{\omega}\]

\subsubsection{频域微分特性}

\[\fourier{\pare{\j t}^n\cdot\ft}\pare{t}=F^{\pare{n}}\pare{\omega}\]

\subsubsection{时域积分特性}

\[\fourier{\int_{-\infty}^tf\pare{\tau}\d\tau}=\frac{F\pare{\omega}}{\j\omega}\]

\subsubsection{时域卷积特性}

\[\fourier{f\ast g}=\fourier f\cdot\fourier g\]

\subsubsection{频域卷积特性}

\[\fourier{f\cdot g}=\frac1{2\pi}\fourier f\ast\fourier g\]

\subsection{周期信号傅里叶变换}

$\ft$角频率为$\omega_0$

\[F\pare{\omega}=2\pi\sum_{n\in\sz}F_n\delta\pare{\omega-n\omega_0}\]

\[F_n=\left.\frac1T\int_{-\frac T2}^{\frac T2}\ft\e^{-\j\omega t}\d t\pare{\omega}\right|_{\omega=n\omega_0}\]

\end{document}
