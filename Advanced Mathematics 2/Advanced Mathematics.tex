\documentclass{article}

\usepackage{ctex}
\usepackage{amsfonts}
\usepackage{amsmath}
\usepackage{amsthm}
\usepackage{amssymb}
\usepackage{graphicx}
\usepackage{float}
\usepackage{hyperref}
\usepackage{mathabx}
\usepackage{datetime}
\usepackage{tabularray}
\usepackage{geometry}
\usepackage{centernot}
\usepackage[dvipsnames]{xcolor}

\title{高等数学}
\author{}
\date{\today}

\geometry{a4paper,scale=0.8}

\begin{document}

\hypersetup{
    hidelinks,
    %colorlinks = true,
    allcolors = black,
    %pdfstartview = Fit,
    breaklinks = true
}

\newtheorem{definition}{Definition}[subsection]
\newtheorem{theorem}{Theorem}[subsection]
\newtheorem{corollary}{Corollary}[theorem]
\renewcommand{\proofname}{\indent\bf Proof}
\renewcommand{\Re}{\operatorname{Re}}
\renewcommand{\Im}{\operatorname{Im}}
\numberwithin{equation}{section}

\def\e{\mathrm e}
\def\i{\mathrm i}
\def\d{\mathrm d}
\def\C{\mathrm C}
\def\vecv{\vec{\mathrm v}}
\def\sr{\mathbb R}
\def\sn{\mathbb N}
\def\snp{\mathbb N^+}
\def\sc{\mathbb C}
\def\sz{\mathbb Z}
\def\sech{\mathrm{sech}}
\def\csch{\mathrm{csch}}

\newcommand{\abs}[1]{\left|#1\right|}
\newcommand{\p}[1]{\left(#1\right)}
\newcommand{\jacobi}[2]{\frac{\partial\p{#1}}{\partial\p{#2}}}

\begin{titlepage}
    \maketitle
\end{titlepage}

\tableofcontents
\newpage

\part{极限}

\section{基础}

\subsection{常用极限}

\[\begin{aligned}
         & \lim_{x\to0^+}{\p{1+\frac1x}^x}                     &  & =1                              \\
         & \lim_{x\to\infty}{\p{1+\frac1x}^x}                  &  & =\e                             \\
         & \lim_{n\to\infty}{\frac1n\sum_{i=1}^nf\p{\frac in}} &  & =\int_0^1f\p{x}\d x\p{n\in\snp}
    \end{aligned}\]

\subsection{常用等价无穷小}

$x$为函数,$\lim\limits_{x\to0}$时,可对乘除因子替换

\[x\sim\sin x\sim\tan x\sim\arcsin x\sim\arctan x\]

\[x\sim\p{\e^x-1}\sim\ln\p{x+1}\sim\ln\p{x+\sqrt{1+x^2}}\]

\[x^3\sim6\p{x-\sin x}\sim6\p{\arcsin x-x}\sim3\p{\tan x-x}\]

\[x^3\sim3\p{x-\arctan x}\sim2\p{\tan x -\sin x}\]

\[\begin{aligned}
         & 1-\cos x            &  & \sim\frac{x^2}2                    \\
         & \log_a{\p{1+x}}     &  & \sim\frac x{\ln a}                 \\
         & \p{1+x}^a           &  & \sim ax+1                          \\
         & a^x-1               &  & \sim x\ln a\p{0<a\neq1}            \\
         & \p{1+ax}^\frac1{bx} &  & \sim\e^\frac ab(1-\frac{a^2}{2b}x) \\
    \end{aligned}\]

\section{间断点}

\subsection{第一类间断点}

\[\exists\lim_{x\to x_0^-}\text{且}\exists\lim_{x\to x_0^+}\]

\subsubsection{可去间断点}

\[\lim_{x\to x_0^-}f\p{x}=\lim_{x\to x_0^+}f\p{x}=A\p{\iff\lim_{x\to x_0}f\p{x}=A}\]

\subsubsection{跳跃间断点}

\[\lim_{x\to x_0^-}f\p{x}\neq\lim_{x\to x_0^+}f\p{x}\]

\subsection{第二类间断点}

\[\lim_{x\to x_0^-},\lim_{x\to x_0^+}\text{至少满足有一个}\nexists\]

\subsubsection{振荡间断点}

左、右极限至少一个为振荡不存在

\subsubsection{无穷间断点}

左、右极限至少一个为$\infty$

\section{洛必达法则}

\subsection{使用条件}

\paragraph{定义存在}

\[x\in\mathring{U}\p{x_0}\text{(}x_0\text{可取}\infty\text{)},
    \exists f^\prime\p{x_0},
    \exists g^\prime\p{x_0}\]

\paragraph{极限存在或为无穷}

\[g^\prime\p{x_0}\neq0,\exists\lim_{x\to x_0}\frac{f^\prime\p{x}}{g^\prime\p{x}}\text{或}=\infty\]

\paragraph{符合$\dfrac00$或$\dfrac{\cdot}{\infty}$}

\subsection{结论}

\[\begin{aligned}
        \lim_{x\to x_0}\frac{f^\prime\p{x}}{g^\prime\p{x}}=A       & \implies\lim_{x\to x_0}\frac{f\p{x}}{g\p{x}}=A                 \\
        \lim_{x\to x_0}\frac{f^\prime\p{x}}{g^\prime\p{x}}=\infty  & \implies\lim_{x\to x_0}\frac{f\p{x}}{g\p{x}}=\infty            \\
        \nexists\lim_{x\to x_0}\frac{f^\prime\p{x}}{g^\prime\p{x}} & \centernot\implies\nexists\lim_{x\to x_0}\frac{f\p{x}}{g\p{x}}
    \end{aligned}\]

\section{泰勒展开}

\[f\p{x}\sim
    \sum_{n\in\sn}\frac{f^{\p{n}}\p{x_0}}{n!}\p{x-x_0}^n\]

\subsection{常用泰勒级数}

\[\newcommand{\series}[1]{\sum_{n\in#1}}
    \begin{aligned}
        \e^x           & =1+x+\frac{x^2}{2!}+\frac{x^3}{3!}+o\p{x^3}           &  & =\series\sn\frac{x^n}{n!}                                  &                              \\
        \sin x         & =x-\frac{x^3}{3!}+\frac{x^5}{5!}+o\p{x^5}             &  & =\series\sn\p{-1}^n\frac{x^{2n+1}}{\p{2n+1}!}              &                              \\
        \cos x         & =1-\frac{x^2}{2!}+\frac{x^4}{4!}+o\p{x^4}             &  & =\series\sn\p{-1}^n\frac{x^{2n}}{\p{2n}!}                  &                              \\
        \tan x         & =x+\frac13x^3+\frac2{15}x^5+o\p{x^5}                  &  & =\series\snp\frac{B_{2n}\p{-4}^n\p{1-4^n}}{\p{2n}!}^{2n-1} & x\in\p{-\frac\pi2,\frac\pi2} \\
        \arctan x      & =x-\frac{x^3}3+\frac{x^5}5+o\p{x^5}                   &  & =\series\sn\frac{\p{-1}^n}{2n+1}x^{2n+1}                   & x\in\left[-1,1\right]        \\
        \arcsin x      & =x+\frac16x^3+\frac3{40}x^5+o\p{x^5}                  &  & =\series\sn\frac{\p{2n}!}{4^n\p{n!}^2\p{2n+1}} x^{2n+1}    & x\in\p{-1,1}                 \\
        \ln\p{1+x}     & =x-\frac{x^2}2+\frac{x^3}3+o\p{x^3}                   &  & =\series\snp\p{-1}^{n-1}\frac{x^n}n                        & x\in\left(-1,1\right]        \\
        \ln\p{1-x}     & =-x-\frac{x^2}2-\frac{x^3}3+o\p{x^3}                  &  & =\series\snp-\frac{x^n}n                                   & x\in\left(-1,1\right]        \\
        \frac1{1+x}    & =1-x+x^2-x^3+o\p{x^3}                                 &  & =\series\sn\p{-x}^n                                        & x\in\p{-1,1}                 \\
        \frac1{1-x}    & =1+x+x^2+x^3+o\p{x^3}                                 &  & =\series\sn x^n                                            & x\in\p{-1,1}                 \\
        \p{1+x}^\alpha & =1+\alpha x+\frac{\alpha\p{\alpha-1}}{2!}x^2+o\p{x^2} &  & =\series\sn\binom \alpha nx^n                              & x\in\p{-1,1}
    \end{aligned}\]

\section{极限审敛}

\[\lim_{x\to0^+\atop y\to0^+}\frac{x^py^q}{x^m+y^n}\]

$m$、$n$全为偶数且$\dfrac pm+\dfrac qn>1$时$\lim\limits_{x\to0^+\atop y\to0^+}\dfrac{x^py^q}{x^m+y^n}=0$,否则不存在

$\dfrac pm+\dfrac qn\leqslant1$时,路径$y=kx^{\frac{m-p}q}$可说明极限不存在

\part{导数}

\section{基础}

\subsection{求导法则}

\[\begin{aligned}
        \p{f\p{x}+g\p{x}}^\prime                      & =f^\prime\p{x}+g^\prime\p{x}                                         \\
        \p{f\p{x}g\p{x}}^\prime                       & =f\p{x}g^\prime\p{x}+f^\prime\p{x}g\p{x}                             \\
        \p{f\p{g\p{x}}}^\prime                        & =f^\prime\p{g\p{x}}g^\prime\p{x}                                     \\
        \p{\int_{v\p{x}}^{u\p{x}}f\p{t}\d t}^\prime   & =f\left[u\p{x}\right]u^\prime\p{x}-f\left[v\p{x}\right]v^\prime\p{x} \\
        \p{\int_{v\p{x}}^{u\p{x}}f\p{x,t}\d t}^\prime & =\int_{v\p{x}}^{u\p{x}}f_x^\prime\p{x,t}\d t
        +f\left[x,u\p{x}\right]u^\prime\p{x}-f\left[x,v\p{x}\right]v^\prime\p{x}
    \end{aligned}\]

\subsection{常用高阶导数}

\[\begin{aligned}
         & \sin^{\p{n}}\omega x &  & =\omega^n\sin\p{\omega x+\frac{n\pi}2}   &  & \p{n\in\sn}  \\
         & \cos^{\p{n}}\omega x &  & =\omega^n\cos{\p{\omega x+\frac{n\pi}2}} &  & \p{n\in\sn}  \\
         & \ln^{\p{n}}\p{1+x}   &  & =\p{-1}^{n-1}\frac{\p{n-1}!}{\p{1+x}^n}  &  & \p{n\in\snp} \\
         & \ln^{\p{n}}\p{1-x}   &  & =-\frac{\p{n-1}!}{\p{1-x}^n}             &  & \p{n\in\snp} \\
    \end{aligned}\]

\subsection{莱布尼茨公式}

\[\p{uv}^{\p{n}}=\sum_{k=0}^n\binom nku^{\p{n-k}}v^{\p{k}}\]

\subsection{中值定理}

\[\begin{tblr}{c|c|c}
        \hline
        \text{定理}       & \text{公式}                                                                                         & \text{约束}              \\
        \hline
        \text{积分中值定理}   & f\p{\xi}=\dfrac{\int_a^bf\p{x}\d x}{\left.x\right|_a^b}                                           & \xi\in\left[a,b\right] \\
        \text{罗尔中值定理}   & f^\prime\p{\xi}=0                                                                                 & \xi\in\p{a,b}          \\
        \text{拉格朗日中值定理} & f^\prime\p{\xi}=\dfrac{\left.f\p{x}\right|_a^b}{\left.x\right|_a^b}                               & \xi\in\p{a,b}          \\
        \text{柯西中值定理}   & \dfrac{f^\prime\p{\xi}}{g^\prime\p{\xi}}=\dfrac{\left.f\p{x}\right|_a^b}{\left.g\p{x}\right|_a^b} & \xi\in\p{a,b}          \\
        \hline
    \end{tblr}\]

\subsection{泰勒中值定理}

$R_n\p{x}$为余项

\[P_n\p{x}=\sum_{i=0}^n\left[\p{x-x_0}\frac{\d}{\d x}\right]^i\frac{f\p{x_0}}{i!}+R_n\p{x}\]

\subsection{极值(拉格朗日乘数法)}

\paragraph{二元情况}

\[\left.\begin{aligned}
         & \left\{\begin{aligned}
                      \text{约束条件:} & \varphi\p{x,y}=0 \\
                      \text{目标函数:} & f\p{x,y}
                  \end{aligned}\right.                                                           \\
         & \left\{\begin{aligned}
                       & \nabla f=\lambda\nabla\varphi\p{\text{即}\nabla f\parallel\nabla\varphi} \\
                       & \varphi\p{x,y}=0
                  \end{aligned}\right.
    \end{aligned}\right\}
    \implies
    \begin{aligned}
         & \text{解得几组$\p{x_i,y_i}$即为可能的极值点}         \\
         & \text{(若无约束条件$\varphi\p{x,y}=0$,}        \\
         & \text{可设约束为$0=0$,即}\nabla\varphi=\p{0,0} \\
         & \text{则$\nabla f=\p{0,0}$)}
    \end{aligned}\]

\paragraph{检验可能的极值点$\p{x_0,y_0}$}

\[\begin{aligned}
        \left.\begin{aligned}
                  f^{\prime\prime2}_{xy}\p{x_0,y_0}<f^{\prime\prime}_{xx}\p{x_0,y_0}f^{\prime\prime}_{yy}\p{x_0,y_0} \\
                  f^{\prime\prime}_{xx}\p{x_0,y_0}>0
              \end{aligned}\right\}
                                                                                                           & \implies
        f{\p{x_0,y_0}}\text{为极小值点}                                                                                    \\
        \left.\begin{aligned}
                  f^{\prime\prime2}_{xy}\p{x_0,y_0}<f^{\prime\prime}_{xx}\p{x_0,y_0}f^{\prime\prime}_{yy}\p{x_0,y_0} \\
                  f^{\prime\prime}_{xx}\p{x_0,y_0}<0
              \end{aligned}\right\}
                                                                                                           & \implies
        f{\p{x_0,y_0}}\text{为极大值点}                                                                                    \\
        f^{\prime\prime2}_{xy}\p{x_0,y_0}>f^{\prime\prime}_{xx}\p{x_0,y_0}f^{\prime\prime}_{yy}\p{x_0,y_0} & \implies
        f{\p{x_0,y_0}}\text{不取极值}                                                                                     \\
        f^{\prime\prime2}_{xy}\p{x_0,y_0}=f^{\prime\prime}_{xx}\p{x_0,y_0}f^{\prime\prime}_{yy}\p{x_0,y_0} & \implies
        \text{需进一步讨论}
    \end{aligned}\]

\paragraph{$n$元情况}

\[\def\xs{\p{x_1,x_2,\cdots,x_n}}
    \left.\begin{aligned}
         & \left\{\begin{aligned}
                      \text{约束条件}\Phi\text{:} & \varphi_1\xs=0     \\
                                              & \varphi_2\xs=0     \\
                                              & \vdots             \\
                                              & \varphi_{n-1}\xs=0 \\
                      \text{目标函数:}            & f\xs
                  \end{aligned}\right.             \\
         & \left\{\begin{aligned}
                       & \nabla f=\sum_i\lambda_i\nabla\varphi_i\text{(三元时共面)} \\
                       & \text{约束条件}\Phi
                  \end{aligned}\right.
    \end{aligned}\right\}\implies\text{解得几组}\xs\text{即为可能的极值点}\]

\subsection{隐函数存在定理}

\paragraph{$F\p{x,y}$(二元)}

\[\frac{\d y}{\d x}=-\frac{F_x^\prime}{F_y^\prime}\p{F_y^\prime\neq0}\]

\paragraph{$F\p{x,y,z}$(多元)}

\[\frac{\partial y}{\partial x}=-\frac{F_x^\prime}{F_y^\prime}\p{F_y^\prime\neq0}\]

\subsection{雅可比行列式}

\[\jacobi{\textcolor{red}{u_1},u_2,\cdots,u_n}
    {\textcolor{green}{x_1},x_2,\cdots,x_n}=
    \begin{vmatrix}
        \partial_{\textcolor{green}{x_1}}\textcolor{red}{u_1} & \partial_{x_2}\textcolor{red}{u_1} &
        \cdots                                                & \partial_{x_n}\textcolor{red}{u_1}                   \\
        \partial_{\textcolor{green}{x_1}}u_2                  & \partial_{x_2}u_2                  &
        \cdots                                                & \partial_{x_n}u_2                                    \\
        \vdots                                                & \vdots                             & \ddots & \vdots \\
        \partial_{\textcolor{green}{x_1}}u_n                  & \partial_{x_2}u_n                  &
        \cdots                                                & \partial_{x_n}u_n
    \end{vmatrix}\]

\part{积分}

\section{基础}

\subsection{牛顿-莱布尼茨公式}

\[\int_a^b{f^\prime\p{x}\d x}=\left.f\p{x}\right|_a^b\]

\subsection{第一类换元(凑微分)法}

\[\int f\p{x}g\p{x}\d x=\int f\p{x}\d\p{\int g\p{x}\d x}\]

\subsection{第二类换元法}

\[\begin{aligned}
         & \int f\p{x}\d x                        &  & =\left.\int f\p{t}\d t\right|_{t=\varphi\p{x}}                                                                   \\
         & \int_a^bf\left[\varphi\p{x}\right]\d x &  & =\left.\int_{\varphi\p{a}}^{\varphi\p{b}}{f\p{t}\frac{{\d\varphi}^{-1}\p{t}}{ \d t}\d t}\right|_{t=\varphi\p{x}}
    \end{aligned}\]

\subsection{分部积分}

\[\left\{\begin{aligned}
        u & =u\p{x} \\
        v & =v\p{x}\end{aligned}\right.\]

\[\begin{aligned}
        uv                  & =\int{udv}+\int{vdu}         \\
        \left.uv\right|_a^b & =\int_a^b{udv}+\int_a^b{vdu}\end{aligned}\]

\subsection{常用积分表}

\subsubsection{三角函数总表}

\[\begin{tblr}{c|c|c||c|c|c}
        \hline
        \int f\p{x}\d x+\C        & f\p{x}            & f^\prime\p{x}                &
        \int f\p{x}\d x+\C        & f\p{x}            & f^\prime\p{x}                  \\
        \hline
        -\cos x                   & \sin x            & \cos x                       &
        \sin x                    & \cos x            & -\sin x                        \\
        -\ln\abs{\cos x}          & \tan x            & \sec^2 x                     &
        \ln\abs{\sin x}           & \cot x            & -\csc^2 x                      \\
        \ln\abs{\sec x+\tan x}    & \sec x            & \sec x\tan x                 &
        -\ln\abs{\csc x+\cot x}   & \csc x            & -\csc x\cot x                  \\
                                  & \arcsin x         & \dfrac1{\sqrt{1-x^2}}        &
                                  & \arccos x         & -\dfrac1{\sqrt{1-x^2}}         \\
                                  & \arctan x         & \dfrac1{1+x^2}               &
                                  & \mathrm{arccot}x  & -\dfrac1{1+x^2}                \\
                                  & \mathrm{arcsec} x & \dfrac1{\abs x\sqrt{x^2-1}}  &
                                  & \mathrm{arccsc}x  & -\dfrac1{\abs x\sqrt{x^2-1}}   \\
        \cosh x                   & \sinh x           & \cosh x                      &
        \sinh x                   & \cosh x           & \sinh x                        \\
        \ln\abs{\cosh x}          & \tanh x           & \sech^2 x                    &
        \ln\abs{\sinh x}          & \coth x           & -\csch^2 x                     \\
        \arctan\p{\e^x}           & \sech x           & -\sech x\tanh x              &
        -\ln\abs{\csch x+\coth x} & \csch x           & -\csch x\coth x                \\
                                  & \mathrm{arsinh} x & \dfrac1{\sqrt{x^2+1}}        &
                                  & \mathrm{arcosh} x & \dfrac1{\sqrt{x^2-1}}          \\
                                  & \mathrm{artanh} x & \dfrac1{1-x^2}               &
                                  & \mathrm{arcoth} x & \dfrac1{1-x^2}                 \\
                                  & \mathrm{arsech} x & -\dfrac1{\abs x\sqrt{1-x^2}} &
                                  & \mathrm{arcsch} x & -\dfrac1{\abs x\sqrt{1+x^2}}   \\
        \hline
    \end{tblr}\]

\subsubsection{其他}

\[\begin{aligned}
         & \int a^x\d x                       &  & =\frac{a^x}{\ln a}                  &  & +\C \\
         & \int\frac{\d x}{x^2-a^2}           &  & =\frac1{2a}\ln\abs{\frac{x-a}{x+a}} &  & +\C \\
         & \int\frac{\d x}{a^2+x^2}           &  & =\frac1a\arctan\frac xa             &  & +\C \\
         & \int\frac{\d x}{\sqrt{x^2\pm a^2}} &  & =\ln\abs{x+\sqrt{x^2\pm a^2}}       &  & +\C \\
         & \int\frac{\d x}{\sqrt{a^2-x^2}}    &  & =\arcsin{\frac xa}                  &  & +\C
    \end{aligned}\]

\subsubsection{华里士公式}

\[\int_0^{\frac\pi2}\sin^nx\d x=\int_0^{\frac\pi2}\cos^nx\d x=
    \left\{\begin{aligned}
         & \frac{\p{n-1}!!}{n!!}\cdot\frac\pi2 &  & n\text{为正偶数} \\
         & \frac{\p{n-1}!!}{n!!}               &  & n\text{为正奇数}
    \end{aligned}\right.\]

\subsection{万能代换}

\[x=2\arctan u\implies
    \left\{\begin{aligned}
        \sin x & =\frac{2u}{1+u^2}    \\
        \cos x & =\frac{1-u^2}{1+u^2} \\
        \d x   & =\frac2{1+u^2}\d u
    \end{aligned}\right.\]

\subsection{区间再现}

\[\int_a^bf\p{x}\d x=\int_a^bf\p{a+b-x}\d x\]

\subsubsection{对称区间}

\[\int_{-a}^af\p{x}\d x=\int_0^a\left[f\p{x}+f\p{-x}\right]\d x\]

\subsection{极坐标图形面积}

\[A=\frac12\int_{\alpha}^{\beta}{r^2\p{\theta}\d\theta}\]

\begin{definition}[以下参数方程中都有,且都可轮换]
    \[\left\{\begin{aligned}
            x & =x\p{t} \\
            y & =y\p{t}
        \end{aligned}\right.\]
\end{definition}

\subsection{旋转体体积(参数方程)}

绕$x$轴

圆盘法

\[V=\pi\int_a^bx^\prime y^2dt\]

柱壳法

\[V=2\pi\int_a^bxy^\prime ydt\]

\subsection{旋转体侧面积(参数方程)}

绕$x$轴

\[S=2\pi\int_a^b{y\sqrt{x^{\prime2}+y^{\prime2}}\d t}\]

\subsection{平面曲线弧长(参数方程)}

\[s=\int_a^b{\sqrt{{x^\prime}^2+{y^\prime}^2}\d t}=\int_{\alpha}^{\beta}{\sqrt{r^2\p{\theta}+r^{\prime2}\p{\theta}}\d\theta}\]

\subsection{平面曲线曲率(参数方程)}

曲率半径:$K^{-1}$

\[K=\frac{\abs{x^\prime y^{\prime\prime}-x^{\prime\prime}y^\prime}}{\p{x^{\prime2}+y^{\prime2}}^\frac32}\]

\section{重积分}

\subsection{二重积分}

\begin{definition}[$\d\sigma=\d x\d y$]
    \[\iint_Df\p{x,y}\d\sigma\]
\end{definition}

\subsubsection{换元}

\[\left.\begin{aligned}
        \left\{\begin{aligned}
                   x=x\p{u,v} \\
                   y=y\p{u,v}
               \end{aligned}\right. \\
        \left.\jacobi{x,y}{u,v}\right|_{D^\prime}\neq0
    \end{aligned}\right\}\implies\\
    \iint_Df\p{x,y}\d x\d y=
    \iint_{D^\prime}f\p{x,y}\abs{\jacobi{x,y}{u,v}}\d u\d v\]

\subsubsection{广义极坐标变换}

\[\left\{\begin{aligned}
        x\p{r,\theta} & =x_0+ar\cos\theta \\
        y\p{r,\theta} & =y_0+br\sin\theta
    \end{aligned}\right.\implies\\
    \iint_Df\p{x,y}\d x\d y=
    ab\iint_Df\p{x,y}r\d r\d\theta\]

\subsection{*积分应用}

\paragraph{密度为$\rho\p{x,y}$或$\rho\p{x,y,z}$}

\subsubsection{质量}

\[M=\iint_D\rho\p{x,y}\d\sigma,
    M=\iiint_\Omega\rho\p{x,y,z}\d\sigma\]

\subsubsection{质心}

\paragraph{质心的$x$坐标为}

\[\bar x=\frac{\displaystyle\iint_Dx\rho\p{x,y}\d\sigma}M,
    \bar x=\frac{\displaystyle\iiint_\Omega x\rho\p{x,y,z}\d\sigma}M\]

$\rho\p{\cdots}\equiv1$时,质心相当于形心

\subsubsection{转动惯量}

\paragraph{绕$x$轴时}

\[I_x=\iint_Dy^2\rho\p{x,y}\d\sigma,
    I_x=\iiint_\Omega\p{y^2+z^2}\rho\p{x,y,z}\d\sigma\]

\subsubsection{古尔丁定理}

\paragraph{旋转体体积(平面图形$D$绕直线$l:Ax+By+C=0$旋转)}

\[V=\iint_D2\pi d_l\p{x,y}\d x\d y=2\pi\iint_D\frac{\abs{Ax+By+C}}{\sqrt{A^2+B^2}}\d x\d y\]

\subparagraph{若$D$形心为$\p{x_0,y_0}$}

\[V=2\pi d_l\p{x_0,y_0}S_D=2\pi\frac{\abs{Ax_0+By_0+C}}{\sqrt{A^2+B^2}}\iint_D\d x\d y\]

\part{微分方程}

\section{n阶线性微分方程}

\begin{definition}[]
    \[y^{\p{n}}+\sum_{i=0}^{n-1}p_i\p{x}y^{\p{i}}=f\p{x}\label{LinearDifferentialEquationsOfOrderN}\]
\end{definition}

\subsection{线性相关}

\[\frac{f\p{x}}{g\p{x}}=\C(\C\in\sc)\]

\section{一阶线性微分方程}

\begin{definition}[$f\p{x}\equiv0$时,为齐次]
    \[\left.\begin{aligned}
            \p{\ref{LinearDifferentialEquationsOfOrderN}} \\
            n=1
        \end{aligned}\right\}
        \implies
        y^\prime+P\p{x}y=f\p{x}\]
\end{definition}

\begin{theorem}[通解]
    \[y=\frac{\int{f\p{x}\exp\p{\int P\p{x}\d x}\d x}+\C}{\exp\p{\int P\p{x}\d x}}\]
\end{theorem}

\section{二阶线性微分方程}

\begin{definition}[]
    \[y^{\prime\prime}+P\p{x}y^\prime+Q\p{x}y=f\p{x}\]
\end{definition}

\subsection{齐次、非齐次、通解、特解关系}

齐特+齐特(线性无关)=齐通

齐通+非特=非通

齐特+非特=非特

非特$-$非特=齐特

\section{n阶常系数线性齐次微分方程}

\begin{definition}[]
    \[y^{\p{n}}+\sum_{i=0}^{n-1}p_iy^{\p{i}}=0\p{p_i\in\sc}\]
\end{definition}

\subsection{特征方程}

\[r^n+\sum_{i=0}^{n-1}p_ir^i=0\]

\subsection{通解对应项}

\paragraph{$k$重实根$r$在通解中对应项}

\[y_r=\sum_{i=1}^k\C_ix^{i-1}\cdot\e^{rx}\]

\paragraph{特别的:$r$为共轭复根($r=\alpha\pm\beta\i$)时,可改写为两个实根}

\[y_r=(\C_1\cos{\beta x}+\C_2\sin{\beta x})\e^{\alpha x}\]

\section{二阶常系数线性齐次微分方程}

\begin{definition}[]
    \[y^{\prime\prime}+py^\prime+qy=0\]
\end{definition}

\subsection{特征方程}

\[r^2+pr+q=0\]

\subsection{通解}

\paragraph{$r_1\neq r_2$}

\[y=\C_1\e^{r_1x}+\C_2\e^{r_2x}\]

\paragraph{$r_1=r_2$}

\[y=\p{\C_1+\C_2x}\e^{r_1x}\]

\paragraph{$r_{1,2}=\alpha\pm\beta\i$}

\[y=(\C_1\cos{\beta x}+\C_2\sin{\beta x})\e^{\alpha x}\]

\section{二阶常系数非齐次线性微分方程}

\begin{definition}[]
    \[y^{\prime\prime}+py^\prime+qy=f\p{x}\label{LinearDifferentialEquationsWithSecondOrderConstantCoefficients}\]
\end{definition}

\subsection{特解}

\begin{theorem}[特解]
    $\mathcal P_n$表示$n$次多项式

    \[\left.\begin{aligned}
                   & \p{\ref{LinearDifferentialEquationsWithSecondOrderConstantCoefficients}}                            \\
            f\p{x} & =\left[\mathcal P_{n_1}\p{x}\cos{\omega x}+\mathcal P_{n_2}\p{x}\sin{\omega x}\right]\e^{\lambda x} \\
            m      & =\max{\left\{n_1,n_2\right\}}
        \end{aligned}\right\}
        \implies\]

    \[y^*=x^k\left[\mathcal U_m\p{x}\cos{\omega x}+\mathcal V_m\p{x}\sin{\omega x}\right]\e^{\lambda x}
        \left\{\begin{aligned}
             & k=0 & \p{\lambda\pm\omega \i\text{不是特征方程根}} \\
             & k=1 & \p{\lambda\pm\omega \i\text{是特征方程根}}
        \end{aligned}\right.\]
\end{theorem}

\begin{theorem}[特解的特例]
    $\omega=0$时,$m=n_1$

    \[\left.\begin{aligned}
             & \p{\ref{LinearDifferentialEquationsWithSecondOrderConstantCoefficients}} \\
             & f\p{x}=\mathcal P_m\p{x}\e^{\lambda x}
        \end{aligned}\right\}
        \implies
        y^*=x^k\mathcal Q_m\p{x}\e^{\lambda x}
        \left\{\begin{aligned}
            k=0 &  & \p{\lambda\text{不是特征方程根}} \\
            k=1 &  & \p{\lambda\text{是特征方程单根}} \\
            k=2 &  & \p{\lambda\text{是特征方程重根}}
        \end{aligned}\right.\]
\end{theorem}

\subsubsection{常系数非齐次通解的大致形式}

\[\underbrace{
        \overbrace{\C_1\overbrace{a\p{x}\e^{r_1x}}^{\text{齐次特解}}+
            \C_2
            \rlap{$\underbrace{\phantom{b\p{x}\e^{r_2x}+x^kc\p{x}\e^{\lambda x}}}_{\text{非齐次特解}}$}
            \overbrace{b\p{x}\e^{r_2x}}^{\text{齐次特解}}}^{\text{齐次通解}}+
        \overbrace{x^kc\p{x}\e^{\lambda x}}^{\text{非齐次特解}}}_{\text{非齐次通解}}\]

\subsubsection{算子法求特解}

\begin{definition}[$D$算子]
    \[Df\p{x}=f^\prime\p{x},\frac1Df\p{x}=\int f\p{x}\d x\]
\end{definition}

对于$\p{\ref{LinearDifferentialEquationsWithSecondOrderConstantCoefficients}}$:

\[y^*=\frac1{D^2+pD+q}f\p{x}=\frac1{\mathcal F\p{D}}f\p{x}\]

若代入$D$后分母$\mathcal P\p{D}$出现为$0$的状况,则(可多次使用,$D$算子只对右侧$f\p{x}$有效):

\[y^*=x^n\frac1{\mathcal P\p{D}}f\p{x}\longrightarrow y^*=x^{n+1}\frac1{\mathcal P^\prime\p{D}}f\p{x}\]

\paragraph{$f\p{x}=\C\e^{kx}$:$D$换为$k$}

\paragraph{$f\p{x}=\C\sin ax$或$\C\sin ax$:$D^2$换为$-a^2$}

若代入$D^2$后分母有$mD+n\p{mn>0}$一次多项式,可以配平方将一次多项式化到分子,再代入$D^2$后直接使用$D$算子求导

\paragraph{$f\p{x}=\mathcal P_n\p{x}$:}

使用$\dfrac1{1+x}=\sum\limits_{n\in\sn}\p{-x}^n$泰勒展开$\dfrac1{\mathcal F\p{D}}$($\mathcal F\p{D}-1$当作$x$,不考虑收敛域),
使得展开后$D$的最高次幂与$\mathcal P_n\p{x}$相同即可

\paragraph{$f\p{x}=\e^{kx}y\p{x}$:移位定理}

\[y^*=\frac1{\mathcal F\p{D}}\e^{kx}y\p{x}=\e^{kx}\frac1{\mathcal F\p{D+k}}y\p{x}\]

\paragraph{$f\p{x}=\mathcal P_n\p{x}\C\sin ax$:}

\[y^*=\frac1{\mathcal F\p{D}}\mathcal P_n\p{x}\C\sin ax=\Im\left[{\frac1{\mathcal F\p{D}}\mathcal P_n\e^{\i ax}}\right]\]

\paragraph{$f\p{x}=\mathcal P_n\p{x}\C\cos ax$:}

\[y^*=\frac1{\mathcal F\p{D}}\mathcal P_n\p{x}\C\cos ax=\Re\left[{\frac1{\mathcal F\p{D}}\mathcal P_n\e^{\i ax}}\right]\]

\section{全微分方程}

\subsection{条件(微分换序)}

\[P\p{x,y}\d x+Q\p{x,y}\d y=0\text{是全微分方程}\iff P_y^\prime=Q_x^\prime\]

\end{document}
