\documentclass{article}

\usepackage{ctex}
\usepackage{amsfonts}
\usepackage{amsmath}
\usepackage{amsthm}
\usepackage{graphicx}
\usepackage{float}
\usepackage{hyperref}
\usepackage{mathabx}
\usepackage{datetime}
\usepackage{tabularray}
\usepackage{mathrsfs}
\usepackage{geometry}

\geometry{a4paper,scale=0.8}

\title{离散数学}
\author{}
\date{\today}

\begin{document}

\hypersetup{
    hidelinks,
    %colorlinks = true,
    allcolors = black,
    %pdfstartview = Fit,
    breaklinks = true
}

\newtheorem{definition}{Definition}[subsection]
\newtheorem{theorem}{Theorem}[subsection]
\newtheorem{corollary}{Corollary}[theorem]
\renewcommand{\proofname}{\indent\bf Proof}
\numberwithin{equation}{subsection}

\def\e{\mathrm e}
\def\i{\mathrm i}
\def\j{\mathrm j}
\def\d{\mathrm d}
\def\C{\mathrm C}
\def\div{\mathrm{div}}
\def\rot{\mathrm{rot}}
\def\vecv{\vec{\mathrm v}}
\def\sr{\mathbb R}
\def\sn{\mathbb N}
\def\snp{\mathbb N^+}
\def\sc{\mathbb C}
\def\sz{\mathbb Z}
\def\impint{\int\limits_{-\infty}^{+\infty}}

\newcommand{\abs}[1]{\left|#1\right|}
\newcommand{\pare}[1]{\left(#1\right)}
\newcommand{\pair}[2]{\left<#1,#2\right>}
\newcommand{\jacobi}[2]{\frac{\partial\pare{#1}}{\partial\pare{#2}}}
\newcommand{\conditionset}[2]{\left\{#1|#2\right\}}

\begin{titlepage}
    \maketitle
\end{titlepage}

\tableofcontents
\newpage

\part{命题逻辑}

\[\begin{tblr}{c|c|c}
        \hline
        \text{名词}       & \text{符号}                    & \text{定义}                           \\
        \hline
        \text{否定(非)} & \neg P                         & \overline P                           \\
        \text{合取(与)} & P\wedge Q                      & P\cdot Q                              \\
        \text{析取(或)} & P\vee Q                        & P+Q                                   \\
        \text{条件}       & P\rightarrow Q                 & \overline P+Q                         \\
        \text{同或}       & P\leftrightarrow Q             & P\cdot Q+\overline P\cdot \overline Q \\
        \text{与非}       & P\uparrow Q                    & \overline P+\overline Q               \\
        \text{或非}       & P\downarrow Q                  & \overline P\cdot\overline Q           \\
        \text{条件否定}   & P\mapsto Q                     & P\cdot\overline Q                     \\
        \text{异或}       & P\nabla Q                      & P\cdot\overline Q+\overline P\cdot Q  \\
        \text{任意}       & \forall P\in\left\{P_i\right\} & \bigwedge P_i                         \\
        \text{存在}       & \exists P\in\left\{P_i\right\} & \bigvee P_i                           \\
        \hline
    \end{tblr}\]

\part{集合与关系}

\begin{center}
    \begin{tblr}{c|c}
        \hline
        名词   & 符号           \\
        \hline
        相等   & $A=B$          \\
        属于   & $a\in A$       \\
        子集   & $A\subseteq B$ \\
        真子集 & $A\subset B$   \\
        空集   & $\emptyset$    \\
        全集   & $E$            \\
        交     & $A\cap B$      \\
        并     & $A\cup B$      \\
        \hline
    \end{tblr}
\end{center}

\begin{center}
    \begin{tblr}{c|c}
        \hline
        名词     & 定义                                                        \\
        \hline
        平凡子集 & $A$的平凡子集为$\emptyset$和其本身$A$                       \\
        空关系   & $\emptyset$是$A\times B$的空关系                            \\
        全域关系 & $A\times B$是$A\times B$的全域关系                          \\
        恒等关系 & $I\subseteq A\times A$且$I=\conditionset{\pair xx}{x\in A}$ \\
        \hline
    \end{tblr}
\end{center}

\paragraph{以下默认条件为}

\[R\subseteq A\times A,\forall x,y,z\in A\]

\begin{center}
    \begin{tblr}{c|c}
        \hline
        名词       & 定义                                          \\
        \hline
        自反关系   & $\pair xx\in R$                               \\
        反自反关系 & $\pair xx\notin R$                            \\
        对称关系   & $\pair xy\in R\implies\pair yx\in R$          \\
        反对称关系 & $\pair xy\in R\implies x=y$                   \\
        传递关系   & $\pair xy,\pair yz\in R\implies\pair xz\in R$ \\
        等价关系   & $R$为自反关系、对称关系、传递关系             \\
        \hline
    \end{tblr}
\end{center}

\begin{center}
    \begin{tblr}{c|c|c}
        \hline
        名词         & 符号                                  & 定义或备注                                                                       \\
        \hline
        基数         & $\abs{A}$                             & $A$集合中元素的个数                                                              \\
        幂集         & $2^A$或$P\pare{A}$                    & $A$全部子集构成的集合                                                            \\
        差(相对差) & $A-B$                                 & $A$中不属于$B$元素的集合                                                         \\
        绝对差       & $~A$或$\overline A$                   & $A$对全集的差                                                                    \\
        对称差       & $A\oplus B$                           & 属于但不同时属于$A$和$B$的元素的集合                                             \\
        $n$元组      & $\left<a_1,a_2,\cdots,a_{n-1}\right>$ & $\left<\left<a_1,a_2\right>,a_3\right>=\left<a_1,a_2,a_3\right>$                 \\
        序偶         & $\pair ab$                            & 二元组                                                                           \\
        笛卡尔积     & $A\times B$                           & $\conditionset{\pair ab}{a\in A,b\in B}$($\emptyset\times\emptyset=\emptyset$) \\
        关系         & $aRb$或$\pair ab\in R$                & $a$和$b$有关系$R$                                                                \\
        无关系       & $a\overline Rb$或$\pair ab\notin R$   & $a$和$b$没有关系$R$                                                              \\
        前域         & $\mathrm{dom}\pare{R}$                & $\conditionset{x}{\pair xy\in R}$                                                \\
        值域         & $\mathrm{ran}\pare{R}$                & $\conditionset{y}{\pair xy\in R}$                                                \\
        域           & $\mathrm{FLD}\pare{R}$                & $\mathrm{dom}\pare{R}\cup\mathrm{ran}\pare{R}$                                   \\
        \hline
    \end{tblr}
\end{center}

\end{document}
