\documentclass{article}

\usepackage{ctex}
\usepackage{amsfonts}
\usepackage{amsmath}
\usepackage{amsthm}
\usepackage{graphicx}
\usepackage{float}
\usepackage{hyperref}
\usepackage{mathabx}
\usepackage{datetime}
\usepackage{tabularray}
\usepackage{mathrsfs}

\title{Signals and Systems}
\author{}
\date{\today}

\begin{document}

\hypersetup{
    hidelinks,
    %colorlinks = true,
    allcolors = black,
    %pdfstartview = Fit,
    breaklinks = true
}

\newtheorem{definition}{Definition}[subsection]
\newtheorem{theorem}{Theorem}[subsection]
\newtheorem{corollary}{Corollary}[theorem]
\renewcommand{\proofname}{\indent\bf Proof}
\numberwithin{equation}{subsection}

\def\e{\mathrm e}
\def\i{\mathrm i}
\def\j{\mathrm j}
\def\d{\mathrm d}
\def\C{\mathrm C}
\def\div{\mathrm{div}}
\def\rot{\mathrm{rot}}
\def\vecv{\vec{\mathrm v}}
\def\sr{\mathbb R}
\def\sn{\mathbb N}
\def\snp{\mathbb N^+}
\def\sc{\mathbb C}
\def\sz{\mathbb Z}
\def\impint{\int\limits_{-\infty}^{+\infty}}

\newcommand{\abs}[1]{\left|#1\right|}
\newcommand{\pare}[1]{\left(#1\right)}
\newcommand{\fourier}[1]{\mathscr F\pare{#1}}
\newcommand{\tfourier}[1]{\mathscr F^{-1}\pare{#1}}
\newcommand{\jacobi}[2]{\frac{\partial\pare{#1}}{\partial\pare{#2}}}

\begin{titlepage}
    \maketitle
\end{titlepage}

\tableofcontents
\newpage

\part{命题逻辑}

\section{数理逻辑联结词}

\subsection{基础}

\paragraph{否定(非)}

\[\neg P\]

或

\[\overline P\]

\paragraph{合取(与)}

\[P\wedge Q\]

或

\[P\cdot Q\]

\paragraph{析取(或)}

\[P\vee Q\]

或

\[P+Q\]

\subsection{推广}

\paragraph{条件}

\[P\rightarrow Q\]

\begin{definition}
    \[P\rightarrow Q
        \Leftrightarrow\neg P \vee Q
        \Leftrightarrow\overline P+Q\]
\end{definition}

\paragraph{双条件(同或)}

\[P\leftrightarrow Q\]

\begin{definition}
    \[P\leftrightarrow Q
        \Leftrightarrow\pare{P\wedge Q}\vee\pare{\neg P \vee\neg Q}
        \Leftrightarrow P\cdot Q+\overline P\cdot \overline Q\]
\end{definition}

\paragraph{与非}

\[P\uparrow Q\]

\begin{definition}
    \[P\uparrow Q
        \Leftrightarrow\neg P \vee\neg Q
        \Leftrightarrow\overline P+\overline Q\]
\end{definition}

\paragraph{或非}

\[P\downarrow Q\]

\begin{definition}
    \[P\downarrow Q
        \Leftrightarrow\neg P \wedge\neg Q
        \Leftrightarrow\overline P\cdot\overline Q\]
\end{definition}

\paragraph{条件否定}

\[P\mapsto Q\]

\begin{definition}
    \[P\mapsto Q
        \Leftrightarrow P \vee\neg Q
        \Leftrightarrow P\cdot\overline Q\]
\end{definition}

\paragraph{异或}

\[P\nabla Q\]

\begin{definition}
    \[P\nabla Q
        \Leftrightarrow\pare{P\wedge\neg Q}\vee\pare{\neg P\wedge Q}
        \Leftrightarrow P\cdot\overline Q+\overline P\cdot Q\]
\end{definition}

\section{谓词逻辑联结词}

\paragraph{任意}

\[\forall P\in\left\{P_i\right\}\]

\begin{definition}
    \[\forall P\in\left\{P_i\right\}
        \Leftrightarrow P_1\wedge P_2\wedge\cdots
        \Leftrightarrow\bigwedge P_i\]
\end{definition}

\paragraph{存在}

\[\exists P\in\left\{P_i\right\}\]

\begin{definition}
    \[\exists P\in\left\{P_i\right\}
        \Leftrightarrow P_1\vee P_2\vee\cdots
        \Leftrightarrow\bigvee P_i\]
\end{definition}


\[\begin{tblr}{c|c|c}
        \hline
        \hline
        \text{名词}       & \text{符号}                    & \text{定义}                           \\
        \hline
        \text{否定(非)} & \neg P                         & \overline P                           \\
        \hline
        \text{合取(与)} & P\wedge Q                      & P\cdot Q                              \\
        \hline
        \text{析取(或)} & P\vee Q                        & P+Q                                   \\
        \hline
        \text{条件}       & P\rightarrow Q                 & \overline P+Q                         \\
        \hline
        \text{同或}       & P\leftrightarrow Q             & P\cdot Q+\overline P\cdot \overline Q \\
        \hline
        \text{与非}       & P\uparrow Q                    & \overline P+\overline Q               \\
        \hline
        \text{或非}       & P\downarrow Q                  & \overline P\cdot\overline Q           \\
        \hline
        \text{条件否定}   & P\mapsto Q                     & P\cdot\overline Q                     \\
        \hline
        \text{异或}       & P\nabla Q                      & P\cdot\overline Q+\overline P\cdot Q  \\
        \hline
        \text{任意}       & \forall P\in\left\{P_i\right\} & \bigwedge P_i                         \\
        \hline
        \text{存在}       & \exists P\in\left\{P_i\right\} & \bigvee P_i                           \\
        \hline
        \hline
    \end{tblr}\]

\end{document}
