\documentclass{article}

\usepackage{ctex}
\usepackage{amsfonts}
\usepackage{amsmath}
\usepackage{amsthm}
\usepackage{graphicx}
\usepackage{float}
\usepackage{hyperref}
\usepackage{mathabx}
\usepackage{datetime}
\usepackage{tabularray}
\usepackage{geometry}

\geometry{a4paper,scale=0.8}

\title{高等数学}
\author{}
\date{\today}

\geometry{a4paper,scale=0.8}

\begin{document}

\hypersetup{
    hidelinks,
    %colorlinks = true,
    allcolors = black,
    %pdfstartview = Fit,
    breaklinks = true
}

\newtheorem{definition}{Definition}[subsection]
\newtheorem{theorem}{Theorem}[subsection]
\newtheorem{corollary}{Corollary}[theorem]
\renewcommand{\proofname}{\indent\bf Proof}
\numberwithin{equation}{section}

\def\e{\mathrm e}
\def\i{\mathrm i}
\def\d{\mathrm d}
\def\C{\mathrm C}
\def\vecv{\vec{\mathrm v}}
\def\sr{\mathbb R}
\def\sn{\mathbb N}
\def\snp{\mathbb N^+}
\def\sc{\mathbb C}
\def\sz{\mathbb Z}

\newcommand{\abs}[1]{\left|#1\right|}
\newcommand{\p}[1]{\left(#1\right)}
\newcommand{\jacobi}[2]{\frac{\partial\p{#1}}{\partial\p{#2}}}

\begin{titlepage}
    \maketitle
\end{titlepage}

\tableofcontents
\newpage

\part{极限}

\section{基础}

\subsection{常用极限}

\[\begin{aligned}
         & \lim_{x\to\infty}{\p{1+\frac1x}^x}                  &  & =\e                            \\
         & \lim_{n\to\infty}{\frac1n\sum_{i=1}^nf\p{\frac in}} &  & =\int_0^1f\p x\d x\p{n\in\snp}
    \end{aligned}\]

\subsection{常用等价无穷小}

$x$为函数,$\lim\limits_{x\to0}$时

\[x\sim\sin x\sim\tan x\sim\arcsin x\sim\arctan x\]

\[x\sim\p{\e^x-1}\sim\ln\p{x+1}\sim\ln\p{x+\sqrt{1+x^2}}\]

\[x^3\sim6\p{x-\sin x}\sim6\p{\arcsin x-x}\sim3\p{\tan x-x}\]

\[x^3\sim3\p{x-\arctan x}\sim2\p{\tan x -\sin x}\]

\[\begin{aligned}
         & 1-\cos x            &  & \sim\frac{x^2}2                    \\
         & \log_a{\p{1+x}}     &  & \sim\frac x{\ln a}                 \\
         & \p{1+x}^a           &  & \sim ax+1                          \\
         & a^x-1               &  & \sim x\ln a\p{0<a\neq1}            \\
         & \p{1+ax}^\frac1{bx} &  & \sim\e^\frac ab(1-\frac{a^2}{2b}x) \\
    \end{aligned}\]

\part{导数}

\section{基础}

\subsection{常用高阶导数}

\[\begin{aligned}
         & \sin^{\p n}\omega x &  & =\omega^n\sin\p{\omega x+\frac{n\pi}2}   &  & \p{n\in\sn}  \\
         & \cos^{\p n}\omega x &  & =\omega^n\cos{\p{\omega x+\frac{n\pi}2}} &  & \p{n\in\sn}  \\
         & \ln^{\p n}\p{1+x}   &  & =\p{-1}^{n-1}\frac{\p{n-1}!}{\p{1+x}^n}  &  & \p{n\in\snp} \\
         & \ln^{\p n}\p{1-x}   &  & =-\frac{\p{n-1}!}{\p{1-x}^n}             &  & \p{n\in\snp} \\
    \end{aligned}\]

\subsection{莱布尼茨公式}

\[\p{uv}^{\p n}=\sum_{k=0}^n\binom nku^{\p{n-k}}v^{\p k}\]

\subsection{中值定理}

\[\begin{tblr}{c|c|c}
        \hline
        \text{定理}             & \text{公式}                                                                                     & \text{约束}            \\
        \hline
        \text{积分中值定理}     & f\p{\xi}=\dfrac{\int_a^bf\p x\d x}{\left.x\right|_a^b}                                          & \xi\in\left[a,b\right] \\
        \text{罗尔中值定理}     & f^\prime\p{\xi}=0                                                                               & \xi\in\p{a,b}          \\
        \text{拉格朗日中值定理} & f^\prime\p{\xi}=\dfrac{\left.f\p x\right|_a^b}{\left.x\right|_a^b}                              & \xi\in\p{a,b}          \\
        \text{柯西中值定理}     & \dfrac{f^\prime\p{\xi}}{g^\prime\p{\xi}}=\dfrac{\left.f\p x\right|_a^b}{\left.g\p x\right|_a^b} & \xi\in\p{a,b}          \\
        \hline
    \end{tblr}\]

\subsection{泰勒中值定理}

$R_n\p x$为余项

\[\begin{aligned}
         & P_n\p x    &  & =\sum_{i=0}^n\left[\p{x-x_0}\frac{\d}{\d x}\right]^i\frac{f\p{x_0}}{i!}+R_n\p x                                \\
         & P_n\p{x,y} &  & =\sum_{i=0}^n\left[\p{x-x_0}\partial_x+\p{y-y_0}\partial_y\right]^i\frac{f\p{x_0,y_0}}{i!}+R_n\p{x,y}\label{2}
    \end{aligned}\]

\subsubsection{拉格朗日型余项\label{Lagrange}}

$\theta\in\p{0,1}$

\[\begin{aligned}
        R_n\p x    & =
        \left[\p{x-x_0}\frac{\d}{\d x}\right]^{n+1}\frac{f\p{x_0+\theta\p{x-x_0}}}{\p{n+1}!} \\
        R_n\p{x,y} & =
        \left[\p{x-x_0}\partial_x+\p{y-y_0}\partial_y\right]^i\frac{f\p{x_0+\theta\p{x-x_0},y_0+\theta\p{y-y_0}}}{\p{n+1}!}
    \end{aligned}\]

\subsubsection{佩亚诺型余项}

\[\begin{aligned}
        R_n\p x    & =
        o\left[\p{x-x_0}^n\right] \\
        R_n\p{x,y} & =
        o\left[\sqrt{\p{x-x_0}^2+\p{y-y_0}^2}^n\right]
    \end{aligned}\]

\subsubsection{误差估计式}

\[n\in\sn;\exists M>0\forall x\in D\to M\geqslant\abs{f^{\p{n+1}}\p{\xi}}\]

\[\implies\abs{R_n\p x}\leqslant M\cdot\frac{\abs{x-x_0}^{n+1}}{\p{n+1}!}\]

\subsubsection{特别的:麦克劳林公式}

\[\left.\begin{aligned}
         & \p{\ref{Lagrange}} \\
         & x_0=y_0=0
    \end{aligned}\right\}
    \implies
    \left\{\begin{aligned}
        P_n\p x    & =\sum_{i=0}^n\p{x\frac\d{\d x}}^i\frac{f\p{0}}{i!}+R_n\p x              \\
        P_n\p{x,y} & =\sum_{i=0}^n\p{x\partial_x+y\partial_y}^i\frac{f\p{0,0}}{i!}+R_n\p{x,y}
    \end{aligned}\right.\]

\subsubsection{常用麦克劳林公式}

\paragraph{$\cos x$的$2k$和$2k+1$阶}

\[\cos x=\sum_{i=0}^k\p{-1}^i\frac{x^{2i}}{2i!}+\p{-1}^{k+1}\cos\theta x\frac{x^{2k+2}}{\p{2k+2}!}\]

\paragraph{$\sin x$的$2k-1$和$2k$阶}

\[\sin x=\sum_{i=1}^k\p{-1}^{i-1}\frac{x^{2i-1}}{\p{2i-1}!}+\p{-1}^k\cos\theta x\frac{x^{2k+1}}{\p{2k+1}!}\]

\paragraph{其他函数的$n$阶}

\[\begin{aligned}
        \e^x           & =\sum_{i=0}^n\frac{x^i}{i!}+\e^{\theta x}\frac{x^{n+1}}{\p{n+1}!}                                                                                                 \\
        \ln\p{1+x}     & =\sum_{i=0}^n\p{-1}^{i-1}\frac{x^i}{i}+\frac{\p{-1}^n}{\p{1+\theta x}^{n+1}}\cdot\frac{x^{n+1}}{\p{n+1}}\p{x>-1}                                                  \\
        \p{1+x}^\alpha & =\sum_{i=0}^n\p{\prod_{j=1}^n\p{\alpha-j+1}\cdot\frac{x^i}{i!}}+\frac{\prod\limits_{i=0}^n\p{\alpha-i}}{\p{1+\theta x}^{n+1-\alpha}}\cdot\frac{x^{n+1}}{\p{n+1}!}
    \end{aligned}\]

\subsection{极值}

\[\left.\begin{aligned}
        f^\prime_x\p{x_0,y_0}            & =0                                                                \\
        f^\prime_y\p{x_0,y_0}            & =0                                                                \\
        f^{\prime\prime}_{xy}\p{x_0,y_0} & <f^{\prime\prime}_{xx}\p{x_0,y_0}f^{\prime\prime}_{yy}\p{x_0,y_0}
    \end{aligned}\right\}
    \implies
    f{\p{x_0,y_0}}\text{为极值点}\]

\[\begin{aligned}
        f^{\prime\prime}_{xy}\p{x_0,y_0} & >f^{\prime\prime}_{xx}\p{x_0,y_0}f^{\prime\prime}_{yy}\p{x_0,y_0} &  &
        f{\p{x_0,y_0}}\text{不取极值}                                                                             \\
        f^{\prime\prime}_{xy}\p{x_0,y_0} & =f^{\prime\prime}_{xx}\p{x_0,y_0}f^{\prime\prime}_{yy}\p{x_0,y_0} &  &
        \text{需进一步讨论}
    \end{aligned}\]

\subsection{拉格朗日乘数法(求条件极值)}

\paragraph{二元情况}

\[\begin{aligned}
         & \left\{\begin{aligned}
                      \text{约束条件:} & \varphi\p{x,y}=0 \\
                      \text{目标函数:} & f\p{x,y}
                  \end{aligned}\right.                                                      \\
         & \left\{\begin{aligned}
                       & \nabla f=\lambda\nabla\varphi\p{\text{即}\nabla f\parallel\nabla\varphi} \\
                       & \varphi\p{x,y}=0
                  \end{aligned}\right. \\
         & \implies
        \text{解得几组}\p{x,y}\text{即为极值点}
    \end{aligned}\]

\paragraph{$n$元情况}

\[\def\xs{\p{x_1,x_2,\cdots,x_n}}
    \begin{aligned}
         & \left\{\begin{aligned}
                      \text{约束条件:} & \varphi_1\xs=0     \\
                                        & \varphi_2\xs=0     \\
                                        & \vdots             \\
                                        & \varphi_{n-1}\xs=0 \\
                      \text{目标函数:} & f\xs
                  \end{aligned}\right.                          \\
         & \left\{\begin{aligned}
                       & \nabla f=\sum_i\lambda_i\nabla\varphi_i\text{(三元时共面)} \\
                       & \varphi_1\xs=0                                               \\
                       & \varphi_2\xs=0                                               \\
                       & \vdots                                                       \\
                       & \varphi_{n-1}\xs=0
                  \end{aligned}\right. \\
         & \implies\text{解得几组}\xs\text{即为极值点}
    \end{aligned}\]

\subsection{雅可比行列式}

\[\frac{\partial\p{u_1,u_2,\cdots,u_n}}
    {\partial\p{x_1,x_2,\cdots,x_n}}=
    \begin{vmatrix}
        \partial_{u_1}x_1 & \partial_{u_1}x_2 &
        \cdots            & \partial_{u_1}x_n                   \\
        \partial_{u_2}x_1 & \partial_{u_2}x_2 &
        \cdots            & \partial_{u_2}x_n                   \\
        \vdots            & \vdots            & \ddots & \vdots \\
        \partial_{u_n}x_1 & \partial_{u_n}x_2 &
        \cdots            & \partial_{u_n}x_n
    \end{vmatrix}\]

\part{积分}

\section{基础}

\subsection{牛顿-莱布尼茨公式}

\[\int_a^b{f^\prime\p x\d x}=\left.f\p x\right|_a^b\]

\subsection{第一类换元(凑微分)法}

\[\int f\p xg\p x\d x=\int f\p x\d\p{\int g\p x\d x}\]

\subsection{第二类换元法}

\[\begin{aligned}
         & \int f\p x\d x                        &  & =\left.\int f\p t\d t\right|_{t=\varphi\p x}                                                                \\
         & \int_a^bf\left[\varphi\p x\right]\d x &  & =\left.\int_{\varphi\p a}^{\varphi\p b}{f\p t\frac{{\d\varphi}^{-1}\p t}{ \d t}\d t}\right|_{t=\varphi\p x}
    \end{aligned}\]

\subsection{分部积分}

\[\begin{aligned}
         & u_xv_x                  &  & =\int{u_xdv_x}     &  & +\int{v_xdu_x}     \\
         & \left.u_xv_x\right|_a^b &  & =\int_a^b{u_xdv_x} &  & +\int_a^b{v_xdu_x}\end{aligned}\]

\subsection{常用积分表}

\[\begin{aligned}
         & \int\sinh x\d x                 &  & =\cosh x                            &  & +\C \\
         & \int\cosh x\d x                 &  & =\sinh x                            &  & +\C \\
         & \int\sec^2 x\d x                &  & =\tan x                             &  & +\C \\
         & \int\csc^2 x\d x                &  & =-\cot x                            &  & +\C \\
         & \int\sec x\tan x\d x            &  & =\sec x                             &  & +\C \\
         & \int\csc x\cot x\d x            &  & =-\csc x                            &  & +\C \\
         & \int\tan x\d x                  &  & =-\ln\abs{\cos x}                   &  & +\C \\
         & \int\cot x\d x                  &  & =\ln\abs{\sin x}                    &  & +\C \\
         & \int\sec x\d x                  &  & =\ln\abs{\sec x+\tan x}             &  & +\C \\
         & \int\csc x\d x                  &  & =\ln\abs{\csc x-\cot x}             &  & +\C \\
         & \int\frac{\d x}{x^2-a^2}        &  & =\frac1{2a}\ln\abs{\frac{x-a}{x+a}} &  & +\C \\
         & \int\frac{\d x}{\sqrt{x^2+a^2}} &  & =\ln\p{x+\sqrt{x^2+a^2}}            &  & +\C \\
         & \int\frac{\d x}{\sqrt{x^2-a^2}} &  & =\ln\abs{x+\sqrt{x^2-a^2}}          &  & +\C \\
         & \int\frac{\d x}{a^2+x^2}        &  & =\frac1a\arctan\frac xa             &  & +\C \\
         & \int\frac{\d x}{\sqrt{a^2-x^2}} &  & =\arcsin{\frac xa}                  &  & +\C
    \end{aligned}\]

\subsection{有理函数积分通解(递推)}

\[\int{\frac{x+N}{\p{x^2+px+q}^\lambda}\d x}
    \left\{\begin{aligned}
        0 & >p^2-4q               \\
        a & =\sqrt{q-\frac{p^2}4} \\
        b & =N-\frac p2
    \end{aligned}\right.\]

\[=\left\{\begin{aligned}
         & \frac{2bx+bp-2a^2}{4\p{\lambda-1}a^2\p{x^2+px+q}^{\lambda-1}}+\frac{b\p{2\lambda-3}}{2\p{\lambda-1}a^2}\int\frac{\d x}{\p{x^2+px+q}^{\lambda-1}} &  &
        \p{\lambda>1}                                                                                                                                            \\
         & \frac{\ln\p{x^2+px+q}}2+\frac ba\arctan{\frac{x+2p}{2a}}+\C                                                                                      &  &
        \p{\lambda=1}                                                                                                                                            \\
    \end{aligned}\right.\]

\subsection{万能代换}

\[x=2\arctan u\implies
    \left\{\begin{aligned}
        \sin x & =\frac{2u}{1+u^2}      \\
        \cos x & =\frac{1-u^2}{1+u^2}   \\
        \d x   & =\frac2{\p{1+u^2}}\d u
    \end{aligned}\right.\]

\subsection{极坐标图形面积}

\[A=\frac12\int_{\alpha}^{\beta}{r^2\p{\theta}\d\theta}\]

\begin{definition}[以下参数方程中都有]
    \[\left\{\begin{aligned}
            x & =x\p t \\
            y & =y\p t
        \end{aligned}\right.\]
\end{definition}

\subsection{旋转体体积(参数方程)}

绕$x$轴

\[V=\pi\int_a^b{{x^\prime y}^2dt}\]

\subsection{旋转体侧面积(参数方程)(可轮换)}

绕$x$轴

\[S=2\pi\int_a^b{x\sqrt{x^{\prime2}+y^{\prime2}}\d t}\]

\subsection{平面曲线弧长(参数方程)(可轮换)}

\[s=\int_a^b{\sqrt{{x^\prime}^2+{y^\prime}^2}\d t}=\int_{\alpha}^{\beta}{\sqrt{r^2\p{\theta}+r^{\prime2}\p{\theta}}\d\theta}\]

\subsection{平面曲线曲率(参数方程)}

曲率半径:$K^{-1}$

\[K=\frac{\abs{x^\prime y^{\prime\prime}-x^{\prime\prime}y^\prime}}{\p{x^{\prime2}+y^{\prime2}}^\frac32}\]

\section{重积分}

\subsection{二重积分}

\begin{definition}[$\d\sigma=\d x\d y$]
    \[\iint_Df\p{x,y}\d\sigma\]
\end{definition}

\subsubsection{换元}

\[\left.\begin{aligned}
        \left\{\begin{aligned}
                   x=x\p{u,v} \\
                   y=y\p{u,v}
               \end{aligned}\right. \\
        \left.\jacobi{x,y}{u,v}\right|_{D^\prime}\neq0
    \end{aligned}\right\}\implies\\
    \iint_Df\p{x,y}\d x\d y=
    \iint_{D^\prime}f\p{x,y}\abs{\jacobi{x,y}{u,v}}\d u\d v\]

\subsubsection{广义极坐标变换}

\[\left\{\begin{aligned}
        x\p{r,\theta} & =x_0+ar\cos\theta \\
        y\p{r,\theta} & =y_0+br\sin\theta
    \end{aligned}\right.\implies\\
    \iint_Df\p{x,y}\d x\d y=
    ab\iint_Df\p{x,y}r\d r\d\theta\]

\subsection{三重积分}

\begin{definition}[$\d V=\d x\d y\d z$]
    \[\iiint_\Omega f\p{x,y,z}\d V\]
\end{definition}

\subsubsection{换元}

\[\left.\begin{aligned}
        \left\{\begin{aligned}
                   x & =x\p{u,v,w} \\
                   y & =y\p{u,v,w} \\
                   z & =z\p{u,v,w}
               \end{aligned}\right. \\
        \left.\jacobi{x,y,z}{u,v,w}\right|_{\Omega^\prime}\neq0
    \end{aligned}\right\}\implies
    \iiint_\Omega f\p{x,y,z}\d x\d y\d z=
    \iiint_{\Omega^\prime}f\p{x,y,z}\abs{\jacobi{x,y,z}{u,v,w}}\d u\d v\d w\]

\subsubsection{柱面坐标}

\[\left\{\begin{aligned}
        x=x\p{r,\theta,z} & =x_0+ar\cos\theta \\
        y=y\p{r,\theta,z} & =y_0+br\sin\theta \\
        z=z\p{r,\theta,z} & =z
    \end{aligned}\right.\]

\[
    \iiint_\Omega f\p{x,y,z}\d x\d y\d z=
    \iiint_\Omega f\p{x,y,z}r\d r\d\theta\d z\]

\subsubsection{球面坐标}

\[\left\{\begin{aligned}
        x=x\p{r,\varphi,\theta} & =\rho\sin\varphi\cos\theta \\
        y=y\p{r,\varphi,\theta} & =\rho\sin\varphi\sin\theta \\
        z=z\p{r,\varphi,\theta} & =\rho\cos\varphi           \\
    \end{aligned}\right.\]

\[
    \iiint_\Omega f\p{x,y,z}\d x\d y\d z=
    \iiint_\Omega f\p{x,y,z}\rho^2\sin\varphi\d\rho\d\varphi\d\theta\]

\subsubsection{曲面面积(可轮换)}

\[\left.\begin{aligned}
        z       & =z\p{x,y}  \\
        \p{x,y} & \in D_{xy}
    \end{aligned}\right\}\implies
    S=\iint_{D_{xy}}\sqrt{1+z^{\prime2}_x+z^{\prime2}_y}\d x\d y\]

\section{曲线与曲面积分}

\subsection{曲线积分}

\begin{definition}[]
    \[\left\{\begin{aligned}
            x & =x\p t \\
            y & =y\p t \\
            z & =z\p t
        \end{aligned}\right.
        \left\{\begin{aligned}
            P & =P\p{x,y,z} \\
            Q & =Q\p{x,y,z} \\
            R & =R\p{x,y,z}
        \end{aligned}\right.\]
\end{definition}

\begin{definition}[第一类]

    $t\in\left[\alpha,\beta\right]$

    \[\int_\Gamma f\p{x,y,z}\d s=
        \int^\beta_\alpha f\p{x,y,z}\sqrt{x^{\prime 2}+y^{\prime 2}+z^{\prime 2}}\d t\]
\end{definition}

\begin{definition}[第二类(坐标积分)]

    $t:\alpha\to\beta$

    \[\int_\Gamma P\d x+Q\d y+R\d z=
        \int^\beta_\alpha\p{Px^\prime+Qy^\prime+Rz^\prime}\d t\]
\end{definition}

\subsubsection{格林公式}$L$围成$D$

\[\oint_LP\d x+Q\d y=
    \iint_D\p{Q^\prime_x-P^\prime_y}\d x\d y\]

\subsubsection{平面曲线积分与积分路径无关条件}

$\int_LP\d x+Q\d y$与积分路径无关

\[\begin{aligned}
             & \int_LP\d x+Q\d y=\int^B_AP\d x+Q\d y \\
        \iff & \oint_LP\d x+Q\d y=0                  \\
        \iff & \exists u=u\p{x,y},\d u=P\d x+Q\d y   \\
        \iff & D内,Q^\prime_x= P^\prime_y
    \end{aligned}\]

\subsubsection{曲线积分路径无关}

\[\int_LP\d x+Q\d y\to Q_x^\prime=P_y^\prime\]

\[\int_\Gamma P\d y\d z+Q\d z\d x+R\d x\d y\to\left\{\begin{aligned}
        R_y^\prime=Q_z^\prime \\
        P_z^\prime=R_x^\prime \\
        Q_x^\prime=P_y^\prime
    \end{aligned}\right.\]

\subsection{曲面积分}

\begin{definition}[]
    \[z=z\p{x,y}(可轮换)
        \left\{\begin{aligned}
            P & =P\p{x,y,z} \\
            Q & =Q\p{x,y,z} \\
            R & =R\p{x,y,z}
        \end{aligned}\right.\]
\end{definition}

\begin{definition}[第一类(可轮换)]
    \[t\in\left[\alpha,\beta\right]\]

    \[\iint_\Sigma f\p{x,y,z}\d S=
        \iint_{D_{xy}} f\p{x,y,z}\sqrt{z_x^{\prime 2}+z_y^{\prime 2}+1}\d x\d y\]
\end{definition}

\begin{definition}[第二类(坐标积分)(外(远离原点)侧取正,内(指向原点)侧取负)]
    \[\iint_\Sigma P\d y\d z+Q\d z\d x+R\d x\d y=\pm
        \iint_\Sigma\frac
        {-Pz_x^\prime-Qz_y^\prime+R}
        {\sqrt{z_x^{\prime2}+z_y^{\prime2}+1}}\d S\]
\end{definition}

\subsubsection{三合一投影法(外侧取正,内侧取负)}

\[\iint_\Sigma P\d y\d z+Q\d z\d x+R\d x\d y=
    \pm\iint_{D_{xy}}
    \p{-Pz_x^\prime-Qz_y^\prime+R}
    \d x\d y\]

\subsubsection{高斯公式}

\[\oiint_\Sigma P\d y\d z+Q\d z\d x+R\d x\d y=
    \iiint_\Omega\p{P_x^\prime+Q_y^\prime+R_z^\prime}\d V\]

\subsubsection{曲面积分路径无关}

\[\iint_\Sigma P\d y\d z+Q\d z\d x+R\d x\d y
    \to P_x^\prime+Q_y^\prime+R_z^\prime=0\]

\section{斯托克斯公式}

\[\oint_\Gamma P\d x+Q\d y+R\d z=\iint_\Sigma
    \p{R_y^\prime-Q_z^\prime}\d y\d z+
    \p{P_z^\prime-R_x^\prime}\d z\d x+
    \p{Q_x^\prime-P_y^\prime}\d x\d y\]

\section{向量分析}

\begin{definition}[向量场]
    \[\vec\psi=\left\{P,Q,R\right\}\]
\end{definition}

\subsection{梯度}

\begin{definition}[梯度]
    \[\nabla=\left\{\partial_x,\partial_y,\partial_z\right\}\]
\end{definition}

\subsection{散度}

\begin{definition}[通过$\Sigma$流向指定侧的通量]
    \[\Phi=\iint_\Sigma P\d y\d z+Q\d z\d x+R\d x\d y\]
\end{definition}

\begin{theorem}[散度]
    \[\mathrm{div}\vec\psi=
        \nabla\cdot\vec \psi=
        P_x^\prime+Q_y^\prime+R_z^\prime\]
\end{theorem}

\subsection{旋度}

\begin{definition}[沿封闭曲线$\Gamma$的环流量]
    \[\oint_\Gamma P\d x+Q\d y+R \d z\]
\end{definition}

\begin{theorem}[旋度]
    \[\mathrm{rot}\vec\psi=
        \nabla\times\vec\psi=\left\{
        R_y^\prime-Q_z^\prime,
        P_z^\prime-R_x^\prime,
        Q_x^\prime-P_y^\prime
        \right\}\]
\end{theorem}

\part{微分方程}

\section{基础}

\begin{definition}[$n$阶线性微分方程]
    \[y^{\p n}+\sum_{i=0}^{n-1}p_i\p xy^{\p i}=f\p x\label{LinearDifferentialEquationsOfOrderN}\]
\end{definition}

\subsection{线性相关}

\[\frac{f\p x}{g\p x}=\C(\C\in\sc)\]

\subsection{伯努利方程}
\[y^\prime+P\p xy=Q\p xy^\alpha\overset{z=y^{1-\alpha}}{=\!\!\!\implies} z^\prime+\p{1-\alpha}P\p xz=\p{1-\alpha}Q\p x\]
\section{一阶线性微分方程}

\begin{definition}[$f\p x\equiv0$时,为齐次]
    \[\left.\begin{aligned}
            \p{\ref{LinearDifferentialEquationsOfOrderN}} \\
            n=1
        \end{aligned}\right\}
        \implies
        y^\prime+P\p xy=f\p x\]
\end{definition}

\begin{theorem}[通解]
    \[y=\frac{\int{f\p x\exp\p{\int P\p x\d x}\d x}+\C}{\exp\p{\int P\p x\d x}}\]
\end{theorem}

\section{二阶线性微分方程}

\begin{definition}[]
    \[y^{\prime\prime}+P\p xy^\prime+Q\p xy=f\p x\]
\end{definition}

\subsection{齐次、非齐次、通解、特解关系}

齐特+齐特(线性无关)=齐通

齐通+非特=非通

齐特+非特=非特

非特$-$非特=齐特

\section{n阶常系数线性齐次微分方程}

\begin{definition}[]
    \[y^{\p n}+\sum_{i=0}^{n-1}p_iy^{\p i}=0\p{p_i\in\sc}\]
\end{definition}

\subsection{特征方程}

\[r^n+\sum_{i=0}^{n-1}p_ir^i=0\]

\subsection{对应项}

\paragraph{$k$重实根$r$在通解中对应项}

\[y_r=\sum_{i=1}^k\C_ix^{i-1}\cdot\e^{rx}\]

\paragraph{特别的:$r$为复根时,可改写为两个实根}

\[\C e^{rx}=(\C_1\cos{\beta x}+\C_2\sin{\beta x})\e^{\alpha x}\p{r=\alpha\pm\beta \i}\]

\section{二阶常系数线性微分方程}

\begin{definition}[]
    \[y^{\prime\prime}+py^\prime+qy=f\p x\label{LinearDifferentialEquationsWithSecondOrderConstantCoefficients}\]
\end{definition}


\begin{theorem}[特解]
    \[\left.\begin{aligned}
            \p{\ref{LinearDifferentialEquationsWithSecondOrderConstantCoefficients}} &                                                                                                   \\
            f\p x                            & =\left[\mathcal P_{n_1}\p x\cos{\omega x}+\mathcal P_{n_2}\p x\sin{\omega x}\right]\e^{\lambda x} \\
            m                                & =\max{\left\{n_1,n_2\right\}}
        \end{aligned}\right\}
        \implies\]

    \[y^*=x^k\left[\mathcal U_m\p x\cos{\omega x}+\mathcal V_m\p x\sin{\omega x}\right]\e^{\lambda x}
        \left\{\begin{aligned}
             & k=0 & \p{\lambda\pm\omega \i\text{不是特征方程根}} \\
             & k=1 & \p{\lambda\pm\omega \i\text{是特征方程根}}
        \end{aligned}\right.\]
\end{theorem}

\begin{theorem}[特解的特解]
    $\omega=0$时,即$m=n_1$

    \[\left.\begin{aligned}
             & \p{\ref{LinearDifferentialEquationsWithSecondOrderConstantCoefficients}}     \\
             & f\p x=\mathcal P_m\p x\e^{\lambda x}
        \end{aligned}\right\}
        \implies
        y^*=x^k\mathcal Q_m\p x\e^{\lambda x}
        \left\{\begin{aligned}
            k=0 &  & \p{\lambda\text{不是特征方程根}} \\
            k=1 &  & \p{\lambda\text{是特征方程单根}} \\
            k=2 &  & \p{\lambda\text{是特征方程重根}}
        \end{aligned}\right.\]
\end{theorem}

\section{全微分方程}

\subsection{条件}

\paragraph{$P\p{x,y}\d x+Q\p{x,y}\d y=0$是全微分方程的条件(微分换序)}

\[P_y^\prime=Q_x^\prime\]

\part{空间解析几何}

\section{基础}

\subsection{向量的方向余弦}

\[\vecv^0=
    \begin{bmatrix}
        \cos\alpha \\
        \cos\beta  \\
        \cos\gamma
    \end{bmatrix}=
    \frac1{\left\Vert\vecv\right\Vert}
    \begin{bmatrix}\vecv_x\\\vecv_y\\\vecv_z\end{bmatrix}\]

\section{空间曲面}

\subsection{基础}

\begin{definition}[]
    \[F\p{x,y,z}=0\]
\end{definition}

\subsubsection{法向量}

\[\nabla F\]

\subsection{平面}

\begin{definition}[]
    \[Ax+By+Cz+D=0\]
\end{definition}

\subsubsection{平面点法式}

过$\p{x_0,y_0,z_0}$,法向量$\begin{bmatrix}A\\B\\C\end{bmatrix}$

\[A\p{x-x_0}+
    B\p{y-y_0}+
    C\p{z-z_0}=0\]

\subsubsection{平面截距式}

\[\frac xa+\frac yb+\frac zc=1\]

\section{空间曲线}

\begin{definition}[]
    \[\left\{\begin{aligned}
            F\p{x,y,z} & =0 \\
            G\p{x,y,z} & =0
        \end{aligned}\right.\]
\end{definition}

\subsection{参数方程}

\[\left\{\begin{aligned}
        x & =x\p t \\
        y & =y\p t \\
        z & =z\p t
    \end{aligned}\right.\]

\subsection{切向量}

\[\vec\tau=\nabla F\times\nabla G\]

\subsection{直线}

\begin{definition}[]
    \[\left\{\begin{aligned}
            A_1x+B_1y+C_1z+D_1 & =0 \\
            A_2x+B_2y+C_2z+D_2 & =0
        \end{aligned}\right.\]
\end{definition}

\subsection{直线对称式(点向式)方程}

过$\p{x_0,y_0,z_0}$,方向向量$\begin{bmatrix}m\\n\\p\end{bmatrix}$

\[\frac{x-x_0}m=\frac{y-y_0}n=\frac{z-z_0}p=t\]

\subsection{直线参数方程}

\[\left\{\begin{aligned}
        x & =x_0+mt \\
        y & =y_0+nt \\
        z & =z_0+pt
    \end{aligned}\right.\]

\section{特殊曲面}

\begin{definition}[绕$z$轴旋转曲面:(原曲线为$f\p{y_1,z}=0$)]
    \[\left.\begin{aligned}
            f\p{y_1,z}     & =0         \\
            \sqrt{x^2+y^2} & =\abs{y_1}
        \end{aligned}\right\}
        \implies
        f\p{\pm\sqrt{x^2+y^2},z}=0\]
\end{definition}

\subsection{圆锥面}

\[z^2=\cot^2\alpha\cdot\p{x^2+y^2}\]

\begin{definition}[以下二次曲面方程中都有]
    \[pq>0\]
\end{definition}

\subsection{椭球面}

\[\frac{x^2}{a^2}+\frac{y^2}{b^2}+\frac{z^2}{c^2}=1\]

\subsection{椭圆抛物面}

\[\frac{x^2}{2p}+\frac{y^2}{2q}=z\]

\subsection{双曲抛物面(马鞍面)}

\[-\frac{x^2}{2p}+\frac{y^2}{2q}=z\]

\[z=xy\]

\subsection{单叶双曲面}

\[\frac{x^2}{a^2}+\frac{y^2}{b^2}-\frac{z^2}{c^2}=1\]

\subsection{双叶双曲面}

\[\frac{x^2}{a^2}+\frac{y^2}{b^2}-\frac{z^2}{c^2}=-1\]

\part{级数}

\section{收敛与发散}

\subsection{条件收敛}

\[\sum_{n\in\snp}u_n=s,s\in\sc\]

\subsection{绝对收敛}

\[\sum_{n\in\snp}\abs{u_n}=s,s\in\sc\]

\[\text{绝对收敛}\implies\text{条件收敛}\]

\section{正项级数}

\subsection{积分审敛法}

\[\sum_{n\in\snp}f\p n\text{与}
    \int_1^{+\infty}f\p x\d x\text{敛散同}\]

\subsection{比较审敛法}

\[\lim\frac{u_n}{v_n}
    \left\{\begin{aligned}
         & =0               & \implies & \sum v_n\text{收敛则}\sum u_n\text{收敛} \\
         & \in\p{0,+\infty} & \implies & \sum v_n,\sum u_n\text{敛散同}           \\
         & =+\infty         & \implies & \sum v_n\text{发散则}\sum u_n\text{发散}
    \end{aligned}\right.\]

\subsection{比值审敛法(拉朗贝尔判别法)}

\[\lim\frac{u_{n+1}}{u_n}
    \left\{\begin{aligned}
         & <1 & \implies & \sum u_n\text{收敛}             \\
         & =1 & \implies & \sum u_n\text{可能收敛可能发散} \\
         & >1 & \implies & \sum u_n\text{发散}
    \end{aligned}\right.\]

\subsection{根值审敛法(柯西判别法)}

\[\lim\sqrt[n]{u_n}
    \left\{\begin{aligned}
         & <1 & \implies & \sum u_n\text{收敛}             \\
         & =1 & \implies & \sum u_n\text{可能收敛可能发散} \\
         & >1 & \implies & \sum u_n\text{发散}
    \end{aligned}\right.\]

\section{交错级数}

\subsection{莱布尼兹判别法}

\[\left.\begin{aligned}
        \text{正项级数}u_n\searrow \\
        \lim_{n\to\infty}u_n=0
    \end{aligned}\right\}
    \implies
    \text{交错级数}\sum\p{-1}^n\text{(或}\sum\p{-1}^{n-1}u_n\text{)收敛}\]

\section{幂(泰勒)级数}

\begin{definition}[以下默认幂级数形式]

    \[\sum_{n\in\sn}=a_nx^n,\text{收敛半径为}R\]

\end{definition}

\subsection{阿贝尔定理}

\[\left\{\begin{aligned}
        x\in & \p{-R,R}                        & \text{收敛}     \\
        x\in & \left\{-R,R\right\}             & \text{单独讨论} \\
        x\in & \p{-\infty,-R}\cup\p{R,+\infty} & \text{发散}
    \end{aligned}\right.\]

\subsection{系数模比值法}

\[R^{-1}=\rho=\lim_{n\to\infty}\abs{\frac{a_{n+1}}{a_n}}\]

\subsection{系数模根值法}

\[R^{-1}=\rho=\lim_{n\to\infty}\sqrt[n]{\abs{a_n}}\]

\subsection{加减运算}

\[\left.\begin{aligned}
        \sum a_nx^n\text{收敛域为}I_a \\
        \sum b_nx^n\text{收敛域为}I_b
    \end{aligned}\right\}
    \implies\sum a_nx^n\pm\sum b_nx^n=\sum\p{a_n\pm b_n}x^n,x\in I_a\cap I_b\]

\subsection{泰勒级数}

\[f\p{x}\sim
    \sum_{n\in\sn}\frac{f^{\p{n}}\p{x_0}}{n!}\p{x-x_0}^n\]

\subsection{常用泰勒级数}

\[\newcommand{\series}[1]{\sum_{n\in#1}}
    \begin{aligned}
        \e^x           & =\series\sn\frac{x^n}{n!}                                                                              \\
        \sin x         & =\series\sn\p{-1}^n\frac{x^{2n+1}}{\p{2n+1}!}                                                          \\
        \cos x         & =\series\sn\p{-1}^n\frac{x^{2n}}{\p{2n}!}                                                              \\
        \ln\p{1+x}     & =\series\snp\p{-1}^{n-1}\frac{x^n}n                    & x\in\left(-1,1\right]                         \\
        \ln\p{1-x}     & =\series\snp-\frac{x^n}n                               & x\in\left(-1,1\right]                         \\
        \frac1{1+x}    & =\series\sn\p{-1}^{n-1}x^n                             & x\in\p{-1,1}                                  \\
        \frac1{1-x}    & =\series\sn x^n                                        & x\in\p{-1,1}                                  \\
        \p{1+x}^\alpha & =\series\sn\frac{\prod_{i=0}^{n-1}\p{\alpha-i}}{n!}x^n & \text{此处定义}\prod_{i=0}^{-1}\p{\alpha-i}=1
    \end{aligned}\]

\section{三角(傅里叶)级数}

\subsection{傅里叶级数}

\[f\p{x}\sim
    \frac{a_0}2+\sum_{n\in\snp}\p{a_n\cos nx+b_n\sin nx}\]

$f\p{x}$周期为$2l$时($l$常取$\pi$),有傅里叶系数:

\[\left\{\begin{aligned}
        a_n & =\frac1l\int_{-l}^lf\p{x}\cos\frac{n\pi x}l\d x,n\in\sn  \\
        b_n & =\frac1l\int_{-l}^lf\p{x}\sin\frac{n\pi x}l\d x,n\in\snp
    \end{aligned}\right.\]

\[T=2l\]

\[\omega=\frac{2\pi}T=\frac\pi l\]

\subsection{狄利克雷收敛定理}

$f\p{x}$在一个周期内有:

\begin{enumerate}
    \item 连续或只有有限个第一类间断点
    \item 只有有限个极值点
\end{enumerate}

即$f\p{x}$的傅里叶级数在$\sr$连续,且

\begin{enumerate}
    \item $x_0$连续时,级数收敛于$f(x_0)$
    \item $x_0$是第一类间断点时,级数收敛于$\frac{f(x_0^-)+f(x_0^+)}2$
\end{enumerate}

\end{document}
